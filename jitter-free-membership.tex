
%% bare_jrnl.tex
%% V1.4b
%% 2015/08/26
%% by Michael Shell
%% see http://www.michaelshell.org/
%% for current contact information.
%%
%% This is a skeleton file demonstrating the use of IEEEtran.cls
%% (requires IEEEtran.cls version 1.8b or later) with an IEEE
%% journal paper.
%%
%% Support sites:
%% http://www.michaelshell.org/tex/ieeetran/
%% http://www.ctan.org/pkg/ieeetran
%% and
%% http://www.ieee.org/

%%*************************************************************************
%% Legal Notice:
%% This code is offered as-is without any warranty either expressed or
%% implied; without even the implied warranty of MERCHANTABILITY or
%% FITNESS FOR A PARTICULAR PURPOSE! 
%% User assumes all risk.
%% In no event shall the IEEE or any contributor to this code be liable for
%% any damages or losses, including, but not limited to, incidental,
%% consequential, or any other damages, resulting from the use or misuse
%% of any information contained here.
%%
%% All comments are the opinions of their respective authors and are not
%% necessarily endorsed by the IEEE.
%%
%% This work is distributed under the LaTeX Project Public License (LPPL)
%% ( http://www.latex-project.org/ ) version 1.3, and may be freely used,
%% distributed and modified. A copy of the LPPL, version 1.3, is included
%% in the base LaTeX documentation of all distributions of LaTeX released
%% 2003/12/01 or later.
%% Retain all contribution notices and credits.
%% ** Modified files should be clearly indicated as such, including  **
%% ** renaming them and changing author support contact information. **
%%*************************************************************************


% *** Authors should verify (and, if needed, correct) their LaTeX system  ***
% *** with the testflow diagnostic prior to trusting their LaTeX platform ***
% *** with production work. The IEEE's font choices and paper sizes can   ***
% *** trigger bugs that do not appear when using other class files.       ***                          ***
% The testflow support page is at:
% http://www.michaelshell.org/tex/testflow/



\documentclass[journal]{IEEEtran}
%\usepackage[paperwidth=310mm,paperheight=200mm]{geometry} % TODO just for on-screen
%
% If IEEEtran.cls has not been installed into the LaTeX system files,
% manually specify the path to it like:
% \documentclass[journal]{../sty/IEEEtran}





% Some very useful LaTeX packages include:
% (uncomment the ones you want to load)


% *** MISC UTILITY PACKAGES ***
%
%\usepackage{ifpdf}
% Heiko Oberdiek's ifpdf.sty is very useful if you need conditional
% compilation based on whether the output is pdf or dvi.
% usage:
% \ifpdf
%   % pdf code
% \else
%   % dvi code
% \fi
% The latest version of ifpdf.sty can be obtained from:
% http://www.ctan.org/pkg/ifpdf
% Also, note that IEEEtran.cls V1.7 and later provides a builtin
% \ifCLASSINFOpdf conditional that works the same way.
% When switching from latex to pdflatex and vice-versa, the compiler may
% have to be run twice to clear warning/error messages.





% *** CITATION PACKAGES ***
%
%\usepackage{cite}
% cite.sty was written by Donald Arseneau
% V1.6 and later of IEEEtran pre-defines the format of the cite.sty package
% \cite{} output to follow that of the IEEE. Loading the cite package will
% result in citation numbers being automatically sorted and properly
% "compressed/ranged". e.g., [1], [9], [2], [7], [5], [6] without using
% cite.sty will become [1], [2], [5]--[7], [9] using cite.sty. cite.sty's
% \cite will automatically add leading space, if needed. Use cite.sty's
% noadjust option (cite.sty V3.8 and later) if you want to turn this off
% such as if a citation ever needs to be enclosed in parenthesis.
% cite.sty is already installed on most LaTeX systems. Be sure and use
% version 5.0 (2009-03-20) and later if using hyperref.sty.
% The latest version can be obtained at:
% http://www.ctan.org/pkg/cite
% The documentation is contained in the cite.sty file itself.






% *** GRAPHICS RELATED PACKAGES ***
%
\ifCLASSINFOpdf
  \usepackage[pdftex]{graphicx}
  % declare the path(s) where your graphic files are
  % \graphicspath{{../pdf/}{../jpeg/}}
  % and their extensions so you won't have to specify these with
  % every instance of \includegraphics
  % \DeclareGraphicsExtensions{.pdf,.jpeg,.png}
\else
  % or other class option (dvipsone, dvipdf, if not using dvips). graphicx
  % will default to the driver specified in the system graphics.cfg if no
  % driver is specified.
  % \usepackage[dvips]{graphicx}
  % declare the path(s) where your graphic files are
  % \graphicspath{{../eps/}}
  % and their extensions so you won't have to specify these with
  % every instance of \includegraphics
  % \DeclareGraphicsExtensions{.eps}
\fi
% graphicx was written by David Carlisle and Sebastian Rahtz. It is
% required if you want graphics, photos, etc. graphicx.sty is already
% installed on most LaTeX systems. The latest version and documentation
% can be obtained at: 
% http://www.ctan.org/pkg/graphicx
% Another good source of documentation is "Using Imported Graphics in
% LaTeX2e" by Keith Reckdahl which can be found at:
% http://www.ctan.org/pkg/epslatex
%
% latex, and pdflatex in dvi mode, support graphics in encapsulated
% postscript (.eps) format. pdflatex in pdf mode supports graphics
% in .pdf, .jpeg, .png and .mps (metapost) formats. Users should ensure
% that all non-photo figures use a vector format (.eps, .pdf, .mps) and
% not a bitmapped formats (.jpeg, .png). The IEEE frowns on bitmapped formats
% which can result in "jaggedy"/blurry rendering of lines and letters as
% well as large increases in file sizes.
%
% You can find documentation about the pdfTeX application at:
% http://www.tug.org/applications/pdftex





% *** MATH PACKAGES ***
%
\usepackage{amsmath}
\usepackage{amssymb}

\usepackage{amsthm}
\newtheorem{theorem}{Theorem}
\newtheorem{lemma}[theorem]{Lemma}
\newtheorem{corollary}[theorem]{Corollary}
% A popular package from the American Mathematical Society that provides
% many useful and powerful commands for dealing with mathematics.
%
% Note that the amsmath package sets \interdisplaylinepenalty to 10000
% thus preventing page breaks from occurring within multiline equations. Use:
%\interdisplaylinepenalty=2500
% after loading amsmath to restore such page breaks as IEEEtran.cls normally
% does. amsmath.sty is already installed on most LaTeX systems. The latest
% version and documentation can be obtained at:
% http://www.ctan.org/pkg/amsmath





% *** SPECIALIZED LIST PACKAGES ***
%
%\usepackage{algorithmic}
% algorithmic.sty was written by Peter Williams and Rogerio Brito.
% This package provides an algorithmic environment fo describing algorithms.
% You can use the algorithmic environment in-text or within a figure
% environment to provide for a floating algorithm. Do NOT use the algorithm
% floating environment provided by algorithm.sty (by the same authors) or
% algorithm2e.sty (by Christophe Fiorio) as the IEEE does not use dedicated
% algorithm float types and packages that provide these will not provide
% correct IEEE style captions. The latest version and documentation of
% algorithmic.sty can be obtained at:
% http://www.ctan.org/pkg/algorithms
% Also of interest may be the (relatively newer and more customizable)
% algorithmicx.sty package by Szasz Janos:
% http://www.ctan.org/pkg/algorithmicx




% *** ALIGNMENT PACKAGES ***
%
%\usepackage{array}
% Frank Mittelbach's and David Carlisle's array.sty patches and improves
% the standard LaTeX2e array and tabular environments to provide better
% appearance and additional user controls. As the default LaTeX2e table
% generation code is lacking to the point of almost being broken with
% respect to the quality of the end results, all users are strongly
% advised to use an enhanced (at the very least that provided by array.sty)
% set of table tools. array.sty is already installed on most systems. The
% latest version and documentation can be obtained at:
% http://www.ctan.org/pkg/array


% IEEEtran contains the IEEEeqnarray family of commands that can be used to
% generate multiline equations as well as matrices, tables, etc., of high
% quality.




% *** SUBFIGURE PACKAGES ***
%\ifCLASSOPTIONcompsoc
%  \usepackage[caption=false,font=normalsize,labelfont=sf,textfont=sf]{subfig}
%\else
%  \usepackage[caption=false,font=footnotesize]{subfig}
%\fi
% subfig.sty, written by Steven Douglas Cochran, is the modern replacement
% for subfigure.sty, the latter of which is no longer maintained and is
% incompatible with some LaTeX packages including fixltx2e. However,
% subfig.sty requires and automatically loads Axel Sommerfeldt's caption.sty
% which will override IEEEtran.cls' handling of captions and this will result
% in non-IEEE style figure/table captions. To prevent this problem, be sure
% and invoke subfig.sty's "caption=false" package option (available since
% subfig.sty version 1.3, 2005/06/28) as this is will preserve IEEEtran.cls
% handling of captions.
% Note that the Computer Society format requires a larger sans serif font
% than the serif footnote size font used in traditional IEEE formatting
% and thus the need to invoke different subfig.sty package options depending
% on whether compsoc mode has been enabled.
%
% The latest version and documentation of subfig.sty can be obtained at:
% http://www.ctan.org/pkg/subfig




% *** FLOAT PACKAGES ***
%
%\usepackage{fixltx2e}
% fixltx2e, the successor to the earlier fix2col.sty, was written by
% Frank Mittelbach and David Carlisle. This package corrects a few problems
% in the LaTeX2e kernel, the most notable of which is that in current
% LaTeX2e releases, the ordering of single and double column floats is not
% guaranteed to be preserved. Thus, an unpatched LaTeX2e can allow a
% single column figure to be placed prior to an earlier double column
% figure.
% Be aware that LaTeX2e kernels dated 2015 and later have fixltx2e.sty's
% corrections already built into the system in which case a warning will
% be issued if an attempt is made to load fixltx2e.sty as it is no longer
% needed.
% The latest version and documentation can be found at:
% http://www.ctan.org/pkg/fixltx2e


%\usepackage{stfloats}
% stfloats.sty was written by Sigitas Tolusis. This package gives LaTeX2e
% the ability to do double column floats at the bottom of the page as well
% as the top. (e.g., "\begin{figure*}[!b]" is not normally possible in
% LaTeX2e). It also provides a command:
%\fnbelowfloat
% to enable the placement of footnotes below bottom floats (the standard
% LaTeX2e kernel puts them above bottom floats). This is an invasive package
% which rewrites many portions of the LaTeX2e float routines. It may not work
% with other packages that modify the LaTeX2e float routines. The latest
% version and documentation can be obtained at:
% http://www.ctan.org/pkg/stfloats
% Do not use the stfloats baselinefloat ability as the IEEE does not allow
% \baselineskip to stretch. Authors submitting work to the IEEE should note
% that the IEEE rarely uses double column equations and that authors should try
% to avoid such use. Do not be tempted to use the cuted.sty or midfloat.sty
% packages (also by Sigitas Tolusis) as the IEEE does not format its papers in
% such ways.
% Do not attempt to use stfloats with fixltx2e as they are incompatible.
% Instead, use Morten Hogholm'a dblfloatfix which combines the features
% of both fixltx2e and stfloats:
%
% \usepackage{dblfloatfix}
% The latest version can be found at:
% http://www.ctan.org/pkg/dblfloatfix




%\ifCLASSOPTIONcaptionsoff
%  \usepackage[nomarkers]{endfloat}
% \let\MYoriglatexcaption\caption
% \renewcommand{\caption}[2][\relax]{\MYoriglatexcaption[#2]{#2}}
%\fi
% endfloat.sty was written by James Darrell McCauley, Jeff Goldberg and 
% Axel Sommerfeldt. This package may be useful when used in conjunction with 
% IEEEtran.cls'  captionsoff option. Some IEEE journals/societies require that
% submissions have lists of figures/tables at the end of the paper and that
% figures/tables without any captions are placed on a page by themselves at
% the end of the document. If needed, the draftcls IEEEtran class option or
% \CLASSINPUTbaselinestretch interface can be used to increase the line
% spacing as well. Be sure and use the nomarkers option of endfloat to
% prevent endfloat from "marking" where the figures would have been placed
% in the text. The two hack lines of code above are a slight modification of
% that suggested by in the endfloat docs (section 8.4.1) to ensure that
% the full captions always appear in the list of figures/tables - even if
% the user used the short optional argument of \caption[]{}.
% IEEE papers do not typically make use of \caption[]'s optional argument,
% so this should not be an issue. A similar trick can be used to disable
% captions of packages such as subfig.sty that lack options to turn off
% the subcaptions:
% For subfig.sty:
% \let\MYorigsubfloat\subfloat
% \renewcommand{\subfloat}[2][\relax]{\MYorigsubfloat[]{#2}}
% However, the above trick will not work if both optional arguments of
% the \subfloat command are used. Furthermore, there needs to be a
% description of each subfigure *somewhere* and endfloat does not add
% subfigure captions to its list of figures. Thus, the best approach is to
% avoid the use of subfigure captions (many IEEE journals avoid them anyway)
% and instead reference/explain all the subfigures within the main caption.
% The latest version of endfloat.sty and its documentation can obtained at:
% http://www.ctan.org/pkg/endfloat
%
% The IEEEtran \ifCLASSOPTIONcaptionsoff conditional can also be used
% later in the document, say, to conditionally put the References on a 
% page by themselves.




% *** PDF, URL AND HYPERLINK PACKAGES ***
%
%\usepackage{url}
% url.sty was written by Donald Arseneau. It provides better support for
% handling and breaking URLs. url.sty is already installed on most LaTeX
% systems. The latest version and documentation can be obtained at:
% http://www.ctan.org/pkg/url
% Basically, \url{my_url_here}.




% *** Do not adjust lengths that control margins, column widths, etc. ***
% *** Do not use packages that alter fonts (such as pslatex).         ***
% There should be no need to do such things with IEEEtran.cls V1.6 and later.
% (Unless specifically asked to do so by the journal or conference you plan
% to submit to, of course. )


% correct bad hyphenation here
\hyphenation{op-tical net-works semi-conduc-tor}


\begin{document}
%
% paper title
% Titles are generally capitalized except for words such as a, an, and, as,
% at, but, by, for, in, nor, of, on, or, the, to and up, which are usually
% not capitalized unless they are the first or last word of the title.
% Linebreaks \\ can be used within to get better formatting as desired.
% Do not put math or special symbols in the title.
\title{Jitter-free Dynamic Paxos}
%
%
% author names and IEEE memberships
% note positions of commas and nonbreaking spaces ( ~ ) LaTeX will not break
% a structure at a ~ so this keeps an author's name from being broken across
% two lines.
% use \thanks{} to gain access to the first footnote area
% a separate \thanks must be used for each paragraph as LaTeX2e's \thanks
% was not built to handle multiple paragraphs
%

\author{David C. Turner% <-this % stops a space
\thanks{D. Turner is with the Operations and Planning Systems division of Tracsis plc
}% <-this % stops a space
\thanks{Manuscript received April 19, 2005; revised August 26, 2015.}}

% note the % following the last \IEEEmembership and also \thanks - 
% these prevent an unwanted space from occurring between the last author name
% and the end of the author line. i.e., if you had this:
% 
% \author{....lastname \thanks{...} \thanks{...} }
%                     ^------------^------------^----Do not want these spaces!
%
% a space would be appended to the last name and could cause every name on that
% line to be shifted left slightly. This is one of those "LaTeX things". For
% instance, "\textbf{A} \textbf{B}" will typeset as "A B" not "AB". To get
% "AB" then you have to do: "\textbf{A}\textbf{B}"
% \thanks is no different in this regard, so shield the last } of each \thanks
% that ends a line with a % and do not let a space in before the next \thanks.
% Spaces after \IEEEmembership other than the last one are OK (and needed) as
% you are supposed to have spaces between the names. For what it is worth,
% this is a minor point as most people would not even notice if the said evil
% space somehow managed to creep in.



% The paper headers
\markboth{Journal of \LaTeX\ Class Files,~Vol.~14, No.~8, August~2015}%
{Turner: Jitter-free Dynamic Paxos}
% The only time the second header will appear is for the odd numbered pages
% after the title page when using the twoside option.
% 
% *** Note that you probably will NOT want to include the author's ***
% *** name in the headers of peer review papers.                   ***
% You can use \ifCLASSOPTIONpeerreview for conditional compilation here if
% you desire.




% If you want to put a publisher's ID mark on the page you can do it like
% this:
%\IEEEpubid{0000--0000/00\$00.00~\copyright~2015 IEEE}
% Remember, if you use this you must call \IEEEpubidadjcol in the second
% column for its text to clear the IEEEpubid mark.



% use for special paper notices
%\IEEEspecialpapernotice{(Invited Paper)}




% make the title area
\maketitle

% As a general rule, do not put math, special symbols or citations
% in the abstract or keywords.
\begin{abstract}
The abstract goes here. %TODO
\end{abstract}

% Note that keywords are not normally used for peerreview papers.
%\begin{IEEEkeywords} TODO???
%IEEE, IEEEtran, journal, \LaTeX, paper, template.
%\end{IEEEkeywords}






% For peer review papers, you can put extra information on the cover
% page as needed:
% \ifCLASSOPTIONpeerreview
% \begin{center} \bfseries EDICS Category: 3-BBND \end{center}
% \fi
%
% For peerreview papers, this IEEEtran command inserts a page break and
% creates the second title. It will be ignored for other modes.
\IEEEpeerreviewmaketitle

\section{Introduction}

\IEEEPARstart{P}{axos} is an algorithm for achieving consensus on (or
\textit{choosing}) a sequence $v_0, v_1, \ldots$ of values in a distributed,
fault-tolerant fashion. It guarantees system-wide \textit{consistency}, in the
sense that the values chosen are guaranteed to be equal across the whole system
even in the presence of failures. It typically runs on a cluster of $2f+1$
nodes, where $f$ is the number of faulty nodes that should be tolerated, and
consensus is achieved on a value $v_i \in \mathbb V$ when $\ge f+1$ (i.e.  a
majority) of the nodes agree.  More generally, if $\mathbb A$ is the set of all
nodes then there is a set $Q \subseteq \mathcal P \mathbb A$ of
\textit{quorums}, and a value is chosen when all nodes $a \in q$ agree for some
quorum $q \in Q$. The set of quorums is called the \textit{membership} of the
cluster.

It is normally necessary to be able to add or remove nodes from the cluster,
which amounts to the ability to change the membership while the cluster is
running. It is crucial that all participating nodes agree on the cluster
membership at all times, and Dynamic Paxos\footnote{TODO cite} achieves this by
tracking the cluster membership with a state machine whose transitions occur
when particular values in the sequence are chosen.

Each value $v_i \in \mathbb V$ is chosen using a separate instance $i$ of a
two-phase consensus protocol known as \textit{Synod}. The full Paxos algorithm
essentially runs an infinite sequence of Synod protocols in parallel: it starts
by running phase I of all instances at once and then runs phase II of each
instance in turn to yield the desired sequence of chosen values. It normally
remains in phase II for extended periods of time, but will return to phase I if
certain nodes become faulty, or if messages between certain pairs of nodes
cease to be delivered reliably for a period, or if the membership changes.

Paxos allows for \textit{pipelining} in phase II: later instances can be
started before earlier ones have been chosen, which means that the rate of
choosing values is not limited by network latency. The pipeline runs smoothly
as long as phase I does not need to run again, but may stall if phase I is
re-run. A pipeline stall can be observed as jitter in the rate at which values
are chosen.

It is unavoidable that a faulty node or disrupted communications may have an
impact on the system's performance, but changing its membership is a routine
task which should have no such effect. Dynamic Paxos\footnote{TODO cite} avoids
a pipeline stall when a cluster membership change is chosen at instance $i$ by
postponing the effects of the change until instance $i + \alpha$ for some
$\alpha > 0$. If $\alpha$ is large enough, there is time to run the new phase I
before the pipeline stalls.  On the other hand, if $\alpha$ is too large then
the membership change can take an unreasonably long time to take effect.

The value of the delay parameter $\alpha$ may be allowed to vary while the
system is running, so it is certainly possible to design a mechanism for
choosing a reasonably suitable value and adapting it to changing system
conditions. The approach described here shows an alternative technique which
allows phase II to start using a new membership without re-running phase I
first; this is shown to preserve consistency as long as the membership change
is sufficiently small (in a sense which is made precise in section
\ref{types-of-membership-change} below.) By removing phase I from the critical
path of a membership change this technique eliminates the possibility of jitter
caused by a pipeline stall.

The technique shown here also supports arbitrary membership changes happening
in a single step just as in Dynamic Paxos, but any changes that are not
sufficiently small continue to carry the risk of jitter. In reality, arbitrary
changes would normally be performed as a sequence of these small, jitter-free
changes.

\section{A Review of Paxos}

The Paxos algorithm was introduced in TODO. It is briefly reviewed here for the
sake of consistency of notation and to introduce some generalisations which are
discussed in more detail in section \ref{deviations-from-classic}.

\subsection{The Synod algorithm}

Synod is an algorithm for achieving consensus on a single value in a
distributed system comprising a set of nodes which can communicate by sending
messages to each other. Communication is asynchronous but not Byzantine, in the
sense that each messages may be delayed, reordered, duplicated and dropped but
not corrupted.

Let $\mathbb P$ be a set of \textit{proposal identifiers} with a wellfounded
total order $\prec$. Let $\mathbb A$ be a set of \textit{node identifiers} and
let $\mathbb V$ be the set of values that may be chosen.

\def\prep#1{\mathbf{prepare}(#1)}
\def\mprom#1#2#3{\mathbf{promised}_{\ge #1}(#2,#3)}
\def\fprom#1#2#3{\mathbf{promised}_{#1}(#2,#3)}
\def\bprom#1#2#3#4{\mathbf{promised}_{#1}(#2,#3;#4)}
\def\prop#1#2{\mathbf{proposed}_{#1}(#2)}
\def\acc#1#2#3{\mathbf{accepted}_{#1}(#2,#3)}
\def\chosen#1#2{\mathbf{chosen}_{#1}(#2)}
\def\owner#1{\mathrm{owner}(#1)}

The Synod algorithm involves four kinds of message, described below.
Throughout, $p, p' \in \mathbb P$ and $a \in \mathbb A$.  Phase I starts with
the broadcast of a \textit{prepare} message $\prep{p}$ to which each node $a$
may respond with a \textit{promise} message, either a \textit{free promise}
written $\fprom{}{a}{p}$ or a \textit{forced promise} written
$\bprom{}{a}{p}{p'}$.  Phase II starts with the broadcast of a
\textit{proposal} message $\prop{}{p}$ to which each node $a$ may respond with
an \textit{acceptance} message $\acc{}{a}{p}$.

It is convenient to treat these symbols as predicates indicating whether the
corresponding messages have been sent. For instance, the predicate
$\acc{}{a}{p}$ indicates that the message $\acc{}{a}{p}$ has been sent and
$\neg \prop{}{p}$ indicates that the message $\prop{}{p}$ has not been sent.
Because messages may be delayed or dropped, a node can only know a message has
not been sent if it can prove, based solely on the messages it alone has sent
and received, that no other node has sent it. This works in practice because
the messages $\fprom{}{a}{p}$, $\bprom{}{a}{p}{p'}$ and $\acc{}{a}{p}$ may only
originate from the node $a$, and there is a function $\mathrm{owner} : \mathbb
P \to \mathbb A$ assigning an owner to each proposal such that the message
$\prop{}{p}$ may only originate from node $\owner{p}$.

It is convenient to add another predicate $\chosen{}{p}$ which indicates that
some node has chosen a proposal $p$. There is also a function $v : \mathbb P
\to \mathbb V$ assigning a value to each proposal, and the system satisfies a
set of invariants as described in appendix \ref{synod-safety} from which can be
shown the key consistency property of the Synod algorithm: \[\textrm{if }
\chosen{}{p} \textrm{ and } \chosen{}{p'} \textrm { then } v(p) = v(p').\]

Each node operates as a state machine whose transitions are caused by the
receipts of messages. A node may emit $\prop{}{p}$ when it considers phase I to
be complete at proposal $p$, and similarly may emit $\chosen{}{p}$ when phase
II is complete.  A node considers phase I to be complete at proposal $p$ when
promise messages for $p$ have been received from a quorum of nodes, and
similarly considers phase II to be complete at $p$ when a quorum of acceptances
of $p$ has been received. The quorums themselves can depend on the proposal
number $p$ and the phase, but for any $p$ it must be the case that there is at
least one node that is in a quorum of each phase.

\subsection{The Paxos algorithm}

Conceptually, Paxos is a sequence of distinct instances of the Synod algorithm
all running simultaneously. To achieve this, the messages of the Synod
algorithm above are indexed with the instance number $i \in \mathbb N$:
$\fprom{i}{a}{p}$, $\bprom{i}{a}{p}{p'}$, $\prop{i}{p}$ and $\acc{i}{a}{p}$.
There is another kind of message known as a \textit{multi-promise}, written
$\mprom{i}{a}{p}$, which can be thought of as standing for the infinite set of
free promises ${\{ \fprom{j}{a}{p} \mid j \ge i \}}$. The predicate
$\chosen{i}{p}$ is also indexed by the instance number $i$, but in this
presentation prepare messages $\prep{p}$ apply to all instances so do not need
to be indexed.

Recall that in each Synod instance each node is a state machine whose
transitions are caused by the receipt of messages. In Paxos, let $S_i$ be the
state machine for instance $i$. Because at any time each node has only ever
received finitely many messages, there must be an instance number $i_\infty$
such that all instances $i \ge i_\infty$ are in the same state. A Paxos node
therefore needs to have a separate state machine $S_i$ for each instance $i <
i_\infty$ but then just a single state machine $S_\infty$ that represents all
instances $i \ge i_\infty$. When a message mentioning an instance $i \ge
i_\infty$ is received, the node must increase $i_\infty$ to $i'_\infty > i$ and
create (or \textit{spawn}) state machines $S_j$ for $i_\infty \le j <
i'_\infty$, all of which start in the same state as $S_\infty$. Typically, when
all the instances up to $i$ are chosen at a node then this node ignores any
further messages about these instances and discards the corresponding state
machines.

As in the Synod algorithm, phase I starts with a broadcast of $\prep{p}$ for
some $p$ to which each node $a$ may respond with a set of promises
$\fprom{i}{a}{p}$, $\bprom{i}{a}{p}{p'}$ and $\mprom{i}{a}{p}$ according to the
state of its state machines. If a node receives $\mprom{i}{a}{p}$ messages from
a quorum of nodes then its state machine $S_\infty$ has completed phase I at
$p$, and therefore any state machines that are subsequently spawned will also
have completed phase I at $p$. Each phase II instance $i$ operates just as in
the Synod algorithm, starting with a broadcast of $\prop{i}{p}$ to which each
node $a$ may respond with an acceptance $\acc{i}{a}{p}$ and finally
$\chosen{i}{p}$ holds once acceptances have been received from a quorum of
nodes.

There is a function $v_i : \mathbb P \to \mathbb V$ for each instance $i$
defining the value for each proposal, and theorem \ref{paxos-safety-theorem}
shows that whenever $\chosen{i}{p}$ and $\chosen{i}{p'}$ it follows that
$v_i(p) = v_i(p')$.

\section{Differences from classic Paxos} \label{deviations-from-classic}

\subsection{Value function}

In the classic presentation of the Synod algorithm, the values of proposals are
carried along with their identifiers in the messages $\bprom{}{a}{p}{ p',v(p')
}$, $\prop{}{ p,v(p)}$ and $\acc{}{a}{ p,v(p) }$. Observation O4 in
TODO\footnote{Cheap Paxos} notes that these values can be replaced in some
cases by hashes, which is useful if the values themselves are expensive to
transmit between nodes. The presentation here goes one step further and shows
that the values can be completely elided from the messages that take part in
the Synod protocol, allowing considerably more freedom in the implementation of
the function $v$ without sacrificing consistency.  This works because if the
invariants of appendix \ref{synod-safety} are satisfied with a value function
$v$ then they are satisfied for any value function $v'$ that agrees with $v$ on
sent proposals, i.e. having $v(p) = v'(p)$ whenever $\prop{}{p}$.  Furthermore
only $\owner{p}$ may propose $p$, so if it has not yet proposed a value for $p$
it can, in effect, freely modify $v(p)$ as desired.

To ensure the system satisfies the desired liveness properties, the
implementation of $v$ must be resilient to the same failure modes as the rest
of the system. In these terms the classic presentation ensures this resilience
by including pertinent values of $v$ in all relevant messages, but since the
value of each proposal $p$ has at most one writer, $\owner{p}$, more efficient
mechanisms are available. For instance, the values of proposed proposals may be
represented as an insert-only set of pairs $\{ \langle p, v(p) \rangle \mid
\prop{}{p} \}$ which can be implemented as a simple conflict-free replicated
data type or CRDT\footnote{TODO cite}.

\subsection{Per-phase quorums}\label{per-phase-quorums}

The heart of the consistency property of the Synod algorithm is that the set of
nodes involved in completing phase I must always intersect the set of nodes
involved in completing phase II, so that there is at least one node to carry
information between the two phases. Write $Q_1 \smile Q_2$ to indicate that
every element of $Q_1$ intersects every member of $Q_2$: more formally $Q_1
\smile Q_2$ if every $q_1 \in Q_1$ and $q_2 \in Q_2$ have ${q_1 \cap q_2 \ne
\varnothing}$.

In the original presentation of the Synod algorithm any (weighted) majority of
the nodes could be used as a quorum: roughly speaking, $Q^\textrm{I}(p) =
Q^\textrm{II}(p) = Q_\mathrm{maj}$ was a constant, and any two majorities
must intersect: $Q_\mathrm{maj} \smile
Q_\mathrm{maj}$.

The ability to vary the set of quorums by proposal and by phase is the key
ingredient in being able to perform jitter-free membership changes. The proof
of theorem \ref{synod-safety-theorem} in appendix \ref{synod-safety} is a
generalisation of the original one to demonstrate that consistency is preserved
even when $Q^\textrm{I}(p_1) \ne Q^\textrm{II}(p_2)$ as long as
$Q^\textrm{I}(p_1) \smile Q^\textrm{II}(p_2)$ for all proposals $p_1 \succ p_2$
having $\prop{}{p_1}$ and $\chosen{}{p_2}$.

\section{Membership changes}\label{membership-changes}

Changes in the cluster membership are handled with a sequence of numbered
\textit{epochs} $1, 2, \ldots$ and memberships $Q_0, Q_1, Q_2, \ldots$
satisfying $Q_e \smile Q_e$ and $Q_e \smile Q_{e-1}$ for all epochs $e$.  Taken
together, these conditions mean that $Q_{e-1} \cup Q_e \smile Q_e$ which means
that $Q_{e-1} \cup Q_e$ and $Q_e$ can be respectively used for the phase-I and
phase-II memberships of each Paxos instance by the observation of
\ref{per-phase-quorums} above.  There are also two nondecreasing functions,
both written $e(\cdot)$, which respectively assign an epoch number $e(i)$ to
each instance $i$, and an epoch number $e(p)$ to each proposal $p$. A node may
emit $\prop{i}{p}$ if $e(p) \le e(i)$ and it has received promises for proposal
$p$ in instance $i$ from a quorum $q^\textrm{I} \in Q_{e(p)}$. A node considers
phase II to be complete at instance $i$ and proposal $p$ if ${e(i) \le e(p)+1}$
and it has received $\acc{i}{a}{p}$ from a quorum ${q^\textrm{II} \in
Q_{e(i)}}$.  Thus if $\chosen{i}{p}$ then ${e(p) \in \{ e(i)-1, e(i) \}}$ and
hence \[q^\textrm{I} \in Q_{e(p)} \subseteq Q_{e(i)-1} \cup Q_{e(i)} \smile
Q_{e(i)} \ni q^\textrm{II}\] so that $q^\textrm{I} \cap q^\textrm{II} \ne
\varnothing$ as required. It is worth emphasising the subtle point that phase I
uses the membership $Q_{e(p)}$ corresponding to the proposal $p$ whereas phase
II uses the membership $Q_{e(i)}$ for the instance $i$.

\subsection{Dynamic membership changes}

As in Dynamic Paxos, the memberships $Q_0, Q_1, \ldots$ and epoch numbers
$e(i)$ are themselves chosen by Paxos. In more detail, memberships are tracked
by a state machine whose state is the finite sequence $Q_0, \ldots,
Q_{e_\mathrm{max}}$ and the epoch numbers of instances are tracked by a state
machine whose state is the finite sequence $e(0), \ldots, e(i_\mathrm{max})$.
Their transitions append one or more elements to these sequences, but do not
change any existing elements, and the tracked sequences are always long enough
to make progress: if $\chosen{i}{p}$ then the epoch-number state machine
satisfies $i_\mathrm{max} > i+1$ and $e_\mathrm{max} \ge e(i_\mathrm{max})$ in
the membership-sequence state machine. Epoch numbers $e(p)$ for proposals $p$
are fixed in advance and not chosen by Paxos.

Note that in order to check whether phase I is complete for a proposal $p$ at
an instance $i$ a node must check whether it has received promises from a
quorum of nodes $q \in Q_{e(p)}$, for which it needs to know $Q_{e(p)}$.  Nodes
should therefore not emit $\prep{p}$ messages for proposals for which
$Q_{e(p)}$ is not known. The node must also check that $e(p) \le e(i)$; if
$e(i)$ is not known then $i_\mathrm{max} < i$, but $e$ is an increasing
function so $e(i_\mathrm{max}) \le e(i)$ and it is therefore sufficient to
check that $e(p) \le e(i_\mathrm{max})$.

Similarly, in order to check whether phase II is complete for a proposal $p$ at
an instance $i$ a node must check whether it has received acceptances from a
quorum of nodes $q \in Q_{e(i)}$ and must therefore know the value of
$Q_{e(i)}$. This will be known if the node has chosen the previous instance,
because if $\chosen{i-1}{p'}$ then $i < i_\mathrm{max}$ and hence ${e(i) \le
e(i_\mathrm{max}) \le e_\mathrm{max}}$ as required.

\subsection{Allowing for membership changes...}

TODO do this the same, but better

To see that this allows for membership changes without requiring a new phase I
to run, let $i_0$ be a fixed instance and suppose that the node $\owner{p}$ has
received a quorum of promises $q^\textrm{I} \in Q_{e(p)}$ for proposal $p$ for
all instances $i \ge i_0$ and suppose that $e(i_0) = e(p)$. The node can
therefore send $\prop{i}{p}$ for any $i \ge i_0$ and will consider phase II to
be complete at instance $i$ when it receives acceptances from a quorum
$q^\textrm{II} \in Q_{e(i)}$ as long as $e(i) \le e(p) + 1$. Thus it can
complete phase II for all instances in the same epoch as $i_0$ and also for all
instances in the next epoch too.  However it cannot complete phase II for any
further epochs, so once it has entered the epoch $e(i_0) + 1$ it should prepare
for the next membership change by running a new phase I for a new proposal $p'$
having $e(p') = e(i_0) + 1$. Importantly, at this point this new phase I is not
on the critical path of the phase II of any future instances, so cannot cause
jitter if it is delayed. Typically, as with Cheap Paxos, this new phase I is
run on a subset of all the nodes while the remaining nodes continue to run
phase II on proposal $p$.


\subsection{Examples of membership changes}\label{types-of-membership-change}

\def\maj#1{\mathbf{maj}(#1)}

Implementations can ensure $Q_e \smile Q_e$ by, for instance, arranging for
each quorum in $Q_e$ to comprise a majority subset of some finite set of nodes.
If $Q_{e-1}$ and $Q_e$ are nonempty then intuitively the condition that
$Q_{e-1} \smile Q_e$ ensures that the membership change from $Q_{e-1}$ to $Q_e$
is sufficiently small that phase I need not re-run when moving from epoch $e-1$
to epoch $e$.  For instance, if $Q_{e-1} \smile Q_{e-1}$ then the trivial
membership change, having $Q_e = Q_{e-1}$, certainly satisfies $Q_e \smile
Q_{e-1}$ and $Q_e \smile Q_e$.

For a less trivial example, let \[\maj{s} = \{ q \subseteq s \mid 2 |q| > |s|
\}\] be the set of all majority subsets of a finite set $s \subseteq \mathbb
A$, and note that $\maj{s} \smile \maj{s}$ for all such $s$.  If $\mathbb A =
\{ a_1, a_2, \ldots \}$ then let $s_n = \{a_1, \ldots, a_n\}$ and observe that
${\maj{s_3} \smile \maj{s_4}}$ and ${\maj{s_4} \smile \maj{s_5}}$ but ${\maj{s_3}
\not\smile \maj{s_5}}$ because $\{a_1, a_2\} \in \maj{s_3}$ and $\{a_3, a_4,
a_5\} \in \maj{s_5}$ do not intersect. Thus the majority subsets of $s_3$ and
$s_4$ are sufficiently similar to be consecutive memberships, but those of
$s_3$ and $s_5$ are not.

The intuition that $Q_{e-1} \smile Q_e$ means that $Q_{e-1}$ and $Q_e$ are
similar breaks down if either membership is empty, because $\varnothing \smile
Q$ vacuously for all memberships $Q$. Clearly if there is an instance $i$ such
that $Q_{e(i)} = \varnothing$ then no value can ever be chosen for $i$, but
epochs may be skipped by letting $e(i) = e(i-1) + 2$ at some instance $i$.
This recovers the ability to perform arbitrary membership changes as in Dynamic
Paxos: if the system is currently using membership $Q_{e(i)}$ and an operator
wishes to change to an unrelated membership $Q'$ (where necessarily $Q' \smile
Q'$) then they can set $Q_{e(i)+1} = \varnothing$ and $Q_{e(i)+2} = Q'$, which
satisfies that \[Q_{e(i)} \smile Q_{e(i) + 1} \smile Q_{e(i)+1} \smile Q_{e(i)
+ 2} \smile Q_{e(i) + 2}\] as desired. However in this situation phase I must
re-run before the new membership can be used because an epoch is skipped, which
reintroduces the risk of jitter. As in Dynamic Paxos, to mitigate this risk the
operator chooses an $\alpha > 0$ and sets $e(i+\alpha) = e(i)+2$ and $e(j) =
e(i)$ for $i < j \le i + \alpha$, effectively delaying the membership change.

\subsection{Weight-based memberships}
\label{weight-based-memberships}

Let a \textit{weight} function be a function $w : \mathbb A \to \mathbb N$ that
only takes finitely many nonzero values. This can be used to define a
membership $M(w)$ by \textit{majority}: \[M(w) = \left\{ q \;\middle|\; \sum_{a
\in q} 2 w(a) > \sum_{a \in \mathbb A} w(a) \right\}.\] If $w(a) = 0$ for all
$a$ then $w$ is said to be \textit{weightless} and $M(w) = \varnothing$.
Corollary \ref{weights-equal} shows that $M(w) \smile M(w)$ for any weight
function $w$.

A sequence $w_0, w_1, \ldots$ of weight functions can be used to define the
sequence of memberships used by a cluster, which must satisfy $M(w_{e-1})
\smile M(w_e)$ for all $e$.  There are three simple ways to ensure this:

\begin{enumerate}

\item If $w_{e-1}$ and $w_e$ differ by a constant factor, in the sense that
there are positive integers $m$ and $n$ such that $m w_{e-1}(a) = n w_e(a)$
for all $a$, then ${M(w_{e-1}) = M(w_e)}$ and hence the property holds.

\item By theorem \ref{weights-nearly-equal} if $w_{e-1}$ and $w_e$
differ by at most $1$ at at most one node then the property holds.

\item Finally, if $w_{e-1}$ is weightless then the property holds vacuously.

\end{enumerate}

\section{Discussion}

Paxos supports both the crash-stop and crash-recover failure modes.

To do crash-recovery, there must be a write to stable storage between
receiving a $\prop{i}{p}$ message and replying with $\acc{i}{a}{p}$.

Raises the question of what to do if storage itself is lost? What does
``stable'' mean?

In a realistic system, all nodes will eventually fail unrecoverably.





\subsection{TODO merge with the above}

Typically the nodes in a system must be taken offline for maintenance from time
to time. Since Paxos is designed to be resilient to failures of nodes, a simple
approach is to treat offline nodes just as if they have failed. By taking each
node offline in turn, the system can remain available even while maintenance is
being performed.

However, a na\"ive approach to this puts the system at risk during the upgrade:
in a cluster of three equally-weighted nodes if one of the nodes is taken
offline then the failure of another node would render the system unavailable.
More generally, if the system is to be resilient to $\le f$ failures then a
node $a_{\textrm{maint}}$ may be safely taken offline if all other nodes have
weight $< \sum_{a \in \mathbb A} \frac{w(a)}{2f} - w(a_{\textrm{maint}})$. For
example, if $f = 1$ and all nodes have equal weights then the smallest cluster
in which a node may be safely taken offline for maintenance has five nodes.
Assuming that the cost of operating a cluster is proportional to the number of
nodes, this represents a significant overhead simply to cover relatively rare
periods of maintenance: if maintenance is ignored then only three nodes are
required.

The required number of nodes can be reduced using a membership change as
follows. A cluster of four equally-weighted nodes can only tolerate a single
failure, but by reducing the weight of node $a_{\textrm{maint}}$ to zero before
taking it offline, the remaining three nodes can tolerate the loss of another
node. The sequence of weights used for this procedure is as follows, with
maintenance on $a_{\textrm{maint}}$ occurring during while $w_{e+1}$ is in use.
\[\begin{array}{rcccc}
\textrm{node}&a_{\textrm{maint}}&a_1&a_2&a_3 \\
w_e&1&1&1&1\\
w_{e+1}&0&1&1&1\\
w_{e+2}&1&1&1&1\\
\end{array}\]
Note that this sequence satisfies the consecutive epoch intersection property
since each step changes the weight of a single node by $1$.

In many situations it is feasible simply to replace a node $a_{\textrm{old}}$
with a new node $a_{\textrm{new}}$ rather than to upgrade it in-situ. If
$a_{\textrm{new}}$ is brought online before $a_{\textrm{old}}$ is taken offline
then the system is not put at risk, requiring four nodes to perform maintenance
but only three in normal operation:
\[\begin{array}{rcccc}
\textrm{node}&a_{\textrm{old}}&a_{\textrm{new}}&a_1&a_2 \\
w_e&1&0&1&1\\
w_{e+1}&1&1&1&1\\
w_{e+2}&0&1&1&1\\
\end{array}\]

Assuming that node failures are independent, the probability of a cluster-wide
failure can appear to be extremely small. However, in a real-world system node
failures are not always independent: certain sets of nodes may share critical
infrastructure such as power or network connectivity, and a failure of this
infrastructure would cause a simultaneous failure of all the dependent nodes.
To control this sitation, the cluster may be divided into \textit{zones}, where
the nodes within a single zone may share infrastructure but failures in
different zones can be considered to be independent. In a traditional data
centre environment, individual racks are often assumed to be independent, so
each rack could be a separate zone. If geographical redundancy is required then
the cluster will be separated across multiple data centres, each of which is an
independent zone, and many cloud environments also divide their services into
zones in this way.  For the system to be resilient to simultaneous zone-wide
failures of $f_z > 0$ zones, it must have nodes running in at least $2f_z + 1
\ge 3$ zones.

The four-node maintenance process described above is only safe if each node is
in a different zone: if any two of them are in the same zone then system will
fail if there is a zone-wide failure in epoch $w_{e+1}$.  In many operating
environments it may be expensive or complicated to arrange for four independent
zones\footnote{For instance, at time of writing, only one Amazon Web Services
region (\texttt{us-east-1}) has four independent zones, whereas four of them
have three: \texttt{ap-southeast-2}, \texttt{eu-west-1}, \texttt{sa-east-1} and
\texttt{us-west-2}. Similarly, only one Google Cloud Platform region
(\texttt{us-central1}) has four zones and all the others have three zones.},
particularly if the fourth zone is only required to ensure consistency in
relatively rare periods of maintenance.  Fortunately, it is possible to safely
replace a node within a single zone $z_{\textrm{maint}}$ using only three zones
and the following sequence of operations.
\[\begin{array}{rcccc}
\textrm{zone}&z_{\textrm{maint}}&z_{\textrm{maint}}&z_1&z_2\\
\textrm{node}&a_{\textrm{old}}&a_{\textrm{new}}&a_1&a_2\\
w_e&1&0&1&1\\
w_{e+1}&2&0&2&2\\
w_{e+2}&1&0&2&2\\
w_{e+3}&1&1&2&2\\
w_{e+4}&0&1&2&2\\
w_{e+5}&0&2&2&2\\
w_{e+6}&0&1&1&1\\
\end{array}\]
At every stage it is clear that every zone contains less than half of the total
weight, so the system is never at risk. Note that the first and last steps rely
on the ability to simultaneously change the weights of every node by a constant
factor, and every other step changes the weight of a single node by one, so
this satisfies the consecutive epoch intersection property and can be performed
without jitter.



% An example of a floating figure using the graphicx package.
% Note that \label must occur AFTER (or within) \caption.
% For figures, \caption should occur after the \includegraphics.
% Note that IEEEtran v1.7 and later has special internal code that
% is designed to preserve the operation of \label within \caption
% even when the captionsoff option is in effect. However, because
% of issues like this, it may be the safest practice to put all your
% \label just after \caption rather than within \caption{}.
%
% Reminder: the "draftcls" or "draftclsnofoot", not "draft", class
% option should be used if it is desired that the figures are to be
% displayed while in draft mode.
%
%\begin{figure}[!t]
%\centering
%\includegraphics[width=2.5in]{myfigure}
% where an .eps filename suffix will be assumed under latex, 
% and a .pdf suffix will be assumed for pdflatex; or what has been declared
% via \DeclareGraphicsExtensions.
%\caption{Simulation results for the network.}
%\label{fig_sim}
%\end{figure}

% Note that the IEEE typically puts floats only at the top, even when this
% results in a large percentage of a column being occupied by floats.


% An example of a double column floating figure using two subfigures.
% (The subfig.sty package must be loaded for this to work.)
% The subfigure \label commands are set within each subfloat command,
% and the \label for the overall figure must come after \caption.
% \hfil is used as a separator to get equal spacing.
% Watch out that the combined width of all the subfigures on a 
% line do not exceed the text width or a line break will occur.
%
%\begin{figure*}[!t]
%\centering
%\subfloat[Case I]{\includegraphics[width=2.5in]{box}%
%\label{fig_first_case}}
%\hfil
%\subfloat[Case II]{\includegraphics[width=2.5in]{box}%
%\label{fig_second_case}}
%\caption{Simulation results for the network.}
%\label{fig_sim}
%\end{figure*}
%
% Note that often IEEE papers with subfigures do not employ subfigure
% captions (using the optional argument to \subfloat[]), but instead will
% reference/describe all of them (a), (b), etc., within the main caption.
% Be aware that for subfig.sty to generate the (a), (b), etc., subfigure
% labels, the optional argument to \subfloat must be present. If a
% subcaption is not desired, just leave its contents blank,
% e.g., \subfloat[].


% An example of a floating table. Note that, for IEEE style tables, the
% \caption command should come BEFORE the table and, given that table
% captions serve much like titles, are usually capitalized except for words
% such as a, an, and, as, at, but, by, for, in, nor, of, on, or, the, to
% and up, which are usually not capitalized unless they are the first or
% last word of the caption. Table text will default to \footnotesize as
% the IEEE normally uses this smaller font for tables.
% The \label must come after \caption as always.
%
%\begin{table}[!t]
%% increase table row spacing, adjust to taste
%\renewcommand{\arraystretch}{1.3}
% if using array.sty, it might be a good idea to tweak the value of
% \extrarowheight as needed to properly center the text within the cells
%\caption{An Example of a Table}
%\label{table_example}
%\centering
%% Some packages, such as MDW tools, offer better commands for making tables
%% than the plain LaTeX2e tabular which is used here.
%\begin{tabular}{|c||c|}
%\hline
%One & Two\\
%\hline
%Three & Four\\
%\hline
%\end{tabular}
%\end{table}


% Note that the IEEE does not put floats in the very first column
% - or typically anywhere on the first page for that matter. Also,
% in-text middle ("here") positioning is typically not used, but it
% is allowed and encouraged for Computer Society conferences (but
% not Computer Society journals). Most IEEE journals/conferences use
% top floats exclusively. 
% Note that, LaTeX2e, unlike IEEE journals/conferences, places
% footnotes above bottom floats. This can be corrected via the
% \fnbelowfloat command of the stfloats package.




\section{Conclusion}
The conclusion goes here.

TODO note that $\mathbb P = \mathbb N \times \mathbb N \times \mathbb A$
with the lexicographic order.




% if have a single appendix:
%\appendix[Proof of the Zonklar Equations]
% or
%\appendix  % for no appendix heading
% do not use \section anymore after \appendix, only \section*
% is possibly needed

% use appendices with more than one appendix
% then use \section to start each appendix
% you must declare a \section before using any
% \subsection or using \label (\appendices by itself
% starts a section numbered zero.)
%


\appendices
\section{Consistency of the Synod algorithm}
\label{synod-safety}

The consistency property of the Synod algorithm is derived from the following
invariants.

\begin{enumerate}

\item \label{synod-quorums} For proposal identifiers $p_2 \prec p_1 \in \mathbb
P$ there are sets of quorums $Q^\textrm{I}(p_1)$ and
$Q^\textrm{II}(p_2) \subseteq \mathcal P \mathbb A$ such that if $\prop{}{p_1}$
and $\chosen{}{p_2}$ then ${Q^\textrm{I}(p_1) \smile Q^\textrm{II}(p_2)}$.

\item \label{synod-fprom} $\fprom{}{a}{p}$ only if $\neg\acc{}{a}{p'}$ for all
${p' \prec p}$.

\item \label{synod-bprom} $\bprom{}{a}{p}{p'}$ only if
\begin{itemize}
\item $p' \prec p$,
\item $\acc{}{a}{p'}$, and
\item $p'$ is the greatest such proposal, in the sense that
$\neg\acc{}{a}{p''}$ for all $p''$ having $p' \prec p'' \prec p$.
\end{itemize}

\item \label{synod-prop} $\prop{}{p}$ only if there is a quorum $q^\textrm{I}
\in Q^\textrm{I}(p)$ such that
\begin{itemize}
\item for every $a \in q^\textrm{I}$ either $\fprom{}{a}{p}$ or $\exists p'.
\bprom{}{a}{p}{p'}$, and
\item if $P = \{ p' \mid \exists a \in q^\textrm{I}. \bprom{}{a}{p}{p'} \}
\ne \varnothing$ then $v(p) = v(\mathrm{max}(P))$.
\end{itemize}

\item \label{synod-acc} $\acc{}{a}{p}$ only if $\prop{}{p}$.

\item \label{synod-chosen} $\chosen{}{p}$ only if there is a quorum
$q^\textrm{II} \in Q^\textrm{II}(p)$ such that $\acc{}{a}{p}$ for every $a \in
q^\textrm{II}$.

\end{enumerate}

\begin{lemma}\label{synod-acc-bprom}If $\acc{}{a}{p_2}$, $\bprom{}{a}{p_1}{p_3}$
and $p_2 \prec p_1$ then $p_2 \preceq p_3$.\end{lemma}

\begin{proof}Since $\bprom{}{a}{p_1}{p_3}$ from invariant \ref{synod-bprom} it
follows that $p_3 \prec p_1$, $\acc{}{a}{p_3}$ and $p_3$ is the largest such
proposal. Therefore either $p_2 \preceq p_3$ or $p_1 \preceq p_2$.  But $p_2
\prec p_1$ so it must be that $p_2 \preceq p_3$ as required.  \end{proof}

\begin{lemma}\label{synod-lemma} If $\chosen{}{p_2}$, $\prop{}{p_1}$ and $p_2
\prec p_1$ then $v(p_1) = v(p_2)$. \end{lemma}

\begin{proof}Suppose for a contradiction the result is false, and since $\prec$
is wellfounded suppose without loss of generality that $p_1$ is the
smallest-numbered proposal at which it is false.  Since $\chosen{}{p_2}$, by
invariant \ref{synod-chosen} there is a quorum $q^\textrm{II} \in
Q^\textrm{II}(p_2)$ such that $\acc{}{a}{p_2}$ for every $a \in q^\textrm{II}$.
By invariant \ref{synod-fprom} it cannot be that $\fprom{}{a}{p_1}$ for any $a
\in q^\textrm{II}$.  Also, since $\prop{}{p_1}$, by invariant \ref{synod-prop}
there is a quorum $q^\textrm{I} \in Q^\textrm{I}(p_1)$ such that either
$\fprom{}{a}{p_1}$ or $\exists p'.  \bprom{}{a}{p_1}{p'}$ for all $a \in
q^\textrm{I}$.  Let $P = \{ p' \mid \exists a \in q^\textrm{I}.
\bprom{}{a}{p_1}{p'} \}$ be the set of proposals referenced by forced promises.
By invariant \ref{synod-quorums}, ${Q^\textrm{I}(p_1) \smile
Q^\textrm{II}(p_2)}$ and hence $q^\textrm{I} \cap q^\textrm{II} \ne
\varnothing$ so it follows that $P \ne \varnothing$, which means that $v(p_1) =
v(\mathrm{max}(P))$ by invariant \ref{synod-prop}. Let $a_{\mathrm{max}} \in
q^\textrm{I}$ be such that $\bprom{}{a_{\mathrm{max}}}{p_1}{\mathrm{max}(P)}$.
By invariant \ref{synod-bprom} it follows that $\mathrm{max}(P) \prec p_1$ and
also that $\acc{}{a_{\mathrm{max}}}{\mathrm{max}(P)}$ and hence
$\prop{}{\mathrm{max}(P)}$ by invariant \ref{synod-acc}. Furthermore by lemma
\ref{synod-acc-bprom} it follows that $p_2 \preceq \mathrm{max}(P)$ and since
$p_1$ was assumed to be the smallest counterexample it must be that
$v(\mathrm{max}(P)) = v(p_2)$.  Hence $v(p_1) = v(p_2)$ which is a
contradiction as required.  \end{proof}

\begin{theorem}\label{synod-safety-theorem} If $\chosen{}{p_1}$ and
$\chosen{}{p_2}$ then $v(p_1) = v(p_2)$.  \end{theorem}

\begin{proof} If $p_1 = p_2$ then the result is clear, so assume that $p_1 \ne
p_2$ and therefore without loss of generality that ${p_2 \prec p_1}$. By
invariant \ref{synod-chosen} there is a quorum $q \in Q^\textrm{II}(p_1)$ such
that $\acc{}{a}{p_1}$ for every node $a \in q$. Since $q$ is nonempty, by
invariant \ref{synod-acc} it must be that $\prop{}{p_1}$ was previously sent.
Therefore by lemma \ref{synod-lemma} it follows that $v(p_1) = v(p_2)$ as
required.  \end{proof}




% you can choose not to have a title for an appendix
% if you want by leaving the argument blank
\section{Consistency of the Paxos algorithm}

The consistency property of the Paxos algorithm is derived from the following
invariants.

\begin{enumerate}

\item\label{paxos-quorums} There is a sequence of quorums $Q_0, Q_1, \ldots$
where ${Q_e \smile Q_e \smile Q_{e+1}}$ for all $e$.

\item\label{paxos-mprom} $\mprom{i}{a}{p}$ only if $\neg\acc{i'}{a}{p'}$ for all
$i' \ge i$ and $p' \prec p$.

\item\label{paxos-fprom} $\fprom{i}{a}{p}$ only if $\neg\acc{i}{a}{p'}$ for all
$p' \prec p$.

\item\label{paxos-bprom} $\bprom{i}{a}{p}{p'}$ only if \begin{itemize} \item
$p' \prec p$ \item $\acc{i}{a}{p'}$, and \item $p'$ is the greatest such
proposal, in the sense that $\neg \acc{i}{a}{p''}$ for all $p''$ having $p'
\prec p'' \prec p$. \end{itemize}

\item\label{paxos-prop} $\prop{i}{p}$ only if $e(p) \le e(i)$ and there is a
quorum $q \in Q_{e(p)}$ such that
\begin{itemize}
\item for every $a \in q$ one of the following holds:
%
\begin{itemize}
\item $\mprom{i'}{a}{p}$ for some $i' \le i$, or
\item $\fprom{i}{a}{p}$, or
\item $\bprom{i}{a}{p}{p'}$ for some $p'$, and
\end{itemize}

\item if $P = \{ p' \mid \exists a \in q. \bprom{i}{a}{p}{p'} \}
\ne \varnothing$ then $v_i(p) = v_i(\mathrm{max}(P))$.
\end{itemize}

\item \label{paxos-acc} $\acc{i}{a}{p}$ only if $\prop{i}{p}$.

\item \label{paxos-chosen} $\chosen{i}{p}$ only if
\begin{itemize}
\item $e(i) \le e(p) + 1$,
\item if $i > 0$ then $\chosen{i-1}{p'}$ for some $p'$, and
\item there is a quorum $q \in
Q_{e(i)}$ such that $\acc{i}{a}{p}$ for every $a \in q$.
\end{itemize}

\end{enumerate}

\begin{theorem}\label{paxos-safety-theorem} If $\chosen{i}{p_1}$ and
$\chosen{i}{p_2}$ then ${v_i(p_1) = v_i(p_2)}$.  \end{theorem}

\begin{proof}
If $\chosen{i}{p_1}$ then the Paxos invariants imply the Synod invariants for
instance $i$.  In more detail, let
\[\begin{array}{rl}
Q^\textrm{I}(p) &= \begin{cases}
Q_{e(p)} & e(p) \le e(i) \\
\varnothing & \textrm{otherwise,} \end{cases} \\
v(p) &= v_i(p)\\
Q^\textrm{II}(p) &= Q_{e(i)}, \\
\fprom{}{a}{p} &= \fprom{i}{a}{p} \\
& \ {}\vee \exists i' \le i. \mprom{i'}{a}{p}, \\
\bprom{}{a}{p}{p'} &= \bprom{i}{a}{p}{p'}, \\
\prop{}{p} &= \prop{i}{p}, \\
\acc{}{a}{p} &= \acc{i}{a}{p} \textrm{ and} \\
\chosen{}{p} &= \chosen{i}{p}. \\
\end{array}
\]

To show Synod invariant \ref{synod-quorums}, let $p'_2 \prec p'_1$ where
$\prop{i}{p'_1}$ and $\chosen{i}{p'_2}$. Certainly $\varnothing \smile
Q_{e(i)}$, so it remains to show that $Q_{e(p'_1)} \smile Q_{e(i)}$ if $e(p'_1)
\le e(i)$.  But since $\chosen{i}{p'_1}$ it follows that $e(i) \le e(p'_1) + 1$
and hence $e(p'_1) \in \{e(i) - 1, e(i)\}$ and the result follows since ${Q_e
\smile Q_e \smile Q_{e+1}}$ for all $e$. The remaining invariants are simple to
show and hence $v_i(p_1) = v_i(p_2)$ by theorem \ref{synod-safety-theorem}.
\end{proof}

\section{Preserving the Paxos invariants}

If no messages have been sent then the invariants shown above certainly hold,
so as long as each message that each node sends preserves the invariants then
consistency is maintained.

The interplay between promises and acceptances is reasonably straightforward:
the node $a$ may emit promises $\fprom{i}{a}{p}$ etc. and acceptances
$\acc{i}{a}{p}$ where they do not conflict with previously-sent messages. In
more detail, it may emit $\mprom{i}{a}{p}$ if it has previously sent no
$\acc{i'}{a}{p'}$ for any $p'$ and any $i' \le i$.  It may emit
$\fprom{i}{a}{p}$ if it has previously sent no $\acc{i}{a}{p'}$ for any $p'$.
It may emit $\bprom{i}{a}{p}{p'}$ if $p' \prec p$, $\acc{i}{a}{p'}$ and $p'$ is
the greatest such proposal. It may emit $\acc{i}{a}{p}$ as long as
$\prop{i}{p}$ and it does not violate any previously-made promises, in the
sense that whenever \begin{itemize} \item $\mprom{i'}{a}{p'}$ for any $i' \le
i$, or \item $\fprom{i}{a}{p'}$, or \item $\bprom{i}{a}{p'}{p''}$\end{itemize}
it is the case that $p' \preceq p$.

The node $\owner{p}$ may change the value $v_i(p)$ as long as it has not
already emitted $\prop{p}{i}$, and may consider phase I to be complete for
proposal $p$ at instance $i$, and may therefore emit $\prop{p}{i}$, there is an
earlier instance ${i' < i}$ and proposal $p'$ such that $\chosen{i'}{p'}$ and
$e(p) \le e(i')$, and there is a quorum $q \in Q_{e(p)}$ such that for every $a
\in q$ it has previously received a promise from node $a$ for proposal $p$ in
instance $i$. If it has not already emitted $\prop{p}{i}$, and any of the
promises from nodes in $q$ are of the form $\bprom{i}{a}{p}{p'}$ then it must
first set \[v_i(p) = v_i\bigl(\mathrm{max} \{p' \mid \bprom{i}{a}{p}{p'}
\textrm{ received}, a \in q \}\bigr).\]

Finally any node may consider phase II to be complete for proposal $p$ at instance
$i$, and therefore $\chosen{p}{i}$, if
\begin{itemize}
\item $e(i) \le e(p) + 1$,
\item if $i > 0$ then $\chosen{i-1}{p'}$ for some $p'$, and
\item there is a quorum $q \in
Q_{e(i)}$ such that $\acc{i}{a}{p}$ has been received from every $a \in q$.
\end{itemize}


\section{Consecutive epochs using weight-based membership}

\begin{theorem} \label{weights-nearly-equal} Let $w, w' : \mathbb A \to \mathbb
N$ be weight functions and let $a_0 \in \mathbb A$ be such that $|w(a_0) -
w'(a_0)| \le 1$ and $w(a) = w'(a)$ for all $a \ne a_0$. Then $M(w) \smile
M(w')$.  \end{theorem}

\begin{proof} Let $q \in M(w)$ and $q' \in M(w')$. By the
  definition of $M$, and since $w$ and $w'$ take only integer values,
%
$\sum_{a \in q} 2 w(a) \ge \sum_{a \in \mathbb A} w(a) + 1$
%
and
%
$\sum_{a \in q'} 2 w'(a) \ge \sum_{a \in \mathbb A} w'(a) + 1$.
%
Let $d_{\mathbb A} = w'(a_0) - w(a_0)$ so that $\sum_{a \in \mathbb A} w'(a) =
\sum_{a \in \mathbb A} w(a) + d_{\mathbb A}$. Also let \[
d_{q'} =
\begin{cases}
%
d_{\mathbb A} & a_0 \in q' \\
%
0 & \textrm{otherwise,}
%
\end{cases}
\]
so that $\sum_{a \in q'} w'(a) = \sum_{a \in q'} w(a) + d_{q'}$.
%
Then
\begin{gather*}
%
\sum_{a \in \mathbb A} 2w(a) + d_{\mathbb A} + 2 \\
%
\begin{aligned}
%
&= \left( \sum_{a \in \mathbb A} w(a)  + 1\right)
+  \left( \sum_{a \in \mathbb A} w'(a) + 1\right) \\
%
&\le \sum_{a \in q}  2w(a)
+    \sum_{a \in q'} 2w'(a) \\
%
&= \sum_{a \in q}  2w(a)
+  \sum_{a \in q'} 2w(a) + 2d_{q'}\\
%
&= \sum_{a \in q \cup q'} 2w(a)
+  \sum_{a \in q \cap q'} 2w(a) + 2d_{q'}\\
%
&\le \sum_{a \in \mathbb A} 2w(a)
+    \sum_{a \in q \cap q'} 2w(a) + 2d_{q'},\\
%
\end{aligned}\end{gather*} so that $\sum_{a \in q \cap q'} 2w(a) \ge d_{\mathbb
A} + 2 - 2d_{q'} = 2 \pm d_\mathbb A \ge 1$ and hence $q \cap q' \ne \varnothing$ as desired.
\end{proof}

\begin{corollary} \label{weights-equal} If $w : \mathbb A \to \mathbb N$ is a
weight function then \[M(w) \smile M(w).\]  \end{corollary}

\begin{proof}
This is a special case of theorem \ref{weights-nearly-equal}, where $w' = w$.
\end{proof}

% use section* for acknowledgment
\section*{Acknowledgment}


The authors would like to thank...


% Can use something like this to put references on a page
% by themselves when using endfloat and the captionsoff option.
\ifCLASSOPTIONcaptionsoff
  \newpage
\fi



% trigger a \newpage just before the given reference
% number - used to balance the columns on the last page
% adjust value as needed - may need to be readjusted if
% the document is modified later
%\IEEEtriggeratref{8}
% The "triggered" command can be changed if desired:
%\IEEEtriggercmd{\enlargethispage{-5in}}

% references section

% can use a bibliography generated by BibTeX as a .bbl file
% BibTeX documentation can be easily obtained at:
% http://mirror.ctan.org/biblio/bibtex/contrib/doc/
% The IEEEtran BibTeX style support page is at:
% http://www.michaelshell.org/tex/ieeetran/bibtex/
%\bibliographystyle{IEEEtran}
% argument is your BibTeX string definitions and bibliography database(s)
%\bibliography{IEEEabrv,../bib/paper}
%
% <OR> manually copy in the resultant .bbl file
% set second argument of \begin to the number of references
% (used to reserve space for the reference number labels box)
\begin{thebibliography}{1}

\bibitem{IEEEhowto:kopka}
H.~Kopka and P.~W. Daly, \emph{A Guide to \LaTeX}, 3rd~ed.\hskip 1em plus
  0.5em minus 0.4em\relax Harlow, England: Addison-Wesley, 1999.

\end{thebibliography}

% biography section
% 
% If you have an EPS/PDF photo (graphicx package needed) extra braces are
% needed around the contents of the optional argument to biography to prevent
% the LaTeX parser from getting confused when it sees the complicated
% \includegraphics command within an optional argument. (You could create
% your own custom macro containing the \includegraphics command to make things
% simpler here.)
%\begin{IEEEbiography}[{\includegraphics[width=1in,height=1.25in,clip,keepaspectratio]{mshell}}]{Michael Shell}
% or if you just want to reserve a space for a photo:

\begin{IEEEbiography}{Michael Shell}
Biography text here.
\end{IEEEbiography}

% if you will not have a photo at all:
\begin{IEEEbiographynophoto}{John Doe}
Biography text here.
\end{IEEEbiographynophoto}

% insert where needed to balance the two columns on the last page with
% biographies
%\newpage

\begin{IEEEbiographynophoto}{Jane Doe}
Biography text here.
\end{IEEEbiographynophoto}

% You can push biographies down or up by placing
% a \vfill before or after them. The appropriate
% use of \vfill depends on what kind of text is
% on the last page and whether or not the columns
% are being equalized.

%\vfill

% Can be used to pull up biographies so that the bottom of the last one
% is flush with the other column.
%\enlargethispage{-5in}



% that's all folks
\end{document}


