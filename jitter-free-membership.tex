
%% bare_jrnl.tex
%% V1.4b
%% 2015/08/26
%% by Michael Shell
%% see http://www.michaelshell.org/
%% for current contact information.
%%
%% This is a skeleton file demonstrating the use of IEEEtran.cls
%% (requires IEEEtran.cls version 1.8b or later) with an IEEE
%% journal paper.
%%
%% Support sites:
%% http://www.michaelshell.org/tex/ieeetran/
%% http://www.ctan.org/pkg/ieeetran
%% and
%% http://www.ieee.org/

%%*************************************************************************
%% Legal Notice:
%% This code is offered as-is without any warranty either expressed or
%% implied; without even the implied warranty of MERCHANTABILITY or
%% FITNESS FOR A PARTICULAR PURPOSE! 
%% User assumes all risk.
%% In no event shall the IEEE or any contributor to this code be liable for
%% any damages or losses, including, but not limited to, incidental,
%% consequential, or any other damages, resulting from the use or misuse
%% of any information contained here.
%%
%% All comments are the opinions of their respective authors and are not
%% necessarily endorsed by the IEEE.
%%
%% This work is distributed under the LaTeX Project Public License (LPPL)
%% ( http://www.latex-project.org/ ) version 1.3, and may be freely used,
%% distributed and modified. A copy of the LPPL, version 1.3, is included
%% in the base LaTeX documentation of all distributions of LaTeX released
%% 2003/12/01 or later.
%% Retain all contribution notices and credits.
%% ** Modified files should be clearly indicated as such, including  **
%% ** renaming them and changing author support contact information. **
%%*************************************************************************


% *** Authors should verify (and, if needed, correct) their LaTeX system  ***
% *** with the testflow diagnostic prior to trusting their LaTeX platform ***
% *** with production work. The IEEE's font choices and paper sizes can   ***
% *** trigger bugs that do not appear when using other class files.       ***                          ***
% The testflow support page is at:
% http://www.michaelshell.org/tex/testflow/



\documentclass[journal]{IEEEtran}
%\usepackage[paperwidth=310mm,paperheight=200mm]{geometry} % TODO just for on-screen
%
% If IEEEtran.cls has not been installed into the LaTeX system files,
% manually specify the path to it like:
% \documentclass[journal]{../sty/IEEEtran}





% Some very useful LaTeX packages include:
% (uncomment the ones you want to load)


% *** MISC UTILITY PACKAGES ***
%
%\usepackage{ifpdf}
% Heiko Oberdiek's ifpdf.sty is very useful if you need conditional
% compilation based on whether the output is pdf or dvi.
% usage:
% \ifpdf
%   % pdf code
% \else
%   % dvi code
% \fi
% The latest version of ifpdf.sty can be obtained from:
% http://www.ctan.org/pkg/ifpdf
% Also, note that IEEEtran.cls V1.7 and later provides a builtin
% \ifCLASSINFOpdf conditional that works the same way.
% When switching from latex to pdflatex and vice-versa, the compiler may
% have to be run twice to clear warning/error messages.





% *** CITATION PACKAGES ***
%
\usepackage{cite}
% cite.sty was written by Donald Arseneau
% V1.6 and later of IEEEtran pre-defines the format of the cite.sty package
% \cite{} output to follow that of the IEEE. Loading the cite package will
% result in citation numbers being automatically sorted and properly
% "compressed/ranged". e.g., [1], [9], [2], [7], [5], [6] without using
% cite.sty will become [1], [2], [5]--[7], [9] using cite.sty. cite.sty's
% \cite will automatically add leading space, if needed. Use cite.sty's
% noadjust option (cite.sty V3.8 and later) if you want to turn this off
% such as if a citation ever needs to be enclosed in parenthesis.
% cite.sty is already installed on most LaTeX systems. Be sure and use
% version 5.0 (2009-03-20) and later if using hyperref.sty.
% The latest version can be obtained at:
% http://www.ctan.org/pkg/cite
% The documentation is contained in the cite.sty file itself.






% *** GRAPHICS RELATED PACKAGES ***
%
\ifCLASSINFOpdf
  \usepackage[pdftex]{graphicx}
  % declare the path(s) where your graphic files are
  % \graphicspath{{../pdf/}{../jpeg/}}
  % and their extensions so you won't have to specify these with
  % every instance of \includegraphics
  % \DeclareGraphicsExtensions{.pdf,.jpeg,.png}
\else
  % or other class option (dvipsone, dvipdf, if not using dvips). graphicx
  % will default to the driver specified in the system graphics.cfg if no
  % driver is specified.
  % \usepackage[dvips]{graphicx}
  % declare the path(s) where your graphic files are
  % \graphicspath{{../eps/}}
  % and their extensions so you won't have to specify these with
  % every instance of \includegraphics
  % \DeclareGraphicsExtensions{.eps}
\fi
% graphicx was written by David Carlisle and Sebastian Rahtz. It is
% required if you want graphics, photos, etc. graphicx.sty is already
% installed on most LaTeX systems. The latest version and documentation
% can be obtained at: 
% http://www.ctan.org/pkg/graphicx
% Another good source of documentation is "Using Imported Graphics in
% LaTeX2e" by Keith Reckdahl which can be found at:
% http://www.ctan.org/pkg/epslatex
%
% latex, and pdflatex in dvi mode, support graphics in encapsulated
% postscript (.eps) format. pdflatex in pdf mode supports graphics
% in .pdf, .jpeg, .png and .mps (metapost) formats. Users should ensure
% that all non-photo figures use a vector format (.eps, .pdf, .mps) and
% not a bitmapped formats (.jpeg, .png). The IEEE frowns on bitmapped formats
% which can result in "jaggedy"/blurry rendering of lines and letters as
% well as large increases in file sizes.
%
% You can find documentation about the pdfTeX application at:
% http://www.tug.org/applications/pdftex





% *** MATH PACKAGES ***
%
\usepackage{amsmath}
\usepackage{amssymb}

\usepackage{amsthm}
\newtheorem{theorem}{Theorem}
\newtheorem{lemma}[theorem]{Lemma}
\newtheorem{corollary}[theorem]{Corollary}
% A popular package from the American Mathematical Society that provides
% many useful and powerful commands for dealing with mathematics.
%
% Note that the amsmath package sets \interdisplaylinepenalty to 10000
% thus preventing page breaks from occurring within multiline equations. Use:
%\interdisplaylinepenalty=2500
% after loading amsmath to restore such page breaks as IEEEtran.cls normally
% does. amsmath.sty is already installed on most LaTeX systems. The latest
% version and documentation can be obtained at:
% http://www.ctan.org/pkg/amsmath





% *** SPECIALIZED LIST PACKAGES ***
%
%\usepackage{algorithmic}
% algorithmic.sty was written by Peter Williams and Rogerio Brito.
% This package provides an algorithmic environment fo describing algorithms.
% You can use the algorithmic environment in-text or within a figure
% environment to provide for a floating algorithm. Do NOT use the algorithm
% floating environment provided by algorithm.sty (by the same authors) or
% algorithm2e.sty (by Christophe Fiorio) as the IEEE does not use dedicated
% algorithm float types and packages that provide these will not provide
% correct IEEE style captions. The latest version and documentation of
% algorithmic.sty can be obtained at:
% http://www.ctan.org/pkg/algorithms
% Also of interest may be the (relatively newer and more customizable)
% algorithmicx.sty package by Szasz Janos:
% http://www.ctan.org/pkg/algorithmicx




% *** ALIGNMENT PACKAGES ***
%
%\usepackage{array}
% Frank Mittelbach's and David Carlisle's array.sty patches and improves
% the standard LaTeX2e array and tabular environments to provide better
% appearance and additional user controls. As the default LaTeX2e table
% generation code is lacking to the point of almost being broken with
% respect to the quality of the end results, all users are strongly
% advised to use an enhanced (at the very least that provided by array.sty)
% set of table tools. array.sty is already installed on most systems. The
% latest version and documentation can be obtained at:
% http://www.ctan.org/pkg/array


% IEEEtran contains the IEEEeqnarray family of commands that can be used to
% generate multiline equations as well as matrices, tables, etc., of high
% quality.




% *** SUBFIGURE PACKAGES ***
%\ifCLASSOPTIONcompsoc
%  \usepackage[caption=false,font=normalsize,labelfont=sf,textfont=sf]{subfig}
%\else
%  \usepackage[caption=false,font=footnotesize]{subfig}
%\fi
% subfig.sty, written by Steven Douglas Cochran, is the modern replacement
% for subfigure.sty, the latter of which is no longer maintained and is
% incompatible with some LaTeX packages including fixltx2e. However,
% subfig.sty requires and automatically loads Axel Sommerfeldt's caption.sty
% which will override IEEEtran.cls' handling of captions and this will result
% in non-IEEE style figure/table captions. To prevent this problem, be sure
% and invoke subfig.sty's "caption=false" package option (available since
% subfig.sty version 1.3, 2005/06/28) as this is will preserve IEEEtran.cls
% handling of captions.
% Note that the Computer Society format requires a larger sans serif font
% than the serif footnote size font used in traditional IEEE formatting
% and thus the need to invoke different subfig.sty package options depending
% on whether compsoc mode has been enabled.
%
% The latest version and documentation of subfig.sty can be obtained at:
% http://www.ctan.org/pkg/subfig




% *** FLOAT PACKAGES ***
%
%\usepackage{fixltx2e}
% fixltx2e, the successor to the earlier fix2col.sty, was written by
% Frank Mittelbach and David Carlisle. This package corrects a few problems
% in the LaTeX2e kernel, the most notable of which is that in current
% LaTeX2e releases, the ordering of single and double column floats is not
% guaranteed to be preserved. Thus, an unpatched LaTeX2e can allow a
% single column figure to be placed prior to an earlier double column
% figure.
% Be aware that LaTeX2e kernels dated 2015 and later have fixltx2e.sty's
% corrections already built into the system in which case a warning will
% be issued if an attempt is made to load fixltx2e.sty as it is no longer
% needed.
% The latest version and documentation can be found at:
% http://www.ctan.org/pkg/fixltx2e


%\usepackage{stfloats}
% stfloats.sty was written by Sigitas Tolusis. This package gives LaTeX2e
% the ability to do double column floats at the bottom of the page as well
% as the top. (e.g., "\begin{figure*}[!b]" is not normally possible in
% LaTeX2e). It also provides a command:
%\fnbelowfloat
% to enable the placement of footnotes below bottom floats (the standard
% LaTeX2e kernel puts them above bottom floats). This is an invasive package
% which rewrites many portions of the LaTeX2e float routines. It may not work
% with other packages that modify the LaTeX2e float routines. The latest
% version and documentation can be obtained at:
% http://www.ctan.org/pkg/stfloats
% Do not use the stfloats baselinefloat ability as the IEEE does not allow
% \baselineskip to stretch. Authors submitting work to the IEEE should note
% that the IEEE rarely uses double column equations and that authors should try
% to avoid such use. Do not be tempted to use the cuted.sty or midfloat.sty
% packages (also by Sigitas Tolusis) as the IEEE does not format its papers in
% such ways.
% Do not attempt to use stfloats with fixltx2e as they are incompatible.
% Instead, use Morten Hogholm'a dblfloatfix which combines the features
% of both fixltx2e and stfloats:
%
% \usepackage{dblfloatfix}
% The latest version can be found at:
% http://www.ctan.org/pkg/dblfloatfix




%\ifCLASSOPTIONcaptionsoff
%  \usepackage[nomarkers]{endfloat}
% \let\MYoriglatexcaption\caption
% \renewcommand{\caption}[2][\relax]{\MYoriglatexcaption[#2]{#2}}
%\fi
% endfloat.sty was written by James Darrell McCauley, Jeff Goldberg and 
% Axel Sommerfeldt. This package may be useful when used in conjunction with 
% IEEEtran.cls'  captionsoff option. Some IEEE journals/societies require that
% submissions have lists of figures/tables at the end of the paper and that
% figures/tables without any captions are placed on a page by themselves at
% the end of the document. If needed, the draftcls IEEEtran class option or
% \CLASSINPUTbaselinestretch interface can be used to increase the line
% spacing as well. Be sure and use the nomarkers option of endfloat to
% prevent endfloat from "marking" where the figures would have been placed
% in the text. The two hack lines of code above are a slight modification of
% that suggested by in the endfloat docs (section 8.4.1) to ensure that
% the full captions always appear in the list of figures/tables - even if
% the user used the short optional argument of \caption[]{}.
% IEEE papers do not typically make use of \caption[]'s optional argument,
% so this should not be an issue. A similar trick can be used to disable
% captions of packages such as subfig.sty that lack options to turn off
% the subcaptions:
% For subfig.sty:
% \let\MYorigsubfloat\subfloat
% \renewcommand{\subfloat}[2][\relax]{\MYorigsubfloat[]{#2}}
% However, the above trick will not work if both optional arguments of
% the \subfloat command are used. Furthermore, there needs to be a
% description of each subfigure *somewhere* and endfloat does not add
% subfigure captions to its list of figures. Thus, the best approach is to
% avoid the use of subfigure captions (many IEEE journals avoid them anyway)
% and instead reference/explain all the subfigures within the main caption.
% The latest version of endfloat.sty and its documentation can obtained at:
% http://www.ctan.org/pkg/endfloat
%
% The IEEEtran \ifCLASSOPTIONcaptionsoff conditional can also be used
% later in the document, say, to conditionally put the References on a 
% page by themselves.




% *** PDF, URL AND HYPERLINK PACKAGES ***
%
%\usepackage{url}
% url.sty was written by Donald Arseneau. It provides better support for
% handling and breaking URLs. url.sty is already installed on most LaTeX
% systems. The latest version and documentation can be obtained at:
% http://www.ctan.org/pkg/url
% Basically, \url{my_url_here}.




% *** Do not adjust lengths that control margins, column widths, etc. ***
% *** Do not use packages that alter fonts (such as pslatex).         ***
% There should be no need to do such things with IEEEtran.cls V1.6 and later.
% (Unless specifically asked to do so by the journal or conference you plan
% to submit to, of course. )


% correct bad hyphenation here
\hyphenation{op-tical net-works semi-conduc-tor}


\begin{document}
%
% paper title
% Titles are generally capitalized except for words such as a, an, and, as,
% at, but, by, for, in, nor, of, on, or, the, to and up, which are usually
% not capitalized unless they are the first or last word of the title.
% Linebreaks \\ can be used within to get better formatting as desired.
% Do not put math or special symbols in the title.
\title{Jitter-free Dynamic Paxos}
%
%
% author names and IEEE memberships
% note positions of commas and nonbreaking spaces ( ~ ) LaTeX will not break
% a structure at a ~ so this keeps an author's name from being broken across
% two lines.
% use \thanks{} to gain access to the first footnote area
% a separate \thanks must be used for each paragraph as LaTeX2e's \thanks
% was not built to handle multiple paragraphs
%

\author{David C. Turner% <-this % stops a space
\thanks{The author is with the Operations and Planning Systems division of
Tracsis plc, 103 Clarendon Road, Leeds LS2 9DF, United Kingdom (email:
d.turner@tracsis.com)
}% <-this % stops a space
\thanks{Unpublished draft version
0000000000000000000000000000000000000000
}%
%\thanks{Manuscript received April 19, 2005; revised August 26, 2015.}} TODO
}

% note the % following the last \IEEEmembership and also \thanks - 
% these prevent an unwanted space from occurring between the last author name
% and the end of the author line. i.e., if you had this:
% 
% \author{....lastname \thanks{...} \thanks{...} }
%                     ^------------^------------^----Do not want these spaces!
%
% a space would be appended to the last name and could cause every name on that
% line to be shifted left slightly. This is one of those "LaTeX things". For
% instance, "\textbf{A} \textbf{B}" will typeset as "A B" not "AB". To get
% "AB" then you have to do: "\textbf{A}\textbf{B}"
% \thanks is no different in this regard, so shield the last } of each \thanks
% that ends a line with a % and do not let a space in before the next \thanks.
% Spaces after \IEEEmembership other than the last one are OK (and needed) as
% you are supposed to have spaces between the names. For what it is worth,
% this is a minor point as most people would not even notice if the said evil
% space somehow managed to creep in.



% The paper headers
\markboth{}%TODO Journal of \LaTeX\ Class Files,~Vol.~14, No.~8, August~2015}%
{Turner: Jitter-free Dynamic Paxos}
% The only time the second header will appear is for the odd numbered pages
% after the title page when using the twoside option.
% 
% *** Note that you probably will NOT want to include the author's ***
% *** name in the headers of peer review papers.                   ***
% You can use \ifCLASSOPTIONpeerreview for conditional compilation here if
% you desire.




% If you want to put a publisher's ID mark on the page you can do it like
% this:
%\IEEEpubid{0000--0000/00\$00.00~\copyright~2015 IEEE}
% Remember, if you use this you must call \IEEEpubidadjcol in the second
% column for its text to clear the IEEEpubid mark.



% use for special paper notices
%\IEEEspecialpapernotice{(Invited Paper)}




% make the title area
\maketitle

% As a general rule, do not put math, special symbols or citations in the
% abstract or keywords.
\begin{abstract} Paxos is an algorithm for building fault-tolerant distributed
systems. Practical implementations of this kind of system must support dynamic
reconfiguration in order to be able to replace failed components and perform
other administrative tasks. Paxos can achieve high performance by pipelining
(starting work on new requests before existing requests have completed) but
typically limits the length of the pipeline to support dynamic reconfiguration.
With this approach if the pipeline is too long then a reconfiguration may take
a long time to complete, but if it too short then it may stall when the system
is being reconfigured which can be observed as jitter in the rate that the
system makes progress. Instead the algorithm can be modified to support dynamic
reconfigurations with an unbounded pipeline as long as the reconfigurations are
sufficiently small, and there is a simple characterisation of a set of
sufficiently small reconfigurations that is useful in practice.  With this
approach it is still possible to perform an arbitrary reconfiguration in a
single step by temporarily limiting the pipeline length.\end{abstract}

% Note that keywords are not normally used for peerreview papers.
%\begin{IEEEkeywords} TODO???
%IEEE, IEEEtran, journal, \LaTeX, paper, template.
%\end{IEEEkeywords}






% For peer review papers, you can put extra information on the cover
% page as needed:
% \ifCLASSOPTIONpeerreview
% \begin{center} \bfseries EDICS Category: 3-BBND \end{center}
% \fi
%
% For peerreview papers, this IEEEtran command inserts a page break and
% creates the second title. It will be ignored for other modes.
\IEEEpeerreviewmaketitle

\section{Introduction}

\IEEEPARstart{P}{axos} is an algorithm for achieving consensus on (or
\textit{choosing}) a sequence $v_0, v_1, \ldots$ of values in a distributed,
fault-tolerant fashion~\cite{part-time-parliament}. It guarantees system-wide
\textit{consistency}, in the sense that the values chosen are guaranteed to be
equal across the whole system even in the presence of failures. It typically
runs on a cluster of $2f+1$ nodes, where $f$ is the number of faulty nodes that
should be tolerated, and consensus is achieved on a value $v_i \in \mathbb V$
when $\ge f+1$ (i.e.  a majority) of the nodes agree.  More generally, if
$\mathbb A$ is the set of all nodes then there is a set $Q \subseteq \mathcal P
\mathbb A$ of \textit{quorums}, and a value is chosen when all nodes $a \in q$
agree for some quorum $q \in Q$. The set of quorums is called the
\textit{configuration} of the cluster.

It is normally necessary to be able to add or remove nodes from the cluster,
which amounts to the ability to change the configuration while the cluster is
running. It is crucial that all participating nodes agree on the cluster
configuration at all times, and Dynamic Paxos~\cite{cheap-paxos} achieves this
by tracking the configuration with a state machine whose transitions occur when
particular values in the sequence are chosen.

Each value $v_i \in \mathbb V$ is chosen using a separate instance $i$ of a
two-phase consensus protocol known as \textit{Synod}. The full Paxos algorithm
essentially runs an infinite sequence of Synod protocols in parallel: it starts
by running phase I of all instances at once and then runs phase II of each
instance in turn to yield the desired sequence of chosen values. It normally
remains in phase II for extended periods of time, but will return to phase I if
certain nodes become faulty, or if messages between certain pairs of nodes
cease to be delivered reliably for a period, or if the configuration changes.

Paxos allows for \textit{pipelining} \cite{smart} in phase II: later values can
be proposed before earlier ones have been chosen, which means that the rate of
choosing values is not limited by network latency. The pipeline runs smoothly
as long as phase I does not need to run again, but may stall if phase I is
re-run. A pipeline stall can be observed as jitter in the rate at which values
are chosen.

It is unavoidable that a faulty node or disrupted communications may have an
impact on the system's performance, but changing its configuration is a routine
task which should have no such effect. Dynamic Paxos maintains consistency by
limiting the length of the pipeline to $\alpha > 0$, so that when a cluster
configuration change is chosen at instance $i$ the effects of the change are
postponed until instance $i + \alpha$. If $\alpha$ is large enough, there is
time to run the new phase I before the pipeline stalls.  On the other hand, if
$\alpha$ is too large then the configuration change can take an unreasonably
long time to complete.

The pipeline length $\alpha$ may be allowed to vary while the system is
running, so it is certainly possible to design a mechanism for choosing a
reasonably suitable value and adapting it to changing system conditions,
although this seems inelegant\cite{reconfiguring-a-state-machine}. The approach
described here removes the need for a limited-length pipeline by allowing phase
II to start using a new configuration without re-running phase I; this is shown
to preserve consistency as long as the configuration change is sufficiently
small (in a sense which is made precise in section
\ref{types-of-configuration-change} below.) By removing phase I from the
critical path of a configuration change the possibility of jitter caused by a
pipeline stall is eliminated.

The technique shown here also supports arbitrary configuration changes
happening in a single step just as in Dynamic Paxos, but any changes that are
not sufficiently small require the pipeline length to be temporarily limited
and continue to carry the risk of jitter. In reality, most changes would
normally be performed as a sequence of small, jitter-free changes as discussed
in section \ref{discussion} below.

\section{The Synod Algorithm}

Synod\cite{part-time-parliament} is an algorithm for achieving consensus on a
single value in a distributed system comprising a set of nodes which can
communicate by sending messages to each other. Communication is asynchronous
but not Byzantine, in the sense that messages may be delayed, reordered,
duplicated and dropped but not corrupted. It is briefly reviewed here for the
sake of consistency of notation and to introduce some generalisations.

Let $\mathbb B$ be a set of \textit{ballot identifiers} with a wellfounded
total order $\prec$. Let $\mathbb A$ be a set of \textit{node identifiers} and
let $\mathbb V$ be the set of values that may be chosen.

\def\prep#1{\mathbf{prepare}(#1)}
\def\mprom#1#2#3{\mathbf{promised}_{\ge #1}(#2,#3)}
\def\fprom#1#2#3{\mathbf{promised}_{#1}(#2,#3)}
\def\bprom#1#2#3#4{\mathbf{promised}_{#1}(#2,#3;#4)}
\def\prop#1#2{\mathbf{proposed}_{#1}(#2)}
\def\acc#1#2#3{\mathbf{accepted}_{#1}(#2,#3)}
\def\chosen#1#2{\mathbf{chosen}_{#1}(#2)}
\def\owner#1{\mathrm{owner}(#1)}

The Synod algorithm involves four kinds of message, described below.
Throughout, $a \in \mathbb A$ and $b, b' \in \mathbb B$.  Phase I starts with
the broadcast of a \textit{prepare} message $\prep{b}$ to which each node may
respond with a \textit{promise} message, either a \textit{free promise} written
$\fprom{}{a}{b}$ or a \textit{forced promise} written $\bprom{}{a}{b}{b'}$
where $a$ identifies the responding node.  Phase II starts with the broadcast
of a \textit{proposal} message $\prop{}{b}$ to which each node $a$ may respond
with an \textit{acceptance} message $\acc{}{a}{b}$.

It is convenient also to use these symbols as predicates indicating whether the
corresponding messages have been sent. For instance, the predicate
$\acc{}{a}{b}$ indicates that the message $\acc{}{a}{b}$ has been sent and
$\neg \prop{}{b}$ indicates that the message $\prop{}{b}$ has not been sent.
Because messages may be delayed, reordered or dropped, a node can only know a
message has not been sent if it has not sent it itself and can prove, based
solely on the messages it alone has sent and received, that no other node has
sent it either. This works in practice because the messages $\fprom{}{a}{b}$,
$\bprom{}{a}{b}{b'}$ and $\acc{}{a}{b}$ may only originate from the node $a$,
and there is a function $\mathrm{owner} : \mathbb B \to \mathbb A$ assigning an
owner to each ballot such that the message $\prop{}{b}$ may only originate from
$\owner{b}$.  It is convenient to add another predicate $\chosen{}{b}$ which
indicates that the ballot $b$ may be chosen.

There is a function $v : \mathbb B \to \mathbb V$ assigning a value to each
ballot, discussed in more detail in section \ref{value-function} below, and
the system satisfies a set of invariants as described in appendix
\ref{synod-safety} from which can be shown that consistency is ensured:
\[\textrm{if } \chosen{}{b} \textrm{ and } \chosen{}{b'} \textrm { then } v(b)
= v(b').\]

Each node operates as a state machine whose transitions are caused by the
receipts of messages. A node may emit $\prop{}{b}$ when it considers phase I to
be complete at ballot $b$, which is when promise messages for $b$ have been
received from a quorum of nodes, and similarly may deduce $\chosen{}{b}$ when
it considers phase II to be complete at $b$, which is when acceptances of $b$
from a quorum of nodes have been received. The phase-I and phase-II quorums are
chosen so as to always contain at least one node in common, but may vary
depending on the ballot $b$ and the phase as discussed in section
\ref{per-phase-quorums} below.

\subsection{Implementing the value function}\label{value-function}

In the original presentation of the Synod algorithm, the values of ballots
are carried along with their identifiers in the messages $\bprom{}{a}{b}{
b',v(b') }$, $\prop{}{ b,v(b)}$ and $\acc{}{a}{ b,v(b) }$. Observation O4 in
\cite{cheap-paxos} notes that these values can be replaced in some cases by
hashes, which is useful if the values themselves are expensive to transmit
between nodes. This can be taken one step further and the values can be
completely elided from the messages that take part in the Synod protocol,
allowing considerably more freedom in the implementation of the function $v$
without sacrificing consistency.

Although it appears that the function $v$ is fixed, in practice it is allowed
to change as the system runs. Treating it as fixed simplifies the consistency
proof and highlights that its values need not be included in all messages, but
means that the system cannot be shown to satisfy any useful liveness
properties.

To recover liveness, it is enough to note that if the invariants of appendix
\ref{synod-safety} are satisfied with a value function $v$ then they are
satisfied for any value function $v'$ that agrees with $v$ on proposed ballots,
i.e. having $v(b) = v'(b)$ only whenever $\prop{}{b}$.  Furthermore only
$\owner{b}$ may propose $b$, so if it has not yet proposed a value for $b$ it
can prove that $\neg\prop{}{b}$ and therefore freely modify $v(b)$.

It is also important for liveness that the implementation of $v$ is resilient
to the same failure modes as the rest of the system. Since the value of each
ballot $b$ has at most one writer, $\owner{b}$, it is possible to think of
$v$ as an insert-only set of pairs ${\{ \langle b, v(b) \rangle \mid \prop{}{b}
\}}$ which is an example of a convergent replicated data type\cite{crdts}.
From this viewpoint, including values in all messages ensures convergence
occurs as quickly as possible, and replicating this set across all ${2f+1}$
nodes ensures $v$ may be resilient to as many as $2f$ failures. Cheap Paxos
improves on this by replicating $v$ itself across just the ${f+1}$ primary
processors, with the $f$ auxiliary processors storing just the hashes of values
to ensure integrity.  In other situations even weaker liveness properties may
be acceptable, which could allow for further optimisations.

\subsection{Per-phase quorums}\label{per-phase-quorums}

The heart of the consistency property of the Synod algorithm is that the set of
nodes involved in completing phase I must always intersect the set of nodes
involved in completing phase II, so that there is at least one node to carry
information between the two phases. Write $Q_1 \frown Q_2$ to indicate that
every element of $Q_1$ intersects every element of $Q_2$: more formally $Q_1
\frown Q_2$ iff every $q_1 \in Q_1$ and $q_2 \in Q_2$ have ${q_1 \cap q_2 \ne
\varnothing}$. Write $Q^\textrm{I}(b)$ and $Q^\textrm{II}(b)$ for the sets of
phase-I and phase-II quorums for ballot $b$ respectively.

In the original presentation of the Synod algorithm any (weighted) majority of
the nodes could be used as a quorum, so that $Q^\textrm{I}(b_1) \frown
Q^\textrm{II}(b_2)$ for all $b_1$ and $b_2$ since all majorities intersect.
This intersection property is stronger than needed: the proof of theorem
\ref{synod-safety-theorem} in appendix \ref{synod-safety} is a generalisation
of the original one to demonstrate that consistency still holds even under the
weaker assumption that $Q^\textrm{I}(b_1) \frown Q^\textrm{II}(b_2)$ only when
$\prop{}{b_1}$, $\chosen{}{b_2}$ and $b_1 \succ b_2$.

\section{The Paxos algorithm}

Conceptually, Paxos is a sequence of distinct instances of the Synod algorithm
all running simultaneously. To achieve this, the messages of the Synod
algorithm above are indexed with the instance number $i \in \mathbb N$:
$\fprom{i}{a}{b}$, $\bprom{i}{a}{b}{b'}$, $\prop{i}{b}$ and $\acc{i}{a}{b}$.
There is another kind of message known as a \textit{multi-promise}, written
$\mprom{i}{a}{b}$, which can be thought of as standing for the infinite set of
free promises ${\{ \fprom{j}{a}{b} \mid j \ge i \}}$. The predicate
$\chosen{i}{b}$ is also indexed by the instance number $i$, but in this
presentation prepare messages $\prep{b}$ apply to all instances so do not need
to be indexed.

Recall that in each Synod instance each node is a state machine whose
transitions are caused by the receipt of messages. In a Paxos node, let $S_i$
be the Synod state machine for instance $i$. Because at any time each node has
only ever received finitely many messages, there must be an instance number
$i_\infty$ such that all instances $i \ge i_\infty$ are in the same state, so a
Paxos node therefore needs to have a separate state machine $S_i$ for each
instance $i < i_\infty$ but can represent all instances $i \ge i_\infty$ with a
single state machine $S_\infty$. When a message mentioning an instance $i \ge
i_\infty$ is received, the node must increase $i_\infty$ to $i'_\infty > i$ and
create state machines $S_j$ for $i_\infty \le j < i'_\infty$, each of which
starts out as a copy of $S_\infty$. Typically, after an instance has been
chosen a node will ignore any further messages about that instance and discard
the corresponding state machines, which ensures resource usage does not grow
unboundedly.

As in the Synod algorithm, phase I starts with a broadcast of $\prep{b}$ for
some $b$ to which each node $a$ may respond with a set of promises
$\fprom{i}{a}{b}$, $\bprom{i}{a}{b}{b'}$ and $\mprom{i}{a}{b}$ according to its
set of state machines. If a node receives $\mprom{i}{a}{b}$ messages from a
quorum of nodes then its state machine $S_\infty$ has completed phase I at $b$,
and therefore any state machines that are subsequently spawned will also have
completed phase I at $b$. Each phase II instance $i$ operates just as in the
Synod algorithm, starting with a broadcast of $\prop{i}{b}$ to which each node
$a$ may respond with an acceptance $\acc{i}{a}{b}$ and once acceptances have
been received from a quorum of nodes it follows that $\chosen{i}{b}$.

There is a function $v_i : \mathbb B \to \mathbb V$ for each instance $i$
giving a value to each ballot, and theorem \ref{paxos-safety-theorem} shows
that whenever $\chosen{i}{b}$ and $\chosen{i}{b'}$ it follows that $v_i(b) =
v_i(b')$.

\subsection{Configuration changes}\label{configuration-changes}

Changes in the cluster configuration are handled with a sequence of numbered
\textit{epochs} $1, 2, \ldots$ and configurations $Q_0, Q_1, Q_2, \ldots$
satisfying $Q_{e-1} \frown Q_e$ and $Q_e \frown Q_e$ for all $e$.  This
means $Q_{e-1}$ and $Q_e$ can be used for the phase-I and phase-II
configurations of a Paxos instance respectively by the observation of
\ref{per-phase-quorums} above, or alternatively $Q_e$ can be used for both
phases.

There are also two nondecreasing functions, both written $e(\cdot)$, which
respectively assign an epoch number $e(i)$ to each instance $i$, and an epoch
number $e(b)$ to each ballot $b$. A node may emit $\prop{i}{b}$ if $e(b) \le
e(i)$ and it has received promises for ballot $b$ in instance $i$ from a quorum
in $Q_{e(b)}$. A node may decide $\chosen{i}{b}$ if ${e(i) \le e(b)+1}$ and it
has received $\acc{i}{a}{b}$ from a quorum in $Q_{e(i)}$. Thus if
$\chosen{i}{b}$ then ${e(b) \in \{ e(i)-1, e(i) \}}$ and hence $Q_{e(b)} \frown
Q_{e(i)}$ as required. It is worth emphasising the subtle point that phase I
uses the configuration $Q_{e(b)}$ corresponding to the ballot $b$ whereas phase
II uses the configuration $Q_{e(i)}$ for the instance $i$, with extra
conditions to ensure that $e(b) \le e(i) \le e(b) + 1$ and therefore that the
epochs used in the two phases are close enough to ensure consistency.

\subsection{Dynamic configuration changes}

As in Dynamic Paxos, the configurations $Q_0, Q_1, \ldots$ and epoch numbers
$e(i)$ are themselves chosen by Paxos. In more detail, configurations are
tracked by a state machine whose state is the finite sequence $Q_0, \ldots,
Q_{e_\mathrm{max}}$ and the epoch numbers of instances are tracked by a state
machine whose state is the finite sequence $e(0), \ldots, e(i_\mathrm{max})$.
Their transitions append one or more elements to these sequences, but do not
change any existing elements, and the tracked sequences are always long enough
to make progress: if $\chosen{i}{b}$ then the epoch-number state machine
satisfies $i_\mathrm{max} > i+1$ and $e_\mathrm{max} \ge e(i_\mathrm{max})$ in
the configuration-sequence state machine. Epoch numbers $e(b)$ for ballots
$b$ are fixed in advance and not chosen by Paxos.

Slightly more generally, if the system is being used to build a state machine
that remains consistent even if some of its transitions are reordered then it
may be desirable to be able to deduce $\chosen{i}{b}$ even if the chosen values
of all previous instances are not yet known.  This can be done with a limit
$\kappa > 0$ on how many instances may be chosen concurrently, and the
epoch-number state machine then must satisfy $i_\mathrm{max} > i + \kappa$.
Note carefully that this concurrency parameter $\kappa$ is different from the
parameter $\alpha$ in the original Dynamic Paxos: there $\prop{i}{b}$ may not
occur until all instances $\le i - \alpha$ have been chosen whereas here there
is no limit on which proposals can be made and the only restriction is that
$\chosen{i}{b}$ may not occur until all instances $\le i - \kappa$ are chosen.
In the common case that the transitions may not be reordered, $\kappa = 1$ is
sufficient.

In order for a node to decide whether it can send out $\prop{i}{b}$ it must
check whether it has received promises from a quorum of nodes $q \in Q_{e(b)}$,
for which it needs to know $Q_{e(b)}$. This can be achieved by simply requiring
nodes not to emit $\prep{b}$ messages for which $Q_{e(b)}$ is not known. The
node must also check that $e(b) \le e(i)$; if $e(i)$ is not known then
$i_\mathrm{max} < i$, but $e$ is a nondecreasing function so $e(i_\mathrm{max})
\le e(i)$ and it is therefore sufficient to check that $e(b) \le
e(i_\mathrm{max})$.

Similarly, in order to decide if $\chosen{i}{b}$ a node must check whether it
has received acceptances from a quorum of nodes $q \in Q_{e(i)}$ and must
therefore know the value of $Q_{e(i)}$. This will be known if the node has
chosen all but the previous $\kappa$ instances, because if
$\chosen{i-\kappa}{b'}$ then $i < i_\mathrm{max}$ and hence ${e(i) \le
e(i_\mathrm{max}) \le e_\mathrm{max}}$ as required.

\subsection{Jitter-free configuration changes}

In normal running there is an instance $i_0$ and a ballot $b$ with ${e(b) =
e(i_0) = e(i_\mathrm{max}) = e_\mathrm{max}}$ and the node $\owner{b}$ has
received a quorum of promises $q^\textrm{I} \in Q_{e(b)}$ for ballot $b$ for
all instances $i \ge i_0$. In this state, $\owner{b}$ may emit $\prop{i}{b}$
for any $i \ge i_0$, since $e(b) = e(i_0) \le e(i)$ as $e$ is nondecreasing.
The node $\owner{b}$ is known as the \textit{leader} and its proposals are
normally accepted without undue delay by all other nodes.

When in this state, if an operator wishes to change the cluster configuration
to $Q'$ where $Q_{e(b)} \frown Q' \frown Q'$ they must first arrange to append
$Q'$ to the sequence of configurations, setting $Q_{e(b)+1} = Q'$ and
$e_\mathrm{max} = e(b) + 1$, then pick a future instance $i_c > i_\mathrm{max}$
at which the change should take effect and append values to the sequence of
epochs to set $e(i) = e(i_0)$ for $i_0 \le i < i_c$ and $e(i_c) = e(i_0) + 1$.
Since $e(i_c) = e(i_0) + 1 \le e(b) + 1$, instance $i_c$ and any other
instances with the same epoch may be chosen even though phase I has not yet run
for a ballot in this epoch. As long as $e(i_\mathrm{max})$ is not increased
again, this means there is no limit on how many instances can be chosen without
waiting for another phase I to run, which means the pipeline cannot stall.

At this point, the system is no longer in normal running as defined above
because $e(b) = e_\mathrm{max} - 1$. In order to get back to normal running, a
ballot $b'$ should be chosen with $e(b') = e(b) + 1$ and $\owner{b'}$ should
try and obtain a quorum of promises for $b'$ without preventing progress in
epoch $e(b)$. It can do this by dividing the cluster into two quorums in such a
way that it has a \textit{casting vote} between the two quorums: it finds $q
\in Q_{e(b)}$ and $q' \in Q_{e(b)+1}$ having $q \cap q' = \{\owner{b'}\}$ and
broadcasts $\prep{b'}$ to all nodes in $q'$ except itself. While it is waiting
for the promises to arrive in response to this broadcast, $\owner{b}$ continues
to make progress choosing instances of ballot $b$ using the quorum $q \in
Q_{e(b)}$.  Eventually when promises have arrived from every node in $q'$
except $\owner{b'}$ itself, it can send itself a final promise, immediately
complete phase I for ballot $b'$, and start broadcasting proposals
$\prop{i}{b'}$.  This restores the system to normal running. Observe the
following:
%
\begin{itemize}
%
\item This process runs without jitter as long as there are no failures.  If
there is a failure then consistency and liveness are still preserved but there
may be a performance impact. This is unavoidable in general.
%
\item The first part of this process requires that $e(i_\mathrm{max})$
increases by at most $1$ before the new phase I has finished running. If
$e(i_\mathrm{max})$ were increased by a larger amount then this would put a
limit on how many instances can run in the old epoch, and if the limit were
reached then the pipeline may stall until the corresponding phase I was
completed.
%
\item The second part of this process requires the existence of a node
$\owner{b'}$ that has a casting vote. If $\owner{b} \ne \owner{b'}$ then this
process includes an \textit{abdication}: the original leader $\owner{b}$ hands
over the leadership of the cluster to $\owner{b'}$. As the current leader is
responsible for proposing values, the system may need to take action to
redirect its clients from the old leader to the new leader in order to fully
avoid jitter.
%
\end{itemize}

\subsection{Examples}\label{types-of-configuration-change}

\def\maj#1{\mathbf{maj}(#1)}

Implementations can ensure $Q_e \frown Q_e$ for each $e$ by, for instance,
arranging for each quorum in $Q_e$ to comprise a (weighted) majority subset of
some finite set of nodes.  If $Q_{e-1}$ and $Q_e$ are nonempty then intuitively
the condition that $Q_{e-1} \frown Q_e$ ensures that the configuration change
from $Q_{e-1}$ to $Q_e$ is sufficiently small that phase I need not re-run when
moving from epoch $e-1$ to epoch $e$.  For instance, if $Q_{e-1} \frown
Q_{e-1}$ then the trivial configuration change, having $Q_e = Q_{e-1}$,
certainly satisfies $Q_{e-1} \frown Q_e \frown Q_e$ and is therefore
sufficiently small.
%
For a less trivial example, let \[\maj{s} = \{ q \subseteq s \mid 2 |q| > |s|
\}\] comprise all majority subsets of a finite set $s \subseteq \mathbb A$, and
note that $\maj{s} \frown \maj{s}$ for all such $s$.  If $\mathbb A = \{ a_1,
a_2, \ldots \}$ then let $s_n = \{a_1, \ldots, a_n\}$ and observe that
${\maj{s_3} \frown \maj{s_4}}$ and ${\maj{s_4} \frown \maj{s_5}}$ but
${\maj{s_3} \not\frown \maj{s_5}}$ because $\{a_1, a_2\} \in \maj{s_3}$ and
$\{a_3, a_4, a_5\} \in \maj{s_5}$ do not intersect. Thus the majority subsets
of $s_3$ and $s_4$ are sufficiently similar to be consecutive configurations,
but those of $s_3$ and $s_5$ are not.

The intuition that $Q_{e-1} \frown Q_e$ captures that $Q_{e-1}$ and $Q_e$ are
similar breaks down if either is empty because $\varnothing \frown Q$ vacuously
for all configurations $Q$. Clearly if there is an instance $j$ such that
$Q_{e(j)} = \varnothing$ then no value can ever be chosen for $j$, but an epoch
may be skipped at instance $i$ by letting $e(i) = e(i-1) + 2$.  This recovers
the ability to perform arbitrary configuration changes in a single step as in
Dynamic Paxos: if the system is currently using configuration $Q_{e(i)}$ and an
operator wishes to change to an unrelated configuration $Q'$ (i.e.  $Q_{e(i)}
\not\frown Q' \frown Q'$) then they can set $Q_{e(i)+1} = \varnothing$ and
$Q_{e(i)+2} = Q'$, which satisfies that \[Q_{e(i)} \frown Q_{e(i) + 1} \frown
Q_{e(i)+1} \frown Q_{e(i) + 2} \frown Q_{e(i) + 2}\] as desired. However,
re-running phase I can only be avoided if the epoch increases by $1$ and in
this situation the epoch increases by $2$, so the new phase I must be complete
before any new phase II instances can be started which reintroduces the risk of
jitter.  As in Dynamic Paxos, to mitigate this risk the operator chooses an
$\alpha > 0$ and sets $e(i+\alpha) = e(i)+2$ and $e(j) = e(i)$ for $i < j \le i
+ \alpha$, effectively delaying the configuration change for $\alpha$ instances
in the hope that this is long enough to have completed the new phase I.

\subsection{Weight-based configurations} \label{weight-based-configurations}

Let a \textit{weight} function be a function $w : \mathbb A \to \mathbb N$ that
only takes finitely many nonzero values. This can be used to define a
configuration $M(w)$ by \textit{majority}: \[M(w) = \left\{ q \;\middle|\;
\sum_{a \in q} 2 w(a) > \sum_{a \in \mathbb A} w(a) \right\}.\] If $w(a) = 0$
for all $a$ then $w$ is said to be \textit{weightless} and $M(w) =
\varnothing$.  Corollary \ref{weights-equal} shows that $M(w) \frown M(w)$ for
any weight function $w$.

A sequence $w_0, w_1, \ldots$ of weight functions can be used to define the
sequence of configurations $M(w_0), M(w_1), \ldots$ used by a cluster, which
must satisfy $M(w_{e-1}) \frown M(w_e)$ for all $e$.  There are three simple
ways to ensure this:

\begin{enumerate}

\item If $w_{e-1}$ and $w_e$ differ by a constant factor, in the sense that
there are positive integers $m$ and $n$ such that $m w_{e-1}(a) = n w_e(a)$ for
all $a$, then clearly ${M(w_{e-1}) = M(w_e)}$ and hence the property holds.

\item By theorem \ref{weights-nearly-equal} if $w_{e-1}$ and $w_e$ are equal
except perhaps at a single node at which they differ by $1$ then the property
holds.

\item Finally, if $w_{e-1}$ is weightless then the property holds vacuously.

\end{enumerate}
%
Theorem \ref{weights-nearly-equal} is reminiscent of the amoeba analogy in
\cite{cheap-paxos}.

\subsection{Preserving invariants}

In order to ensure consistency, a running Paxos cluster must satisfy all the
invariants listed in appendix \ref{paxos-invariants} at all times. If no
messages have been sent then the invariants all certainly hold, so as long as
each message that each node sends preserves them then they will always hold by
induction. Notice that all the invariants limit which messages may be sent but
do not insist that any messages are sent at all: this section ignores liveness
and focuses only on safety.

\subsubsection{Prepare messages} Since $\prep{b}$ is not mentioned in appendix
\ref{paxos-invariants} any node may send a $\prep{b}$ message without affecting
consistency.

\subsubsection{Promises and acceptances} The node $a$ may emit promises
$\fprom{i}{a}{b}$ etc. and acceptances $\acc{i}{a}{b}$ where they do not
conflict with previously-sent messages. In more detail: \begin{itemize} \item
It may emit $\mprom{i}{a}{b}$ if it has previously sent no $\acc{i'}{a}{b'}$
for any $b'$ and any $i' \le i$. \item It may emit $\fprom{i}{a}{b}$ if it has
previously sent no $\acc{i}{a}{b'}$ for any $b'$. \item It may emit
$\bprom{i}{a}{b}{b'}$ if $\acc{i}{a}{b'}$, $b' \prec b$ and $b'$ is the
greatest such ballot. \item It may emit $\acc{i}{a}{b}$ as long as
$\prop{i}{b}$ and it does not violate any previously-made promises, in the
sense that whenever \begin{itemize} \item $\mprom{i'}{a}{b'}$ for any $i' \le
i$, or \item $\fprom{i}{a}{b'}$, or \item $\bprom{i}{a}{b'}{b''}$\end{itemize}
it is the case that $b' \preceq b$. \end{itemize}

\subsubsection{Proposing values} From the discussion in \ref{value-function}
above the value $v_i(b)$ of ballot $b$ in instance $i$ may be changed without
affecting any invariants as long as $\neg\prop{b}{i}$. So as to ensure this,
each ballot $b$ has a unique node $\owner{b}$ which is the only one that may
send out $\prop{b}{i}$ and which is therefore responsible for the value
$v_i(b)$ for each $i$. If $\owner{b}$ has already sent $\prop{b}{i}$ then it
may not change the value $v_i(b)$ but if desired it may send out $\prop{b}{i}$
again, since messages may be duplicated. On the other hand if it has not sent
out $\prop{b}{i}$ then it may change $v_i(b)$ freely. Once it has received
promises for $b$ in instance $i$ from every node in a quorum $q \in Q_{e(b)}$
where $e(b) \le e(i)$ then it may send out $\prop{b}{i}$. If any of the
promises from nodes in $q$ is of the form $\bprom{i}{a}{b}{b'}$ then it must
first set \[v_i(b) = v_i\bigl(\mathrm{max} \{b' \mid \bprom{i}{a}{b}{b'}
\textrm{ received}, a \in q \}\bigr).\] If $i > i_\mathrm{max}$ so the value of
$e(i)$ is not known then it is sufficient to find an earlier instance $i' < i$
and ballot $b'$ such that $\chosen{i'}{b'}$ and $e(b) \le e(i')$ because $e$
is nondecreasing.

\subsubsection{Choosing values} Any node may deduce $\chosen{b}{i}$ if
\begin{itemize} \item $e(i) \le e(b) + 1$, \item for each $j \le i - \kappa$
there is a $b'$ with $\chosen{j}{b'}$, and \item there is a quorum $q \in
Q_{e(i)}$ such that $\acc{i}{a}{b}$ has been received from every $a \in q$.
\end{itemize}

\subsubsection{Proposal identifiers} Recall that $\mathbb B$ requires a
wellfounded total order $\prec$, a function $\mathrm{owner} : \mathbb B \to
\mathbb A$ and a nondecreasing function $e : \mathbb B \to \mathbb N$. If
$\mathbb A$ is also wellordered then one possible implementation is $\mathbb N
\times \mathbb N \times \mathbb A$ with the lexicographic order, where
$\owner{\langle e, n, a \rangle} = a$ and $e(\langle e, n, a\rangle) = e$.

\section{Discussion}\label{discussion}

Typically the nodes in a system must be taken offline for maintenance from time
to time. Since Paxos is designed to be resilient to failures of nodes, a simple
approach is to treat offline nodes just as if they have failed. By taking each
node offline in turn, the system can remain available even while maintenance is
being performed.

However, a na\"ive approach to this puts the system at risk during the upgrade:
in a cluster of three equally-weighted nodes if one of the nodes is taken
offline then the failure of another node would render the system unavailable.
More generally, if the system is to be resilient to $\le f$ failures then a
node $a_{\textrm{maint}}$ may be safely taken offline if all other nodes have
weight $< \sum_{a \in \mathbb A} \frac{w(a)}{2f} - w(a_{\textrm{maint}})$. For
example, if $f = 1$ and all nodes have equal weights then the smallest cluster
in which a node may be safely taken offline for maintenance has five nodes.
Assuming that the cost of operating a cluster is proportional to the number of
nodes, this represents a significant overhead simply to cover relatively rare
periods of maintenance: if maintenance is ignored then only three nodes are
required.

The required number of nodes can be reduced using a configuration change as
follows. A cluster of four equally-weighted nodes can only tolerate a single
failure, but by reducing the weight of node $a_{\textrm{maint}}$ to zero before
taking it offline, the remaining three nodes can tolerate the loss of another
node. However, a cluster of four equally-weighted nodes has no nodes with a
casting vote, so instead two of the nodes must have weight $1$ and the other
two have weight $2$, and then the sequence of weights used for this maintenance
procedure is as follows, with maintenance on $a_{\textrm{maint}}$ occurring
during epoch ${e+1}$:
\[\begin{array}{rcccc}
\textrm{node}&a_{\textrm{maint}}&a_1&a_2&a_3 \\
w_e&1&1&2&2\\
w_{e+1}&0&1&2&2\\
w_{e+2}&1&1&2&2\\
\end{array}\]
Since each step changes the weight of a single node by $1$ this sequence
satisfies $M(w_e) \frown M(w_{e+1}) \frown M(w_{e+2})$, and $a_3$ has a casting
vote in each configuration, so this sequence is jitter-free.  If $a_3$ required
maintenance then its casting vote could be moved to $a_1$ first as follows:
\[\begin{array}{rcccc}
\textrm{node}&a_0&a_1&a_2&a_3 \\
w_e&1&1&2&2\\
w_{e+1}&1&2&2&2\\
w_{e+2}&1&2&2&1\\
\end{array}\]
Here $a_2$ has a casting vote throughout so this is also jitter-free.

In many situations it is feasible simply to replace a node $a_{\textrm{old}}$
with a new node $a_{\textrm{new}}$ rather than to maintain it in-situ. This
technique is known as a \textit{migration}.  If $a_{\textrm{new}}$ is brought
online before $a_{\textrm{old}}$ is taken offline then the system is not put at
risk, requiring four nodes to perform maintenance but only three in normal
operation:
\[\begin{array}{rcccc}
\textrm{node}&a_{\textrm{old}}&a_{\textrm{new}}&a_1&a_2 \\
w_e&1&0&2&2\\
w_{e+1}&1&1&2&2\\
w_{e+2}&0&1&2&2\\
\end{array}\]
Again, each step changes the weight of a single node by $1$ so this sequence
satisfies $M(w_e) \frown M(w_{e+1}) \frown M(w_{e+2})$ and $a_2$ has a casting
vote throughout so this sequence is jitter-free. As above, the casting vote can
be moved from $a_2$ to $a_0$ using the following sequence:
\[\begin{array}{rcccc}
\textrm{node}&a_0&a_1&a_2 \\
w_e&1&2&2\\
w_{e+1}&2&2&2\\
w_{e+2}&2&2&1\\
\end{array}\]
It may be more usual to run a cluster of three nodes with equal weights as in
$w_{e+1}$ above. If the weights are all equal to $k \ne 2$, recall that
changing the weights of all nodes by a constant factor does not change the
cluster configuration, so multiplying by $\frac{2}{k}$ has the desired effect:
\[\begin{array}{rcccc}
\textrm{node}&a_0&a_1&a_2 \\
w_e&k&k&k\\
w_{e+1}&2&2&2\\
\end{array}\]

Assuming that node failures are independent, the probability of a cluster-wide
failure can appear to be extremely small. However, in a real-world system node
failures are not always independent: certain sets of nodes may share critical
infrastructure such as power or network connectivity, and a failure of this
infrastructure would cause a simultaneous failure of all the dependent nodes.
To control this sitation, the cluster may be divided into \textit{zones}, where
the nodes within a single zone may share infrastructure but failures in
different zones can be considered to be independent. In a traditional data
centre environment, individual racks are often assumed to be independent, so
each rack could be a separate zone. If geographical redundancy is considered
then the cluster will be separated across multiple data centres, each of which
is an independent zone, and many cloud environments also divide their services
into zones in this way.  For the system to be resilient to simultaneous
zone-wide failures of $f_z > 0$ zones, it must have nodes running in at least
$2f_z + 1 \ge 3$ zones.

The replacement process described above requires a minimal three zones
throughout as it is safe to have $a_\mathrm{old}$ and $a_\mathrm{new}$ in the
same zone. This is important as in many operating environments it may be too
expensive or complicated to arrange for four independent zones\footnote{For
instance, at time of writing, only one Amazon Web Services region
(\texttt{us-east-1}) has four independent zones, whereas four of them have
three: \texttt{ap-southeast-2}, \texttt{eu-west-1}, \texttt{sa-east-1} and
\texttt{us-west-2}. Similarly, only one Google Cloud Platform region
(\texttt{us-central1}) has four zones and all the others have three zones.},
particularly if the fourth zone is only required to ensure consistency in
relatively rare periods of maintenance.

% An example of a floating figure using the graphicx package.
% Note that \label must occur AFTER (or within) \caption.
% For figures, \caption should occur after the \includegraphics.
% Note that IEEEtran v1.7 and later has special internal code that
% is designed to preserve the operation of \label within \caption
% even when the captionsoff option is in effect. However, because
% of issues like this, it may be the safest practice to put all your
% \label just after \caption rather than within \caption{}.
%
% Reminder: the "draftcls" or "draftclsnofoot", not "draft", class
% option should be used if it is desired that the figures are to be
% displayed while in draft mode.
%
%\begin{figure}[!t]
%\centering
%\includegraphics[width=2.5in]{myfigure}
% where an .eps filename suffix will be assumed under latex, 
% and a .pdf suffix will be assumed for pdflatex; or what has been declared
% via \DeclareGraphicsExtensions.
%\caption{Simulation results for the network.}
%\label{fig_sim}
%\end{figure}

% Note that the IEEE typically puts floats only at the top, even when this
% results in a large percentage of a column being occupied by floats.


% An example of a double column floating figure using two subfigures.
% (The subfig.sty package must be loaded for this to work.)
% The subfigure \label commands are set within each subfloat command,
% and the \label for the overall figure must come after \caption.
% \hfil is used as a separator to get equal spacing.
% Watch out that the combined width of all the subfigures on a 
% line do not exceed the text width or a line break will occur.
%
%\begin{figure*}[!t]
%\centering
%\subfloat[Case I]{\includegraphics[width=2.5in]{box}%
%\label{fig_first_case}}
%\hfil
%\subfloat[Case II]{\includegraphics[width=2.5in]{box}%
%\label{fig_second_case}}
%\caption{Simulation results for the network.}
%\label{fig_sim}
%\end{figure*}
%
% Note that often IEEE papers with subfigures do not employ subfigure
% captions (using the optional argument to \subfloat[]), but instead will
% reference/describe all of them (a), (b), etc., within the main caption.
% Be aware that for subfig.sty to generate the (a), (b), etc., subfigure
% labels, the optional argument to \subfloat must be present. If a
% subcaption is not desired, just leave its contents blank,
% e.g., \subfloat[].


% An example of a floating table. Note that, for IEEE style tables, the
% \caption command should come BEFORE the table and, given that table
% captions serve much like titles, are usually capitalized except for words
% such as a, an, and, as, at, but, by, for, in, nor, of, on, or, the, to
% and up, which are usually not capitalized unless they are the first or
% last word of the caption. Table text will default to \footnotesize as
% the IEEE normally uses this smaller font for tables.
% The \label must come after \caption as always.
%
%\begin{table}[!t]
%% increase table row spacing, adjust to taste
%\renewcommand{\arraystretch}{1.3}
% if using array.sty, it might be a good idea to tweak the value of
% \extrarowheight as needed to properly center the text within the cells
%\caption{An Example of a Table}
%\label{table_example}
%\centering
%% Some packages, such as MDW tools, offer better commands for making tables
%% than the plain LaTeX2e tabular which is used here.
%\begin{tabular}{|c||c|}
%\hline
%One & Two\\
%\hline
%Three & Four\\
%\hline
%\end{tabular}
%\end{table}


% Note that the IEEE does not put floats in the very first column
% - or typically anywhere on the first page for that matter. Also,
% in-text middle ("here") positioning is typically not used, but it
% is allowed and encouraged for Computer Society conferences (but
% not Computer Society journals). Most IEEE journals/conferences use
% top floats exclusively. 
% Note that, LaTeX2e, unlike IEEE journals/conferences, places
% footnotes above bottom floats. This can be corrected via the
% \fnbelowfloat command of the stfloats package.




%\section{Conclusion}
%TODO

\vfill

\pagebreak

\appendices
\section{Consistency of the Synod algorithm}
\label{synod-safety}

The consistency property of the Synod algorithm is derived from the following
invariants.

\begin{enumerate}

\item \label{synod-quorums} For $b_1 \succ b_2 \in \mathbb B$ there are
configurations $Q^\textrm{I}(b_1)$ and $Q^\textrm{II}(b_2) \subseteq \mathcal P
\mathbb A$ such that if $\prop{}{b_1}$ and $\chosen{}{b_2}$ then
${Q^\textrm{I}(b_1) \frown Q^\textrm{II}(b_2)}$.

\item \label{synod-fprom} $\fprom{}{a}{b}$ only if $\neg\acc{}{a}{b'}$ for ${b'
\prec b}$.

\item \label{synod-bprom} $\bprom{}{a}{b}{b'}$ only if $b' \prec b$,
$\acc{}{a}{b'}$, and $b'$ is the greatest such ballot, in the sense that
$\neg\acc{}{a}{b''}$ for all $b''$ having $b' \prec b'' \prec b$.

\item \label{synod-prop} $\prop{}{b}$ only if there is a quorum $q^\textrm{I}
\in Q^\textrm{I}(b)$ such that for every node $a \in q^\textrm{I}$ either
$\fprom{}{a}{b}$ or else $\exists b'.  \bprom{}{a}{b}{b'}$, and if ${P = \{ b'
\mid \exists a \in q^\textrm{I}. \bprom{}{a}{b}{b'} \} \ne \varnothing}$ then
it follows that $v(b) = v(\mathrm{max}(P))$.

\item \label{synod-acc} $\acc{}{a}{b}$ only if $\prop{}{b}$.

\item \label{synod-chosen} $\chosen{}{b}$ only if there is a quorum
$q^\textrm{II} \in Q^\textrm{II}(b)$ such that $\acc{}{a}{b}$ for every $a \in
q^\textrm{II}$.

\end{enumerate}

\begin{lemma}\label{synod-acc-bprom}If $\acc{}{a}{b_2}$,
$\bprom{}{a}{b_1}{b_3}$ and $b_2 \prec b_1$ then $b_2 \preceq b_3$.\end{lemma}

\begin{proof} From invariant \ref{synod-bprom} it follows that $b_3$ is the
largest ballot such that $b_3 \prec b_1$ and $\acc{}{a}{b_3}$, but $b_2$ is
also such a ballot and therefore $b_2 \preceq b_3$ as required.  \end{proof}

\begin{lemma}\label{synod-lemma} If $\chosen{}{b_2}$, $\prop{}{b_1}$ and $b_2
\prec b_1$ then $v(b_1) = v(b_2)$. \end{lemma}

\begin{proof}Suppose for a contradiction the result is false, and since $\prec$
is wellfounded suppose without loss of generality that $b_1$ is the minimal
ballot at which it is false.  Since $\chosen{}{b_2}$, by invariant
\ref{synod-chosen} there is a quorum $q^\textrm{II} \in Q^\textrm{II}(b_2)$
such that $\acc{}{a}{b_2}$ for every $a \in q^\textrm{II}$.  By invariant
\ref{synod-fprom} it cannot be that $\fprom{}{a}{b_1}$ for any $a \in
q^\textrm{II}$.  Also, since $\prop{}{b_1}$, by invariant \ref{synod-prop}
there is a quorum $q^\textrm{I} \in Q^\textrm{I}(b_1)$ such that either
$\fprom{}{a}{b_1}$ or $\exists b'.  \bprom{}{a}{b_1}{b'}$ for all $a \in
q^\textrm{I}$.  Let $P = \{ b' \mid \exists a \in q^\textrm{I}.
\bprom{}{a}{b_1}{b'} \}$.  By invariant \ref{synod-quorums},
${Q^\textrm{I}(b_1) \frown Q^\textrm{II}(b_2)}$ and hence $q^\textrm{I} \cap
q^\textrm{II} \ne \varnothing$ so it follows that $P \ne \varnothing$, which
means that $v(b_1) = v(\mathrm{max}(P))$ by invariant \ref{synod-prop}. Let
$a_{\mathrm{max}} \in q^\textrm{I}$ be such that
$\bprom{}{a_{\mathrm{max}}}{b_1}{\mathrm{max}(P)}$.  By invariant
\ref{synod-bprom} it follows that $\mathrm{max}(P) \prec b_1$ and also that
$\acc{}{a_{\mathrm{max}}}{\mathrm{max}(P)}$ and hence
$\prop{}{\mathrm{max}(P)}$ by invariant \ref{synod-acc}. Furthermore by lemma
\ref{synod-acc-bprom} it follows that $b_2 \preceq \mathrm{max}(P)$ and since
$b_1$ was assumed to be the smallest counterexample it must be that
$v(\mathrm{max}(P)) = v(b_2)$.  Hence $v(b_1) = v(b_2)$ which is a
contradiction as required.  \end{proof}

\begin{theorem}\label{synod-safety-theorem} If $\chosen{}{b_1}$ and
$\chosen{}{b_2}$ then $v(b_1) = v(b_2)$.  \end{theorem}

\begin{proof} Without loss of generality assume that ${b_2 \prec b_1}$. By
invariant \ref{synod-chosen} there is a quorum $q \in Q^\textrm{II}(b_1)$ such
that $\acc{}{a}{b_1}$ for every node $a \in q$ and therefore $\prop{}{b_1}$ by
invariant \ref{synod-acc}.  Therefore by lemma \ref{synod-lemma} it follows
that $v(b_1) = v(b_2)$ as required.  \end{proof}

\section{Consistency of the Paxos algorithm}\label{paxos-invariants}

The consistency property of the Paxos algorithm is derived from the following
invariants.

\begin{enumerate}

\item\label{paxos-quorums} There is a sequence of configurations $Q_0, Q_1,
\ldots$ where ${Q_e \frown Q_e \frown Q_{e+1}}$ for each $e$.

\item\label{paxos-mprom} $\mprom{i}{a}{b}$ only if $\neg\acc{i'}{a}{b'}$ for
all $i' \ge i$ and $b' \prec b$.

\item\label{paxos-fprom} $\fprom{i}{a}{b}$ only if $\neg\acc{i}{a}{b'}$ for all
$b' \prec b$.

\item\label{paxos-bprom} $\bprom{i}{a}{b}{b'}$ only if \begin{itemize} \item
$b' \prec b$ \item $\acc{i}{a}{b'}$, and \item $b'$ is the greatest such
ballot, in the sense that $\neg \acc{i}{a}{b''}$ for all $b''$ having $b'
\prec b'' \prec b$. \end{itemize}

\item\label{paxos-prop} $\prop{i}{b}$ only if $e(b) \le e(i)$ and there is a
quorum $q \in Q_{e(b)}$ such that
\begin{itemize}
\item for every $a \in q$ one of the following holds:
%
\begin{itemize}
\item $\mprom{i'}{a}{b}$ for some $i' \le i$, or
\item $\fprom{i}{a}{b}$, or
\item $\bprom{i}{a}{b}{b'}$ for some $b'$, and
\end{itemize}

\item if $P = \{ b' \mid \exists a \in q. \bprom{i}{a}{b}{b'} \} \ne
\varnothing$ then $v_i(b) = v_i(\mathrm{max}(P))$.

\end{itemize}

\item \label{paxos-acc} $\acc{i}{a}{b}$ only if $\prop{i}{b}$.

\item \label{paxos-chosen} $\chosen{i}{b}$ only if \begin{itemize} \item for
each $j \le i - \kappa$ there is a $b'$ with $\chosen{i-1}{b'}$, \item $e(i)
\le e(b) + 1$, and \item there is a quorum $q \in Q_{e(i)}$ with
$\acc{i}{a}{b}$ for every $a \in q$.  \end{itemize}

\end{enumerate}

\begin{lemma}\label{paxos-synod-quorum-invariant} For $b_1 \succ b_2 \in
\mathbb B$, if $\prop{i}{b_1}$ and $\chosen{i}{b_2}$ then ${Q_{e(b_1)} \frown
Q_{e(i)}}$.  \end{lemma}

\begin{proof} $e(i) \le e(b_2) + 1 \le e(b_1) + 1 \le e(i) + 1$ since
$\chosen{i}{b_2}$, $e$ is nondecreasing, and $\prop{i}{b_1}$ respectively.
Therefore $e(b_1) \in \{ e(i) - 1, e(i) \}$ and hence ${Q_{e(b_1)} \frown
Q_{e(i)}}$ by invariant \ref{paxos-quorums}.  \end{proof}

\begin{theorem}\label{paxos-safety-theorem} If $\chosen{i}{b_1}$ and
$\chosen{i}{b_2}$ then ${v_i(b_1) = v_i(b_2)}$.  \end{theorem}

\begin{proof} If $\chosen{i}{b_1}$ then the Paxos invariants imply the Synod
invariants for instance $i$.  In more detail, let
\[\begin{array}{rl}
Q^\textrm{I}(b) &= Q_{e(b)}, \\
Q^\textrm{II}(b) &= Q_{e(i)}, \\
\fprom{}{a}{b} &= \fprom{i}{a}{b} \\
& \ {}\vee \exists i' \le i. \mprom{i'}{a}{b}, \\
\bprom{}{a}{b}{b'} &= \bprom{i}{a}{b}{b'}, \\
\prop{}{b} &= \prop{i}{b}, \\
\acc{}{a}{b} &= \acc{i}{a}{b} \textrm{ and} \\
\chosen{}{b} &= \chosen{i}{b}. \\
v(b) &= v_i(b)\\
\end{array}
\]
Synod's invariant \ref{synod-quorums} follows from lemma
\ref{paxos-synod-quorum-invariant} and the remaining invariants are simple to
show so by theorem \ref{synod-safety-theorem} it follows that $v_i(b_1) =
v_i(b_2)$ as required.  \end{proof}

\section{Consecutive epochs using weight-based configurations}

\begin{theorem} \label{weights-nearly-equal} If $w, w' : \mathbb A \to \mathbb
N$ are weight functions such that $\sum_{a \in \mathbb A} |w'(a) - w(a)| \le 1$
then $M(w) \frown M(w')$.  \end{theorem}

\begin{proof}Since $w$ and $w'$ take integer values, they must differ at at
most one node by at most $1$. Let $a_0$ be the node such that $|w'(a_0) -
w(a_0)| \le 1$ and $w(a) = w'(a)$ for all $a \ne a_0$.
%
Let $q \in M(w)$ and $q' \in M(w')$. By the definition of $M$, and since $w$
and $w'$ take only integer values,
%
$\sum_{a \in q} 2 w(a) \ge \sum_{a \in \mathbb A} w(a) + 1$
%
and
%
$\sum_{a \in q'} 2 w'(a) \ge \sum_{a \in \mathbb A} w'(a) + 1$.
%
Let $d_{\mathbb A} = w'(a_0) - w(a_0)$ so that $\sum_{a \in \mathbb A} w'(a) =
\sum_{a \in \mathbb A} w(a) + d_{\mathbb A}$ and $|d_\mathbb A| \le 1$. Also
let \[ d_{q'} =
\begin{cases}
%
d_{\mathbb A} & a_0 \in q' \\
%
0 & \textrm{otherwise,}
%
\end{cases}
\]
so that $\sum_{a \in q'} w'(a) = \sum_{a \in q'} w(a) + d_{q'}$.
%
Then
\begin{gather*}
%
\sum_{a \in \mathbb A} 2w(a) + d_{\mathbb A} + 2 \\
%
\begin{aligned}
%
&= \left( \sum_{a \in \mathbb A} w(a)  + 1\right)
+  \left( \sum_{a \in \mathbb A} w'(a) + 1\right) \\
%
&\le \sum_{a \in q}  2w(a)
+    \sum_{a \in q'} 2w'(a) \\
%
&= \sum_{a \in q}  2w(a)
+  \sum_{a \in q'} 2w(a) + 2d_{q'}\\
%
&= \sum_{a \in q \cup q'} 2w(a)
+  \sum_{a \in q \cap q'} 2w(a) + 2d_{q'}\\
%
&\le \sum_{a \in \mathbb A} 2w(a)
+    \sum_{a \in q \cap q'} 2w(a) + 2d_{q'},\\
%
\end{aligned}\end{gather*} so that $\sum_{a \in q \cap q'} 2w(a) \ge d_{\mathbb
A} + 2 - 2d_{q'} = 2 \pm d_\mathbb A \ge 1$ and hence $q \cap q' \ne
\varnothing$ as desired.  \end{proof}

\begin{corollary} \label{weights-equal} If $w : \mathbb A \to \mathbb N$ is a
weight function then \[M(w) \frown M(w).\]  \end{corollary}

\begin{proof} This is a special case of theorem \ref{weights-nearly-equal},
where $w' = w$.  \end{proof}

% use section* for acknowledgment
%\section*{Acknowledgment} TODO


%The authors would like to thank...


% Can use something like this to put references on a page
% by themselves when using endfloat and the captionsoff option.
\ifCLASSOPTIONcaptionsoff
  \newpage
\fi



% trigger a \newpage just before the given reference
% number - used to balance the columns on the last page
% adjust value as needed - may need to be readjusted if
% the document is modified later
%\IEEEtriggeratref{8}
% The "triggered" command can be changed if desired:
%\IEEEtriggercmd{\enlargethispage{-5in}}

% references section

% can use a bibliography generated by BibTeX as a .bbl file
% BibTeX documentation can be easily obtained at:
% http://mirror.ctan.org/biblio/bibtex/contrib/doc/
% The IEEEtran BibTeX style support page is at:
% http://www.michaelshell.org/tex/ieeetran/bibtex/
\bibliographystyle{IEEEtran}
% argument is your BibTeX string definitions and bibliography database(s)
\bibliography{jitter-free-membership}
%
% <OR> manually copy in the resultant .bbl file
% set second argument of \begin to the number of references
% (used to reserve space for the reference number labels box)

% biography section
% 
% If you have an EPS/PDF photo (graphicx package needed) extra braces are
% needed around the contents of the optional argument to biography to prevent
% the LaTeX parser from getting confused when it sees the complicated
% \includegraphics command within an optional argument. (You could create
% your own custom macro containing the \includegraphics command to make things
% simpler here.)
%\begin{IEEEbiography}[{\includegraphics[width=1in,height=1.25in,clip,keepaspectratio]{mshell}}]{Michael Shell}
% or if you just want to reserve a space for a photo:

%\begin{IEEEbiography}{Michael Shell}
%Biography text here.
%\end{IEEEbiography}
%
%% if you will not have a photo at all:
%\begin{IEEEbiographynophoto}{John Doe}
%Biography text here.
%\end{IEEEbiographynophoto}
%
%% insert where needed to balance the two columns on the last page with
%% biographies
%%\newpage
%
%\begin{IEEEbiographynophoto}{Jane Doe}
%Biography text here.
%\end{IEEEbiographynophoto}

% You can push biographies down or up by placing
% a \vfill before or after them. The appropriate
% use of \vfill depends on what kind of text is
% on the last page and whether or not the columns
% are being equalized.

%\vfill

% Can be used to pull up biographies so that the bottom of the last one
% is flush with the other column.
%\enlargethispage{-5in}



% that's all folks
\end{document}


