
%% bare_jrnl.tex
%% V1.4b
%% 2015/08/26
%% by Michael Shell
%% see http://www.michaelshell.org/
%% for current contact information.
%%
%% This is a skeleton file demonstrating the use of IEEEtran.cls
%% (requires IEEEtran.cls version 1.8b or later) with an IEEE
%% journal paper.
%%
%% Support sites:
%% http://www.michaelshell.org/tex/ieeetran/
%% http://www.ctan.org/pkg/ieeetran
%% and
%% http://www.ieee.org/

%%*************************************************************************
%% Legal Notice:
%% This code is offered as-is without any warranty either expressed or
%% implied; without even the implied warranty of MERCHANTABILITY or
%% FITNESS FOR A PARTICULAR PURPOSE! 
%% User assumes all risk.
%% In no event shall the IEEE or any contributor to this code be liable for
%% any damages or losses, including, but not limited to, incidental,
%% consequential, or any other damages, resulting from the use or misuse
%% of any information contained here.
%%
%% All comments are the opinions of their respective authors and are not
%% necessarily endorsed by the IEEE.
%%
%% This work is distributed under the LaTeX Project Public License (LPPL)
%% ( http://www.latex-project.org/ ) version 1.3, and may be freely used,
%% distributed and modified. A copy of the LPPL, version 1.3, is included
%% in the base LaTeX documentation of all distributions of LaTeX released
%% 2003/12/01 or later.
%% Retain all contribution notices and credits.
%% ** Modified files should be clearly indicated as such, including  **
%% ** renaming them and changing author support contact information. **
%%*************************************************************************


% *** Authors should verify (and, if needed, correct) their LaTeX system  ***
% *** with the testflow diagnostic prior to trusting their LaTeX platform ***
% *** with production work. The IEEE's font choices and paper sizes can   ***
% *** trigger bugs that do not appear when using other class files.       ***                          ***
% The testflow support page is at:
% http://www.michaelshell.org/tex/testflow/



\documentclass[journal]{IEEEtran}
%
% If IEEEtran.cls has not been installed into the LaTeX system files,
% manually specify the path to it like:
% \documentclass[journal]{../sty/IEEEtran}





% Some very useful LaTeX packages include:
% (uncomment the ones you want to load)


% *** MISC UTILITY PACKAGES ***
%
%\usepackage{ifpdf}
% Heiko Oberdiek's ifpdf.sty is very useful if you need conditional
% compilation based on whether the output is pdf or dvi.
% usage:
% \ifpdf
%   % pdf code
% \else
%   % dvi code
% \fi
% The latest version of ifpdf.sty can be obtained from:
% http://www.ctan.org/pkg/ifpdf
% Also, note that IEEEtran.cls V1.7 and later provides a builtin
% \ifCLASSINFOpdf conditional that works the same way.
% When switching from latex to pdflatex and vice-versa, the compiler may
% have to be run twice to clear warning/error messages.





% *** CITATION PACKAGES ***
%
\usepackage{cite}
% cite.sty was written by Donald Arseneau
% V1.6 and later of IEEEtran pre-defines the format of the cite.sty package
% \cite{} output to follow that of the IEEE. Loading the cite package will
% result in citation numbers being automatically sorted and properly
% "compressed/ranged". e.g., [1], [9], [2], [7], [5], [6] without using
% cite.sty will become [1], [2], [5]--[7], [9] using cite.sty. cite.sty's
% \cite will automatically add leading space, if needed. Use cite.sty's
% noadjust option (cite.sty V3.8 and later) if you want to turn this off
% such as if a citation ever needs to be enclosed in parenthesis.
% cite.sty is already installed on most LaTeX systems. Be sure and use
% version 5.0 (2009-03-20) and later if using hyperref.sty.
% The latest version can be obtained at:
% http://www.ctan.org/pkg/cite
% The documentation is contained in the cite.sty file itself.






% *** GRAPHICS RELATED PACKAGES ***
%
\ifCLASSINFOpdf
  \usepackage[pdftex]{graphicx}
  % declare the path(s) where your graphic files are
  \graphicspath{{genfigs/}}
  % and their extensions so you won't have to specify these with
  % every instance of \includegraphics
  \DeclareGraphicsExtensions{.pdf}
\else
  % or other class option (dvipsone, dvipdf, if not using dvips). graphicx
  % will default to the driver specified in the system graphics.cfg if no
  % driver is specified.
  \usepackage[dvips]{graphicx}
  % declare the path(s) where your graphic files are
  \graphicspath{{genfigs/}}
  % and their extensions so you won't have to specify these with
  % every instance of \includegraphics
  \DeclareGraphicsExtensions{.eps}
\fi
% graphicx was written by David Carlisle and Sebastian Rahtz. It is
% required if you want graphics, photos, etc. graphicx.sty is already
% installed on most LaTeX systems. The latest version and documentation
% can be obtained at: 
% http://www.ctan.org/pkg/graphicx
% Another good source of documentation is "Using Imported Graphics in
% LaTeX2e" by Keith Reckdahl which can be found at:
% http://www.ctan.org/pkg/epslatex
%
% latex, and pdflatex in dvi mode, support graphics in encapsulated
% postscript (.eps) format. pdflatex in pdf mode supports graphics
% in .pdf, .jpeg, .png and .mps (metapost) formats. Users should ensure
% that all non-photo figures use a vector format (.eps, .pdf, .mps) and
% not a bitmapped formats (.jpeg, .png). The IEEE frowns on bitmapped formats
% which can result in "jaggedy"/blurry rendering of lines and letters as
% well as large increases in file sizes.
%
% You can find documentation about the pdfTeX application at:
% http://www.tug.org/applications/pdftex





% *** MATH PACKAGES ***
%
\usepackage{amsmath}
\usepackage{amssymb}

\usepackage{amsthm}
\newtheorem{theorem}{Theorem}
\newtheorem{lemma}[theorem]{Lemma}
\newtheorem{corollary}[theorem]{Corollary}
% A popular package from the American Mathematical Society that provides
% many useful and powerful commands for dealing with mathematics.
%
% Note that the amsmath package sets \interdisplaylinepenalty to 10000
% thus preventing page breaks from occurring within multiline equations. Use:
%\interdisplaylinepenalty=2500
% after loading amsmath to restore such page breaks as IEEEtran.cls normally
% does. amsmath.sty is already installed on most LaTeX systems. The latest
% version and documentation can be obtained at:
% http://www.ctan.org/pkg/amsmath





% *** SPECIALIZED LIST PACKAGES ***
%
%\usepackage{algorithmic}
% algorithmic.sty was written by Peter Williams and Rogerio Brito.
% This package provides an algorithmic environment fo describing algorithms.
% You can use the algorithmic environment in-text or within a figure
% environment to provide for a floating algorithm. Do NOT use the algorithm
% floating environment provided by algorithm.sty (by the same authors) or
% algorithm2e.sty (by Christophe Fiorio) as the IEEE does not use dedicated
% algorithm float types and packages that provide these will not provide
% correct IEEE style captions. The latest version and documentation of
% algorithmic.sty can be obtained at:
% http://www.ctan.org/pkg/algorithms
% Also of interest may be the (relatively newer and more customizable)
% algorithmicx.sty package by Szasz Janos:
% http://www.ctan.org/pkg/algorithmicx




% *** ALIGNMENT PACKAGES ***
%
%\usepackage{array}
% Frank Mittelbach's and David Carlisle's array.sty patches and improves
% the standard LaTeX2e array and tabular environments to provide better
% appearance and additional user controls. As the default LaTeX2e table
% generation code is lacking to the point of almost being broken with
% respect to the quality of the end results, all users are strongly
% advised to use an enhanced (at the very least that provided by array.sty)
% set of table tools. array.sty is already installed on most systems. The
% latest version and documentation can be obtained at:
% http://www.ctan.org/pkg/array


% IEEEtran contains the IEEEeqnarray family of commands that can be used to
% generate multiline equations as well as matrices, tables, etc., of high
% quality.




% *** SUBFIGURE PACKAGES ***
%\ifCLASSOPTIONcompsoc
%  \usepackage[caption=false,font=normalsize,labelfont=sf,textfont=sf]{subfig}
%\else
%  \usepackage[caption=false,font=footnotesize]{subfig}
%\fi
% subfig.sty, written by Steven Douglas Cochran, is the modern replacement
% for subfigure.sty, the latter of which is no longer maintained and is
% incompatible with some LaTeX packages including fixltx2e. However,
% subfig.sty requires and automatically loads Axel Sommerfeldt's caption.sty
% which will override IEEEtran.cls' handling of captions and this will result
% in non-IEEE style figure/table captions. To prevent this problem, be sure
% and invoke subfig.sty's "caption=false" package option (available since
% subfig.sty version 1.3, 2005/06/28) as this is will preserve IEEEtran.cls
% handling of captions.
% Note that the Computer Society format requires a larger sans serif font
% than the serif footnote size font used in traditional IEEE formatting
% and thus the need to invoke different subfig.sty package options depending
% on whether compsoc mode has been enabled.
%
% The latest version and documentation of subfig.sty can be obtained at:
% http://www.ctan.org/pkg/subfig




% *** FLOAT PACKAGES ***
%
%\usepackage{fixltx2e}
% fixltx2e, the successor to the earlier fix2col.sty, was written by
% Frank Mittelbach and David Carlisle. This package corrects a few problems
% in the LaTeX2e kernel, the most notable of which is that in current
% LaTeX2e releases, the ordering of single and double column floats is not
% guaranteed to be preserved. Thus, an unpatched LaTeX2e can allow a
% single column figure to be placed prior to an earlier double column
% figure.
% Be aware that LaTeX2e kernels dated 2015 and later have fixltx2e.sty's
% corrections already built into the system in which case a warning will
% be issued if an attempt is made to load fixltx2e.sty as it is no longer
% needed.
% The latest version and documentation can be found at:
% http://www.ctan.org/pkg/fixltx2e


%\usepackage{stfloats}
% stfloats.sty was written by Sigitas Tolusis. This package gives LaTeX2e
% the ability to do double column floats at the bottom of the page as well
% as the top. (e.g., "\begin{figure*}[!b]" is not normally possible in
% LaTeX2e). It also provides a command:
%\fnbelowfloat
% to enable the placement of footnotes below bottom floats (the standard
% LaTeX2e kernel puts them above bottom floats). This is an invasive package
% which rewrites many portions of the LaTeX2e float routines. It may not work
% with other packages that modify the LaTeX2e float routines. The latest
% version and documentation can be obtained at:
% http://www.ctan.org/pkg/stfloats
% Do not use the stfloats baselinefloat ability as the IEEE does not allow
% \baselineskip to stretch. Authors submitting work to the IEEE should note
% that the IEEE rarely uses double column equations and that authors should try
% to avoid such use. Do not be tempted to use the cuted.sty or midfloat.sty
% packages (also by Sigitas Tolusis) as the IEEE does not format its papers in
% such ways.
% Do not attempt to use stfloats with fixltx2e as they are incompatible.
% Instead, use Morten Hogholm'a dblfloatfix which combines the features
% of both fixltx2e and stfloats:
%
\usepackage{dblfloatfix}
% The latest version can be found at:
% http://www.ctan.org/pkg/dblfloatfix




%\ifCLASSOPTIONcaptionsoff
%  \usepackage[nomarkers]{endfloat}
% \let\MYoriglatexcaption\caption
% \renewcommand{\caption}[2][\relax]{\MYoriglatexcaption[#2]{#2}}
%\fi
% endfloat.sty was written by James Darrell McCauley, Jeff Goldberg and 
% Axel Sommerfeldt. This package may be useful when used in conjunction with 
% IEEEtran.cls'  captionsoff option. Some IEEE journals/societies require that
% submissions have lists of figures/tables at the end of the paper and that
% figures/tables without any captions are placed on a page by themselves at
% the end of the document. If needed, the draftcls IEEEtran class option or
% \CLASSINPUTbaselinestretch interface can be used to increase the line
% spacing as well. Be sure and use the nomarkers option of endfloat to
% prevent endfloat from "marking" where the figures would have been placed
% in the text. The two hack lines of code above are a slight modification of
% that suggested by in the endfloat docs (section 8.4.1) to ensure that
% the full captions always appear in the list of figures/tables - even if
% the user used the short optional argument of \caption[]{}.
% IEEE papers do not typically make use of \caption[]'s optional argument,
% so this should not be an issue. A similar trick can be used to disable
% captions of packages such as subfig.sty that lack options to turn off
% the subcaptions:
% For subfig.sty:
% \let\MYorigsubfloat\subfloat
% \renewcommand{\subfloat}[2][\relax]{\MYorigsubfloat[]{#2}}
% However, the above trick will not work if both optional arguments of
% the \subfloat command are used. Furthermore, there needs to be a
% description of each subfigure *somewhere* and endfloat does not add
% subfigure captions to its list of figures. Thus, the best approach is to
% avoid the use of subfigure captions (many IEEE journals avoid them anyway)
% and instead reference/explain all the subfigures within the main caption.
% The latest version of endfloat.sty and its documentation can obtained at:
% http://www.ctan.org/pkg/endfloat
%
% The IEEEtran \ifCLASSOPTIONcaptionsoff conditional can also be used
% later in the document, say, to conditionally put the References on a 
% page by themselves.

%\usepackage{float}
%\floatstyle{boxed}
%\restylefloat{figure}




% *** PDF, URL AND HYPERLINK PACKAGES ***
%
%\usepackage{url}
% url.sty was written by Donald Arseneau. It provides better support for
% handling and breaking URLs. url.sty is already installed on most LaTeX
% systems. The latest version and documentation can be obtained at:
% http://www.ctan.org/pkg/url
% Basically, \url{my_url_here}.




% *** Do not adjust lengths that control margins, column widths, etc. ***
% *** Do not use packages that alter fonts (such as pslatex).         ***
% There should be no need to do such things with IEEEtran.cls V1.6 and later.
% (Unless specifically asked to do so by the journal or conference you plan
% to submit to, of course. )


% correct bad hyphenation here
\hyphenation{op-tical net-works semi-conduc-tor}

\def\mytitle{Unbounded Pipelining in Dynamically Reconfigurable Paxos Clusters}
\include{inc}

\begin{document}
%
% paper title
% Titles are generally capitalized except for words such as a, an, and, as,
% at, but, by, for, in, nor, of, on, or, the, to and up, which are usually
% not capitalized unless they are the first or last word of the title.
% Linebreaks \\ can be used within to get better formatting as desired.
% Do not put math or special symbols in the title.
\title{\mytitle}
%
%
% author names and IEEE memberships
% note positions of commas and nonbreaking spaces ( ~ ) LaTeX will not break
% a structure at a ~ so this keeps an author's name from being broken across
% two lines.
% use \thanks{} to gain access to the first footnote area
% a separate \thanks must be used for each paragraph as LaTeX2e's \thanks
% was not built to handle multiple paragraphs
%

\author{David C. Turner% <-this % stops a space
\thanks{Manuscript received \pubdate.}%
\thanks{The author is with the Operations and Planning Systems division of
Tracsis plc, Leeds LS2 9DF, United Kingdom (email:
d.turner@tracsis.com)}% <-this % stops a space
\thanks{Digital Object Identifier 99.9999/ZZZ.9999.999999}
%\thanks{Manuscript received April 19, 2005; revised August 26, 2015.}} TODO
}

% note the % following the last \IEEEmembership and also \thanks - 
% these prevent an unwanted space from occurring between the last author name
% and the end of the author line. i.e., if you had this:
% 
% \author{....lastname \thanks{...} \thanks{...} }
%                     ^------------^------------^----Do not want these spaces!
%
% a space would be appended to the last name and could cause every name on that
% line to be shifted left slightly. This is one of those "LaTeX things". For
% instance, "\textbf{A} \textbf{B}" will typeset as "A B" not "AB". To get
% "AB" then you have to do: "\textbf{A}\textbf{B}"
% \thanks is no different in this regard, so shield the last } of each \thanks
% that ends a line with a % and do not let a space in before the next \thanks.
% Spaces after \IEEEmembership other than the last one are OK (and needed) as
% you are supposed to have spaces between the names. For what it is worth,
% this is a minor point as most people would not even notice if the said evil
% space somehow managed to creep in.



% The paper headers
%\markboth{IEEE Transactions on Parallel and Distributed Systems,~Vol.~?, No.~?, %?~? Draft \gitid}{Turner: \mytitle}%
\markboth{Unbounded Pipelining in Dynamically Reconfigurable Paxos Clusters Draft \gitid}%TODO Journal of \LaTeX\ Class Files,~Vol.~14, No.~8, August~2015}%
{Turner: \mytitle}
% The only time the second header will appear is for the odd numbered pages
% after the title page when using the twoside option.
% 
% *** Note that you probably will NOT want to include the author's ***
% *** name in the headers of peer review papers.                   ***
% You can use \ifCLASSOPTIONpeerreview for conditional compilation here if
% you desire.




% If you want to put a publisher's ID mark on the page you can do it like
% this:
%\IEEEpubid{1045--9219~\copyright~2016 IEEE.  Personal use is permitted, but republication/redistribution requires IEEE permission.}
% Remember, if you use this you must call \IEEEpubidadjcol in the second
% column for its text to clear the IEEEpubid mark.



% use for special paper notices
%\IEEEspecialpapernotice{(Invited Paper)}




% make the title area
\maketitle

% As a general rule, do not put math, special symbols or citations in the
% abstract or keywords.

\begin{abstract}Consensus is an essential ingredient of a fault-tolerant
  distributed system systems. When equipped with a consensus algorithm a
  distributed system can act as a replicated state machine (RSM), duplicating
  its state across a cluster of redundant components to avoid the failure of
  any single component leading to a system-wide failure. Paxos and Raft are
  examples of algorithms for achieving distributed consensus. Practical
  implementations of this kind of system must support dynamic reconfiguration
  in order to be able to replace failed components and perform other
  administrative tasks without downtime. Paxos can achieve high performance by
  pipelining (starting work on new requests before existing requests have
  completed) but typically bounds the length of the pipeline to ensure
  consistency during reconfiguration. Raft also supports pipelining and imposes
  no such bound on concurrent requests, preserving consistency instead by
  restricting which reconfigurations may be performed. This article shows how
  to extend Paxos to support a more general form of reconfiguration which
  subsumes the original bounded-pipeline approach as well as Raft-like
  fully-concurrent reconfigurations and more besides.

\end{abstract}

% Note that keywords are not normally used for peerreview papers.
\begin{IEEEkeywords}
Distributed algorithms, fault tolerance.
\end{IEEEkeywords}






% For peer review papers, you can put extra information on the cover
% page as needed:
% \ifCLASSOPTIONpeerreview
% \begin{center} \bfseries EDICS Category: 3-BBND \end{center}
% \fi
%
% For peerreview papers, this IEEEtran command inserts a page break and
% creates the second title. It will be ignored for other modes.
\IEEEpeerreviewmaketitle

\section{Introduction}

\IEEEPARstart{R}{eliable} distributed systems must be able to tolerate a fault
in any individual component without suffering a system-wide failure, and
typically achieve this by ensuring that there is redundancy between the
components.  A replicated state machine (RSM) is a style of fault-tolerant
distributed system in which a deterministic state machine is replicated across
a set of distinct nodes \cite{lampson}. Being deterministic, the nodes' states
remain synchronised if they all start in the same state and perform the same
sequence of transitions.

In order to arrange for each node to perform the same transitions the system
may achieve consensus on (or \textit{choose}) a sequence of values which
describe the transitions. The sequence must be consistent across the whole
system even in the presence of failures, and as long as there are not too many
failures it must remain possible to continue to make progress.  This is known
as the distributed consensus problem, for which a number of solutions are known
to exist, including Paxos \cite{part-time-parliament} and Raft\cite{raft}. They
typically run on a cluster of $2f+1$ nodes, where $f$ is the number of faulty
nodes that should be tolerated, and consensus is achieved when a nonempty
\textit{quorum} of nodes (e.g. at least $f+1$ of them) agree.  The collection
of quorums in use is known as the \textit{configuration} of the cluster.

It is normally necessary to be able to dynamically reconfigure a cluster by
adding or removing nodes while it is running, in order that parts of the system
can be repaired or replaced without needing to take the whole system offline.
It is crucial that all participating nodes agree on the cluster configuration,
and this can be achieved by holding the configuration within the RSM itself and
using the consensus algorithm to choose special reconfiguration commands when a
configuration change is desired.

In Paxos, each value is chosen using a conceptually-separate instance of a
two-phase consensus protocol known as \textit{Synod}. The full Paxos algorithm
essentially runs an infinite sequence of Synod instances in parallel, using
uniformity of the instances to do so without requiring infinite time or
resources.  It starts by running phase I of all instances at once and then runs
phase II of each instance in turn to yield the desired sequence of chosen
values. It normally continues to run phase II for extended periods of time, but
will return to phase I if certain nodes become faulty, or if messages between
certain pairs of nodes cease to be delivered reliably for a period, or if the
configuration changes.  Raft's pattern of execution is similar.

Both algorithms can achieve high throughput by allowing for \textit{pipelining}
\cite{smart} whereby work may begin on an instance even before all previous
instances have fully completed. It is a little tricky to ensure that this
preserves consistency when the configuration is held in the RSM itself because
a value may only be proposed once a quorum of nodes are ready for it, but there
may be a configuration change in the pipeline which would change the quorums so
as to make a proposal invalid. If this case is not handled carefully then it
may lead to inconsistency. Paxos implementations typically solve this problem
by limiting the length of the pipeline to some $\alpha > 0$ and requiring that
a configuration change chosen at instance $i$ does not take effect until at
least instance $i + \alpha$, which means that when a value is proposed there
can be no as-yet-undecided configuration change in the pipeline that could
cause a different value to be chosen. In contrast, Raft imposes no limit on the
number of concurrently-running instances and instead restricts the
reconfigurations that may occur to only allow ones that cannot result in
inconsistency. An operator may then perform a sequence of these restricted
reconfigurations in order to achieve an arbitrary reconfiguration.

In Paxos, the choice of the pipeline length parameter $\alpha$ must be made
carefully. If it is too small then the system may suffer from poor performance
due to lack of parallelism, but if it is too large then configuration changes
can be unreasonably expensive to complete. It is more
elegant\cite{reconfiguring-a-state-machine} to limit the pipeline length only
during reconfiguration and to allow the limit to vary while the system is
running, but it would be better still if there were no need for a limit at all,
even during reconfiguration.

Here it is shown that the pipeline may indeed safely remain unbounded even
during a configuration change as long as the reconfiguration satisfies certain
conditions described in section \ref{configuration-changes} below. It is also
shown that if these conditions are not satisfied then a reconfiguration can
still take place as long as the pipeline length is temporarily limited.

The algorithm is presented here in its entirety for the sake of consistency of
notation and because it modifies the original algorithm in ways that invalidate
its consistency and liveness proofs.  We begin with a recap of the Synod
algorithm in section \ref{synod-text} and follow this by covering the full
Paxos algorithm in section \ref{paxos-text}, then in section
\ref{examples} it is shown how this work generalises and
unifies the previously-known reconfiguration processes supported in Paxos and
Raft. Reworked proofs of liveness and consistency are included in section
\ref{liveness} and appendices \ref{synod-safety} and \ref{paxos-invariants} and
differences from the original are highlighted throughout. The appendices are
informal versions of formal proofs performed using the Isabelle/HOL proof
assistant \cite{isabelle-hol}.

%\IEEEpubidadjcol %must be in 2nd col of 1st page

\section{Related Work}

Lamport's original presentation of Paxos \cite{part-time-parliament} introduced
the bounded-pipeline technique for supporting reconfiguration in which the
pipeline length was defined to be 3. He later clarified that the value `3' was
intended to stand for an arbitrary $\alpha > 0$ in \cite{paxos-made-simple}.
Later still Lamport and Massa \cite{cheap-paxos} drew a distinction between
Static Paxos in which the configuration may not change and Dynamic Paxos which
supports reconfiguration but requires a bounded pipeline.  Dynamic Paxos was
used as a basis for Cheap Paxos in which the system automatically reconfigures
itself to achieve higher resilience to failures that do not occur
simultaneously, and which uses a heterogeneous set of nodes to reduce the costs
of operating a cluster.

Malkhi, Lamport and Zhou \cite{stoppable-paxos} proposed Stoppable Paxos, an
alternative method for reconfiguring clusters in which an RSM may be stopped,
reconfigured, and then restarted with the new configuration. Stoppable Paxos
improves on Dynamic Paxos by removing the need for a pipeline limit when the
configuration is static and by allowing a different limit to be selected for
different reconfigurations.

Malkhi, Lamport and Zhou \cite{vertical-paxos} then proposed Vertical Paxos
which keeps the RSM running throughout a reconfiguration, but requires a
separate oracle to manage the configuration. The Egalitarian Paxos of Moraru,
Andersen and Kaminsky \cite{egalitarian-paxos} uses a similar reconfiguration
scheme.  The problem with needing an external oracle is that for full
resilience it must be possible to reconfigure the oracle itself, which requires
another oracle and so on \textit{ad infinitum}.

Chandra, Griesemer and Redstone \cite{paxos-made-live} noted that the details
of reconfiguration are ``relatively minor'' but ``subtle'' and do not give any
details on the reconfiguration scheme used in their Chubby system. It seems
likely that they also used the bounded-pipeline approach. In contrast, Birman,
Malkhi and van Renesse \cite{virt-synch} noted that allowing for $\alpha > 1$
may require an unacceptably complex implementation, leading many real-world
systems to disable pipelining by setting $\alpha = 1$, or even to disable
reconfiguration entirely.

Ongaro and Ousterhout \cite{raft} developed the Raft protocol which supports an
unbounded pipeline throughout the reconfiguration process without inconsistency
by instead limiting the reconfigurations that can be performed. They performed
a formal proof of the correctness of Raft without reconfiguration, and an
informal argument that consistency is preserved when a single node is added or
removed from the cluster. Raft uses simple majorities of the set of nodes as
its quorums.

Viewstamped Replication and Zab are two other well-known consensus protocols.
Viewstamped Replication supports reconfiguration as described by Liskov and
Cowling \cite{viewstamped-replication} in a similar fashion to Vertical Paxos.
Reed and Junqueira \cite{zab} initially presented Zab without reconfiguration
and this feature was subsequently added by Shraer, Reed, Malkhi and Junqueira
\cite{zab-reconf}, using a limited-pipeline approach much as in Dynamic Paxos.

An interesting alternative approach to reconfiguration was proposed by Jahl and
Meling \cite{async-reconfig} which uses eventual consistency rather than
consensus to determine the configuration of the system, and therefore supports
reconfiguration as long as it is still possible to communicate with a quorum of
nodes even if consensus cannot be achieved due to an inability to elect a
distinguished leader. In such a situation consensus-based approaches such as
the one presented here would fail to make any further progress, whereas one
based on eventual consistency could be reconfigured into one in which a leader
may be elected and thus in which further progress can be made.

\def\prep#1{\mathbf{prepare}(#1)}
\def\mprom#1#2#3{\mathbf{promised}_{\ge #1}(#2,#3)}
\def\fprom#1#2#3{\mathbf{promised}_{#1}(#2,#3)}
\def\bprom#1#2#3#4{\mathbf{promised}_{#1}(#2,#3;#4)}
\def\prop#1#2{\mathbf{proposed}_{#1}(#2)}
\def\acc#1#2#3{\mathbf{accepted}_{#1}(#2,#3)}
\def\chosen#1#2{\mathbf{chosen}_{#1}(#2)}
\def\owner#1{\mathrm{owner}(#1)}

\begin{figure*}[t!]
\caption{Invariants preserved by the Synod algorithm\label{synod-invariants-figure}}

\vspace{3mm}

\renewcommand{\theenumi}{S\arabic{enumi}}

\begin{enumerate}

\item \label{synod-quorums} For all $b_1 \succ b_2 \in \mathbb B$ there are
  sets of quorums $Q^\textrm{I}(b_1)$ and $Q^\textrm{II}(b_2) \subseteq
  \mathcal P \mathbb A$ such that if $\prop{}{b_1}$ and $\chosen{}{b_2}$ then
  ${Q^\textrm{I}(b_1) \frown Q^\textrm{II}(b_2)}$.

\item \label{synod-fprom} If $\fprom{}{a}{b}$ then $\neg\acc{}{a}{b'}$ for all
  ${b' \prec b}$.

\item \label{synod-bprom} If $\bprom{}{a}{b}{b'}$ then $b' \prec b$,
  $\acc{}{a}{b'}$, and $b'$ is the greatest such ballot in the sense that
  $\neg\acc{}{a}{b''}$ for all $b''$ having $b' \prec b'' \prec b$.

\item \label{synod-prop} If $\prop{}{b}$ then there is a quorum $q^\textrm{I}
  \in Q^\textrm{I}(b)$ such that for every node $a \in q^\textrm{I}$ either
  $\fprom{}{a}{b}$ or else there exists a $b'$ such that $\bprom{}{a}{b}{b'}$;
  if also ${P \triangleq \{ b' \mid \exists a \in q^\textrm{I}.
  \bprom{}{a}{b}{b'} \} \ne \varnothing}$ then $v(b) =
  v\bigl(\mathrm{max}(P)\bigr)$.

\item \label{synod-acc} If $\acc{}{a}{b}$ then $\prop{}{b}$.

\item \label{synod-chosen} If $\chosen{}{b}$ then there is a quorum
  $q^\textrm{II} \in Q^\textrm{II}(b)$ such that $\acc{}{a}{b}$ for every $a
  \in q^\textrm{II}$.

\end{enumerate}
\end{figure*}

\section{The Synod Algorithm}\label{synod-text}

Synod\cite{part-time-parliament} is an algorithm for achieving consensus on a
single value in a distributed system comprising a set of nodes which can
communicate by sending messages to each other. The system is asynchronous but
not Byzantine, in the sense that messages may be delayed, reordered, duplicated
and dropped but not corrupted, and processes may run arbitrarily slowly or even
stop but may not deviate from their specifications.

Let $\mathbb B$ be a set of \textit{ballot identifiers} with a wellfounded
total order $\prec$. Let $\mathbb A$ be a set of \textit{node identifiers} and
let $\mathbb V$ be the set of values that may be chosen.

The Synod algorithm involves five kinds of message, in two phases, as described
below.  Throughout, $a \in \mathbb A$ and $b, b' \in \mathbb B$.  Phase I
starts with the broadcast of a \textit{prepare} message $\prep{b}$ to which
each node may respond with a \textit{promise} message, either a \textit{free
promise} written $\fprom{}{a}{b}$ or a \textit{forced promise} written
$\bprom{}{a}{b}{b'}$ where $a$ identifies the responding node.  Phase II starts
with the broadcast of a \textit{proposal} message $\prop{}{b}$ to which each
node $a$ may respond with an \textit{acceptance} message $\acc{}{a}{b}$. Once
phase II is complete a \textit{success} message $\chosen{}{b}$ is broadcast.
It is convenient also to use these symbols as predicates indicating whether the
corresponding messages have been sent.

There is a function $v : \mathbb B \to \mathbb V$ assigning a value to each
ballot, discussed in more detail in section \ref{value-function} below.

The system satisfies a set of invariants listed in fig.
\ref{synod-invariants-figure}, from which it follows that consistency is
guaranteed in the sense that if $\chosen{}{b}$ and $\chosen{}{b'}$ then $v(b) =
v(b')$ as shown by theorem \ref{synod-safety-theorem} in appendix
\ref{synod-safety}.

Each node operates as a state machine whose transitions are caused by the
receipts of messages. Each phase is considered to be complete for a particular
ballot when appropriate messages have been received from sufficiently many
nodes, where ``sufficiently many'' is defined in terms of sets of quorums of
nodes in the system's configuration.

In more detail, a node may emit $\prop{}{b}$ when it considers phase I to be
complete at ballot $b$, which is when promise messages for $b$ have been
received from a quorum of nodes, and similarly may emit $\chosen{}{b}$ when
it considers phase II to be complete at $b$, which is when acceptances of $b$
from a quorum of nodes have been received.

The phase-I and phase-II quorums are defined so as to always contain at least
one node in common, but may vary depending on the ballot $b$ and the phase as
discussed in section \ref{per-phase-quorums} below.

\subsection{Implementing the value function}\label{value-function}

In an implementation of a RSM, the values chosen represent the transitions that
the state machines must perform. It is possible that these values may be
expensive to transfer between nodes because they could carry a large quantity
of data.

In the original presentation of the Synod algorithm, the values of ballots are
carried along with their identifiers in the messages $\bprom{}{a}{b}{ b',v(b')
}$, $\prop{}{ b,v(b)}$ and $\chosen{}{ b,v(b) }$. This means that, in a unicast
network of $2f + 1$ nodes, each value is included in $2f$ messages (at least
$f$ proposals and at least $f$ success messages) even in the absence of faults,
which is twice as many as necessary. Furthermore, a simple method for detecting
faults is to insist that each node sends at least one complete message within a
certain period of time, but this method is unsatisfactory if messages can be
unboundedly large.

Observation O4 in \cite{cheap-paxos} notes that these values can be replaced in
some cases by hashes, but this idea can be taken a step further and the values
can be completely elided from the messages that take part in the Synod
protocol, allowing considerably more freedom in the implementation of the
function $v$ without sacrificing consistency.

By allowing the values to be communicated using a separate mechanism from the
consensus messages themselves it is possible to seek optimisations that rely on
the fact that the values may be large but need not move quickly whereas the
consensus messages are small but must be transported with low latency to ensure
the system has good performance.

Although it appears that the function $v$ is fixed, in practice it is allowed
to change as the system runs. Treating it as fixed simplifies the consistency
proof and highlights that its values need not be included in all messages, but
means that the system cannot be shown to satisfy any useful liveness
properties.  To recover liveness, note that if the invariants of fig.
\ref{synod-invariants-figure} are satisfied with a value function $v$ then they
continue to be satisfied if $v$ is replaced by another value function $v'$ that
agrees with $v$ on proposed ballots, i.e. where $v(b) = v'(b)$ if $\prop{}{b}$
but not necessarily otherwise. Since only $\owner{b}$ may propose $b$, if it
has not yet itself proposed a value for $b$ it can deduce that $\neg\prop{}{b}$
and therefore freely modify $v(b)$.

It is also important for liveness that the implementation of $v$ is resilient
to the same failure modes as the rest of the system. Since $\owner{b}$ is, in a
sense, responsible for the value of the ballot $b$, it is possible to think of
$v$ as an insert-only set of pairs ${\{ \langle b, v(b) \rangle \mid \prop{}{b}
\}}$ which is an example of a convergent replicated data type\cite{crdts} and
can therefore be implemented simply and robustly in a distributed system
without needing to rely on a consensus algorithm.  Indeed the original
presentation can be seen as containing such an implementation, where the
inclusion of values in all messages ensures convergence occurs as quickly as
possible, and replicating this set across all ${2f+1}$ nodes ensures $v$ itself
may be resilient to as many as $2f$ failures.  Cheap Paxos\cite{cheap-paxos} is
cheaper partly because it replicates $v$ across just the ${f+1}$ primary
processors, with the $f$ auxiliary processors storing just the hashes of values
to ensure integrity.

It is also worth comparing this approach to that of Vertical
Paxos\cite{vertical-paxos} which takes great care to ensure that the system
state is completely transferred between nodes before they start to participate
fully in the cluster, with certain optimisations in recognition of the fact
that this state transfer could be an expensive and time-consuming operation
involving a very large quantity of data. However if the implementation of $v$
is separated out then the quantity of data that must be transferred as part of
the consensus algorithm becomes small enough that it needs no special treatment
and a consensus-free technique may be used to implement $v$ more efficiently.

\subsection{Per-phase quorums}\label{per-phase-quorums}

\def\I#1#2{{#1}^\textrm{I}_{#2}}
\def\II#1#2{{#1}^\textrm{II}_{#2}}
\def\QI#1{\I{Q}{#1}}
\def\QII#1{\II{Q}{#1}}

The consistency property of the Synod algorithm relies on the fact that the set
of nodes involved in completing phase I must always intersect the set of nodes
involved in completing phase II so that there is at least one node involved in
both phases.  Write $Q_1 \frown Q_2$ iff every $q_1 \in Q_1$ and $q_2 \in Q_2$
have ${q_1 \cap q_2 \ne \varnothing}$, and write $\QI{}(b)$ and $\QII{}(b)$ for
the sets of phase-I and phase-II quorums for ballot $b$ respectively.

In the original presentation of the Synod algorithm any (weighted) majority of
the nodes could be used as a quorum, so that $\QI{}(b_1) = \QII{}(b_2)$ and
hence $\QI{}(b_1) \frown \QII{}(b_2)$ for all $b_1$ and $b_2$ since all
majorities intersect.

Theorem \ref{synod-safety-theorem} shows that consistency can still be
guaranteed even if sometimes $\QI{}(b_1) \ne \QII{}(b_2)$, as long as
$\QI{}(b_1) \frown \QII{}(b_2)$ when $\prop{}{b_1}$, $\chosen{}{b_2}$ and $b_1
\succ b_2$, as described in invariant \ref{synod-quorums}.  This weaker
invariant is the key to allowing more general reconfigurations to take place
safely as described in section \ref{configuration-changes} below.

\section{The Paxos algorithm}\label{paxos-text}

\begin{figure*} \caption{Invariants preserved by the Paxos
  algorithm\label{paxos-invariants-figure}}

\vspace{3mm}

\renewcommand{\theenumi}{P\arabic{enumi}}

\begin{enumerate}

\item\label{paxos-quorums} There are configurations $\langle \QI{0}, \QII{0}
  \rangle, \langle \QI{1}, \QII{1} \rangle, \ldots$ where ${\QII{e} \frown
  \QI{e} \frown \QII{e+1}}$ for each $e$.

\item\label{paxos-mprom} If $\mprom{i}{a}{b}$ then $e(b) \le e(i)$ and
  $\neg\acc{j}{a}{b'}$ for all $j \ge i$ and all $b' \prec b$.

\item\label{paxos-fprom} If $\fprom{i}{a}{b}$ then $e(b) \le e(i)$ and
  $\neg\acc{i}{a}{b'}$ for all $b' \prec b$.

\item\label{paxos-bprom} If $\bprom{i}{a}{b}{b'}$ then $e(b) \le e(i)$, $b'
  \prec b$, $\acc{i}{a}{b'}$, and $b'$ is the greatest such ballot in the sense
  that $\neg \acc{i}{a}{b''}$ for all $b''$ having $b' \prec b'' \prec b$.

\item\label{paxos-prop} If $\prop{i}{b}$ then there is a quorum $q \in
  \QI{e(b)}$ such that for every $a \in q$ one of the following holds:
%
\begin{itemize}
\item $\mprom{j}{a}{b}$ for some $j \le i$, or
\item $\fprom{i}{a}{b}$, or
\item $\bprom{i}{a}{b}{b'}$ for some $b'$.
\end{itemize}
%
Furthermore if $P \triangleq \{ b' \mid \exists a \in q. \bprom{i}{a}{b}{b'} \}
\ne \varnothing$ then $v_i(b) = v_i\bigl(\mathrm{max}(P)\bigr)$.

\item \label{paxos-acc} If $\acc{i}{a}{b}$ then $\prop{i}{b}$.

\item \label{paxos-chosen} If $\chosen{i}{b}$ then $e(i) \le e(b) + 1$, for
  each $j < i$ there is a $b'$ with $\chosen{j}{b'}$, and there is a quorum
  $q \in \QII{e(i)}$ with $\acc{i}{a}{b}$ for every $a \in q$.

\end{enumerate}

\end{figure*}

Conceptually, Paxos is a sequence of distinct instances of the Synod algorithm
all running simultaneously. To achieve this, the messages of the Synod
algorithm above are indexed with the instance number $i \in \mathbb N$:
$\fprom{i}{a}{b}$, $\bprom{i}{a}{b}{b'}$, $\prop{i}{b}$, $\acc{i}{a}{b}$ and
$\chosen{i}{b}$.  There is another kind of message known as a
\textit{multi-promise}, written $\mprom{i}{a}{b}$, which can be thought of as
standing for the infinite set of free promises ${\{ \fprom{j}{a}{b} \mid j \ge
i \}}$. Prepare messages $\prep{b}$ apply to all instances so are not indexed.

As in the Synod algorithm, phase I starts with a broadcast of $\prep{b}$ for
some $b$ to which each node $a$ may respond with a set of promises
$\fprom{i}{a}{b}$, $\bprom{i}{a}{b}{b'}$ and $\mprom{i}{a}{b}$ according to its
past behaviour. Each phase II instance $i$ operates just as in the Synod
algorithm, starting with a broadcast of $\prop{i}{b}$ to which each node $a$
may respond with an acceptance $\acc{i}{a}{b}$ and once acceptances have been
received from a quorum of nodes it follows that $\chosen{i}{b}$ may be
broadcast.

There is a function $v_i : \mathbb B \to \mathbb V$ for each instance $i$
giving a value to each ballot, and theorem \ref{paxos-safety-theorem} shows
that whenever $\chosen{i}{b}$ and $\chosen{i}{b'}$ it follows that $v_i(b) =
v_i(b')$.

The invariants listed in fig. \ref{paxos-invariants-figure} are roughly the
same as for many other presentations of Paxos with the addition of constraints
on the eras of instances and ballots as discussed below. Note that invariants
\ref{paxos-mprom}, \ref{paxos-fprom} and \ref{paxos-bprom} limit the
acceptances that can be sent as well as the promises, so there is no need to
define separate invariants concerning the sending of acceptances.

\subsection{Configuration changes}\label{configuration-changes}

A fixed cluster configuration is a pair $\langle \QI{}, \QII{} \rangle$ of sets
of quorums satisfying $\QI{} \frown \QII{}$, where $\QI{}$ and $\QII{}$ are the
sets of quorums to use in phase I and phase II respectively.

Changes to the cluster configuration are modelled by a sequence of
configurations $ \langle \QI{0}, \QII{0} \rangle, \langle \QI{1}, \QII{1}
\rangle, \ldots$ that also satisfy $\QI{e} \frown \QII{e+1}$ for all $e$.  The
integer subscript is called the \textit{era} of a configuration.  Intuitively
the cluster is ``in era $e$'' while instances are being chosen using $\langle
\QI{e}, \QII{e}\rangle$. While a change from era $e$ to $e+1$ is in progress
some instances may use the interim configuration $\langle \QI{e}, \QII{e+1}
\rangle$, and once the change to era $e+1$ is complete instances will use
$\langle \QI{e+1}, \QII{e+1} \rangle$. This intuition is captured more
precisely in section \ref{fully-concurrent}, and since $\QII{e} \frown \QI{e}
\frown \QII{e+1} \frown \QI{e+1}$ it follows that consistency is preserved
throughout by the observation of \ref{per-phase-quorums} above.

To achieve this there are also two nondecreasing integer-valued functions, both
written $e(\cdot)$, which respectively assign an era $e(i)$ to each instance
$i$, and an era $e(b)$ to each ballot $b$. Intuitively $e(b)$ records which
quorums may be used in phase I to decide that $\prop{i}{b}$ can be sent, and
$e(i)$ records which quorums may be used in phase II to decide that
$\chosen{i}{b}$ can be sent. More precisely a node may emit $\prop{i}{b}$ only
if it has received promises for ballot $b$ in instance $i$ from a quorum of
nodes in $\QI{e(b)}$, which implies that $e(b) \le e(i)$, and similarly a node
may emit $\chosen{i}{b}$ only if ${e(i) \le e(b)+1}$ and it has received
$\acc{i}{a}{b}$ from a quorum of nodes in $\QII{e(i)}$. These extra conditions
on the eras of ballots and instances in messages ensure that if $\chosen{i}{b}$
then ${e(b) \le e(i) \le e(b)+1}$ and hence $\QI{e(b)} \frown \QII{e(i)}$ as
required to ensure consistency, as shown in lemma
\ref{paxos-synod-quorum-invariant}.

\subsection{Dynamic configuration changes}

As in Dynamic Paxos, the configurations $\langle \QI{0}, \QII{0} \rangle,
\langle \QI{1}, \QII{1} \rangle, \ldots$ and era numbers $e(i)$ are themselves
chosen by consensus. In contrast, era numbers $e(b)$ for ballots $b$ are fixed
in advance and not chosen by consensus.

In more detail, configurations are held within the RSM as a finite sequence
$\langle \QI{0}, \QII{0} \rangle, \langle \QI{1}, \QII{1} \rangle, \ldots,
\langle \QI{e_\mathrm{max}}, \QII{e_\mathrm{max}} \rangle$ and the era numbers
of instances are held similarly as a nondecreasing sequence $e(0), e(1),
\ldots, e(i_\mathrm{max})$.  The transitions that affect these sequences may
append one or more elements, but may not change any existing elements.

The sequences are always long enough to make progress, in the sense that
$e_\mathrm{max} \ge e(i_\mathrm{max})$ and if $\chosen{j}{b}$ for all $j < i$
then $i_\mathrm{max} \ge i$.

Note that a node may emit a promise for $b$ at instance $i$ only if $e(b) \le
e(i)$.  If $i > i_\mathrm{max}$ then $e(i)$ is not known, but $e(\cdot)$ is
nondecreasing so if $e(b) \le e(i_\mathrm{max})$ then it follows that $e(b) \le
e(i)$ as required.

\subsection{Liveness}\label{liveness}

To be useful, a consensus algorithm must not only guarantee consistency but
also ensure that it cannot ``get stuck'', i.e. it is always possible to make
progress by eventually choosing a value for each instance. It is known to be
impossible to guarantee liveness in a deterministic asynchronous
system\cite{flp-impossibility} but as with earlier presentations of
Paxos\cite{paxos-made-simple} here liveness can be shown under the assumption
that a distinguished node $\ell$ is eventually selected as the only one that
may emit $\prep{b}$ messages.

The original liveness proof then proceeded by having $\ell$ emit $\prep{b}$ for
some $b$ that is chosen to be large enough that a quorum of nodes may respond
with promises. In contrast, here there is an upper bound on suitable ballots
since a proposed ballot must not belong to an era which is too large, because
if ${e(b) > e(i)}$ then $\prop{i}{b}$ may never be sent and nor may
$\prop{i}{b'}$ for any $b' \succ b$ as $e(\cdot)$ is nondecreasing.  Therefore
here $\ell$ must be able to choose a ballot that is large enough to be accepted
but which still belongs to the correct era, or, more precisely, for each ballot
$b$, each era $e \ge e(b)$ and each node $a$ there must be a ballot $b' \succ
b$ having $e(b') = e$ and $\owner{b'} = a$.

This means that ballot numbers cannot be simple integers because this would
imply that there exist eras containing only finitely many ballots. Instead an
implementation could, for example, let $\mathbb B = \mathbb N \times \mathbb N
\times \mathbb A$ ordered lexicographically, where $e(\langle e, n, a\rangle)
\triangleq e$ and $\owner{\langle e, n, a \rangle} \triangleq a$. This is the
approach used in Egalitarian Paxos\cite{egalitarian-paxos} in which eras are
known as \textit{epochs} but this terminology is avoided here to prevent
confusion with the epochs (views, terms, \ldots) of leader-election protocols
(e.g.  \cite{omega-meets-paxos}) which track the current leader rather than the
current configuration.

With this in mind the proof of liveness runs much as in the original
presentation:

\begin{theorem}[Liveness]\label{liveness-theorem} Given that there is
  eventually a nonfaulty distinguished node $\ell$ which is the only node that
  may emit prepare messages, and sufficiently many other nonfaulty nodes, and
  given that for every instance there is eventually at least one value to
propose, then eventually a value is chosen for every instance.  \end{theorem}

\begin{proof} The proof proceeds by induction over the instances, so suppose
that $\chosen{j}{b_j}$ for all $j < i$ and show that eventually
$\chosen{i}{b_i}$ as follows.

Firstly recall that $i \le i_{\mathrm{max}}$ and $e(i_{\mathrm{max}}) \le
e_{\mathrm{max}}$, so that the values of $e(i)$, $\QI{e(i)-1}$, $\QI{e(i)}$ and
$\QII{e(i)}$ are known to $\ell$.

The distinguished node $\ell$ first chooses a ballot $b_i$ having $e(b_i) \in
\{ e(i) - 1, e(i) \}$ and $\owner{b_i} = \ell$ and such that enough nodes can
emit $\acc{i}{a}{b_i}$ without breaking any of their previously-made promises.
By invariants \ref{paxos-mprom}, \ref{paxos-fprom} and \ref{paxos-bprom}, all
promises for such a ballot $b'$ at instance $i$ must have $e(b') \le e(i)$ so
that such a $b_i$ does exist.

If $\ell$ has not yet received enough promises for $b_i$ at instance $i$ then
it broadcasts $\prep{b_i}$ and waits to receive promises from a quorum of nodes
in $\QI{e(b_i)}$.

Then, if $\ell$ has not yet emitted $\prop{i}{b_i}$ it selects one of the
values for instance $i$ (which eventually exists), sets $v_i(b_i)$ as
appropriate, broadcasts $\prop{i}{b_i}$, and waits to receive acceptances in
response. When acceptances have been received from a quorum of nodes in
$\QII{e(i)}$ it follows that $\chosen{i}{b_i}$ as required. \end{proof}

\subsection{Fully concurrent configuration changes}\label{fully-concurrent}

The discussion so far shows that, like other Paxos variants, this algorithm
satisfies consistency and liveness properties.  The benefit of this scheme
compared with other variants is that, under normal running conditions, it is
possible to perform a reconfiguration without needing to impose a limit on the
number of concurrently-running instances. This section describes the details of
this procedure.

In normal running there is an instance $i_0$ and a ballot $b$ with ${e(b) =
e(i_0) = e(i_\mathrm{max}) = e_\mathrm{max}}$ and the distinguished node $\ell
= \owner{b}$ has received a quorum of promises $q^\textrm{I} \in \QI{e(b)}$ for
ballot $b$ for all instances $i \ge i_0$. In this state, $\ell$ may emit
$\prop{i}{b}$ for any $i \ge i_0$, since $e(b) = e(i_0) \le e(i)$ as $e$ is
nondecreasing.  The node $\ell$ is known as the \textit{leader} and its
proposals are normally accepted without undue delay by all other nodes.

\def\Qnew#1{Q^\textrm{#1}_\textrm{new}}

Suppose that, in normal running, an operator wishes to change the cluster
configuration to $\langle \Qnew{I}, \Qnew{II} \rangle$ where $\QI{e(b)} \frown
\Qnew{II} \frown \Qnew{I}$. First she appends $\langle \Qnew{I}, \Qnew{II}
\rangle$ to the sequence of configurations, setting $\langle \QI{e(b)+1},
\QII{e(b)+1} \rangle = \langle \Qnew{I}, \Qnew{II} \rangle$ and $e_\mathrm{max}
= e(b) + 1$, then she picks a future instance $i_c > i_\mathrm{max}$ at which
the change should take effect and appends values to the sequence of eras to set
$e(i) = e(i_0)$ for $i_0 \le i < i_c$ and $e(i_c) = e(i_0) + 1$.  Since $e(i_c)
= e(i_0) + 1 \le e(b) + 1$, values for instance $i_c$ and any future instances
with the same era may be proposed and chosen even though phase I has not yet
run for a ballot in this era, so this does not prevent any further instances
from running concurrently.

At this point, the system is no longer in normal running as defined above
because $e(b) = e(i_\mathrm{max}) - 1$. If the operator were to increase
$e(i_\mathrm{max})$ any further then there would be an instance $i$ with $e(i)
> e(b) + 1$ and hence $\neg\chosen{i}{b}$. A value can still eventually be
chosen for instance $i$ due to theorem \ref{liveness-theorem}, but not before
the leader has selected a new ballot $b'$ in an appropriate era, completed
phase I for $b'$, and then broadcast new proposals for $b'$. These steps may
cause the pipeline to stall if not completed quickly enough.

The system must therefore be returned to normal running before any further
reconfiguration can occur. To do this, the leader chooses a new ballot $b'$
having $e(b') = e(b) + 1$ and $\owner{b'} = \ell$ and runs phase I for $b'$ in
a way that does not prevent any progress in era $e(b)$ while it has not
completed. This is possible if $\ell$ has a \textit{casting vote} in the sense
that there are quorums of nonfailed nodes $q \in \QII{e(b)}$ and $q' \in
\QI{e(b)+1}$ having $q \cap q' = \{\ell\}$. With a casting vote, $\ell$ may
broadcast $\prep{b'}$ just to the nodes in $q' \setminus \{\ell\}$ without
preventing further progress in era $e(b)$ since the nodes in $q$ can continue
to accept proposals in this era throughout.  When it has received promises from
all the other nodes in $q'$ it can instantaneously send itself
$\mprom{i'}{\ell}{b'}$ for some sufficiently large $i'$, which completes phase
I at $b'$ and restores the system to normal running in era $e(b') = e(b)+1$.

If $\ell$ does not have a casting vote, but there is some other node $\ell'$
which does, then $\ell$ should first abdicate its leadership to $\ell'$ and
then the new leader should perform the new phase I as described above. If there
is no node with a casting vote at all then the broadcast of $\prep{b'}$ may
prevent progress until phase I is complete at $b'$.

If any node fails during this process then it may be necessary to retry some of
the steps or possibly even elect a new leader. Liveness and consistency
continue to hold if nodes fail but performance may be affected, for instance by
meaning that $\ell$ no longer has a casting vote. In general it is not possible
to prevent node failures from having a performance impact.

\begin{figure}[!t]
\centering
\includegraphics{smooth}
\caption{Message flow during a reconfiguration. \label{seq-diag-smooth}}
\end{figure}

To give a concrete example of this process, fig. \ref{seq-diag-smooth} shows
the flow of messages during a reconfiguration involving the nodes $a_1$ and
$a_2$ and the leader $\ell$. The system starts in era $e$ where $\{\ell, a_1\}
\in \QI{e}$ and $\{\ell, a_2\} \in \QII{e}$, and moves to era $e+1$ where
$\{\ell, a_1\} \in \QI{e+1}$ and $\{\ell, a_2\} \in \QII{e+1}$ too. Because of
these quorums, the leader only needs a response from node $a_1$ to complete
phase I, and similarly only needs a response from node $a_2$ to complete phase
II and choose a value. Notice that this means the leader has a casting vote.
The messages sent from $\ell$ to the other nodes are labelled on the diagram,
but the successful responses (promises and acceptances from $a_1$ and $a_2$
respectively) are left unlabelled for clarity. Initially, $i_\mathrm{max} =
i+3$ and $e(i) = e(i+1) = e(i+2) = e(i+3) = e = e_\mathrm{max}$.

The leader starts by completing phase I at a sufficiently large ballot $b$,
where $e(b) = e$, which starts a period of normal running in era $e$ and means
that $\ell$ may emit $\prop{j}{b}$ for any $j \ge i$. It receives three client
requests, causing it to propose values for instances $i$, $i+1$ and $i+2$ in
turn.  On the receipt of each proposal the node $a_2$ responds with an
acceptance, which when ultimately received by $\ell$ allows it to decide that
each value is chosen because $\{ \ell, a_2 \} \in \QII{e}$.

A configuration change is proposed at instance $i+3$ which, when chosen,
appends the next configuration $\langle \QI{e+1}, \QII{e+1} \rangle$ to the
configuration sequence, sets $e(i+4) = e(i+5) = \ldots = e+1$, and increases
$e_\mathrm{max}$ and $i_\mathrm{max}$ accordingly. This takes the system out of
normal running because now $e(i_\mathrm{max}) = e + 1 \ne e(b)$.  In order to
bring the system back into normal running, the leader must choose a ballot $b'$
having $e(b') = e_\mathrm{max}$ and complete a phase I at $b'$, so it sends
$\prep{b'}$ to $a_1$.

While instance $i+3$ was being chosen, the leader continued to service client
requests by sending out $\prop{i+4}{b}$, $\prop{i+5}{b}$ and $\prop{i+6}{b}$,
which $a_2$ accepts in due course. Although these instances come after the
configuration change at instance $i+3$, when the leader receives
$\acc{i+4}{a_2}{b}$ it may still safely deduce $\chosen{i+4}{b}$ since $e(i+4)
= e + 1 \le e(b) + 1$ and $\{ \ell, a_2 \} \in \QII{e+1}$, and similarly for
instances $i+5$ and $i+6$. It is important to notice that the leader is now
using $\QII{e+1}$ and not $\QII{e}$ to determine when ballots are chosen. The
horizontal dashed line shows the point in time at which the leader moves from
era $e$ to era $e+1$.

While the leader is waiting for the response from $a_1$, client requests
continue to arrive, yielding proposals for subsequent instances $i+7$, $i+8$,
\ldots.  Notice that the leader is still using ballot $b$ for these requests as
it has not yet completed phase I at $b'$, but that it is still safe to deduce
$\chosen{i+7}{b}$, $\chosen{i+8}{b}$, $\ldots$ because $e(i+7) = e(i+8) =
e(b)+1$.  Importantly, there is no limit to how many requests the leader can
handle in this way, so the system will continue to be able to process client
requests even if the phase I messages are arbitrarily delayed.

At last the response from $a_1$ is received just after the sending of
$\prop{i+11}{b}$ and before a proposal has been made for instance $i+12$. The
leader can then instantaneously send itself the message
$\mprom{i+12}{\ell}{b'}$ which completes phase I at ballot $b'$ since $\{\ell,
a_1\} \in \QI{e+1}$.  This restores the system to normal running, and allows
$\ell$ to make proposals $\prop{j}{b'}$ for all $j \ge i+12$.

\begin{figure}[!t]
\centering
\includegraphics{dynamic}
\caption{Message flow for Dynamic Paxos with $\alpha = 2$. \label{seq-diag-dynamic}}
\end{figure}

Fig. \ref{seq-diag-dynamic} shows an equivalent reconfiguration performed in a
Dynamic Paxos cluster with the pipeline length $\alpha = 2$. As there are
frequently two proposals being processed concurrently it seems likely that this
value was selected poorly and that better performance could be achieved by
selecting a higher value. A higher value still could have avoided the pause
between instances $i+5$ and $i+6$ caused by the unexpected delay in completing
phase I at ballot $b'$. On the other hand if the pipeline is too long then
configuration changes can be expensive to complete. It is, in general,
difficult to select an appropriate value for $\alpha$ up-front, and no matter
what value is selected it is possible that a configuration change may cause a
pause if a phase-I message is unexpectedly delayed or a sudden burst of client
requests are received.

\begin{figure}[!t]
\centering
\includegraphics{stoppable2}
\caption{Message flow for Stoppable Paxos. \label{seq-diag-stoppable-ooo}}
\end{figure}

In contrast, fig. \ref{seq-diag-stoppable-ooo} shows an equivalent
reconfiguration performed in Stoppable Paxos. This variant of Paxos allows for
an unlimited number of proposals to run in parallel within each configuration,
and permits out-of-order execution, so so in this illustration the stopping
command is proposed at instance $i+3$ before proposals are made for the two
preceding instances.  By selecting instance $i+3$ for the reconfiguration, the
operator is limiting the system to service at most 2 more client requests
before the reconfiguration completes. As in the illustration of Dynamic Paxos
above, the operator's choice is too conservative so the remaining two instances
are exhausted before phase I is completed at ballot $b'$, which causes the
system to temporarily suspend its processing of client requests.

\section{Examples}\label{examples}

This section contains some examples of configuration changes that satisfy the
conditions described above.  All the configurations described here have equal
sets of quorums in phase I and phase II of each era, so for the sake of
simplicity throughout this section define ${Q_e \triangleq Q^\textrm{I}_e =
Q^\textrm{II}_e}$.  Implementations can ensure $Q_e \frown Q_e$ for each $e$
by, for instance, arranging for each quorum in $Q_e$ to comprise a majority
subset of some finite set of nodes. Slightly more generally, let a
\textit{weight function} be a function $w : \mathbb A \to \mathbb N$ that only
takes finitely many nonzero values. This can be used to define a configuration
$\langle M(w), M(w) \rangle$ by \textit{weighted majority}: \[M(w) \triangleq
  \left\{ q \;\middle|\; \sum_{a \in q} 2 w(a) > \sum_{a \in \mathbb A} w(a)
\right\}.\] Corollary \ref{weights-equal} in appendix \ref{weights-appendix}
demonstrates the well-known result that $M(w) \frown M(w)$ for any weight
function $w$.  Indeed, if $w$ and $w'$ are weight functions that differ by a
constant factor in the sense that that there are positive integers $k$ and $k'$
with $k w(a) = k' w'(a)$ for all $a$, then clearly ${M(w) = M(w')}$ and hence
$M(w) \frown M(w')$.

\def\wl#1{w^{1 \ldots #1}}
\def\Mwl#1{M(\wl{#1})}

Raft's quorums are simple unweighted majorities of a finite set of nodes, which
can be emulated with weight functions that only take values in $\{0, 1\}$. It
only supports adding or removing a single node from this set which amounts to
changing the weight of a single node by $\pm 1$. For instance, if $\mathbb A =
\{ a_1, a_2, \ldots \}$ then define weight functions \[\wl{n}(a) \triangleq
\begin{cases} 1 & \mathrm{if } a \in \{a_1, \ldots, a_n\} \\ 0 &
\mathrm{otherwise}\end{cases}\] and observe that ${\Mwl{3} \frown \Mwl{4}}$ and
${\Mwl{4} \frown \Mwl{5}}$ but ${\Mwl{3} \not\frown \Mwl{5}}$ because $\{a_1,
a_2\} \in \Mwl{3}$ and $\{a_3, a_4, a_5\} \in \Mwl{5}$ do not intersect. This
justifies the restriction against adding or removing more than one node at
once.

In fact there is no need to restrict attention just to weight functions taking
values in $\{0, 1\}$ as shown by theorem \ref{weights-nearly-equal} which is
reminiscent of the amoeba analogy in \cite{cheap-paxos}: any two integer-valued
weight functions whose total absolute difference is at most one can be used to
define consecutive configurations. This extra generality is important for
deployments where nodes may share infrastructure (e.g. power distribution or
network connectivity) because such nodes may suffer correlated failures, and
reducing this correlation by adding more independent infrastructure may be
costly.  In more detail, a na\"ive approach to swapping a node
$a_{\textrm{old}}$ for a replacement $a_{\textrm{new}}$ in a three-node cluster
using unweighted majorities would be to perform the configuration changes given
by this sequence of weight functions starting at era $e$: \[\begin{array}{rcccc}
%
\textrm{node}&a_{\textrm{old}}&a_{\textrm{new}}&a_1&a_2 \\
%
w_e&1&0&1&1\\
%
w_{e+1}&1&1&1&1\\
%
w_{e+2}&0&1&1&1\\
%
\end{array}\]
%
However to be resilient to infrastructure failures this requires all four nodes
and their underlying infrastructures to be completely independent since a
correlated failure of any two nodes would prevent further progress.  It also
has no node with a casting vote in era $e+1$. On the other hand the following
sequence achieves the same overall change but allows $a_{\textrm{old}}$ and
$a_{\textrm{new}}$ to share infrastructure without extra risk, and both $a_1$
and $a_2$ have casting votes throughout: \[\begin{array}{rcccc}
%
\textrm{node}&a_{\textrm{old}}&a_{\textrm{new}}&a_1&a_2 \\
%
w_e&1&0&1&1\\
%
w_{e+1}&2&0&2&2\\
%
w_{e+2}&2&1&2&2\\
%
w_{e+3}&1&1&2&2\\
%
w_{e+4}&0&1&2&2\\
%
w_{e+5}&0&2&2&2\\
%
w_{e+6}&0&1&1&1\\
%
\end{array}\]
%
This is important as in many operating environments it may be too expensive or
complicated to arrange for four independent infrastructures particularly if the
fourth is only required to ensure consistency in relatively rare periods of
maintenance. For instance at time of writing only one Amazon Web Services
region (\texttt{us-east-1}) has four independent zones, whereas four of them
have three: \texttt{ap-southeast-2}, \texttt{eu-west-1}, \texttt{sa-east-1} and
\texttt{us-west-2}. Similarly, only one Google Cloud Platform region
(\texttt{us-central1}) has four zones and all the others have three zones.

Early versions of Raft supported more general reconfigurations using a
technique known as \textit{joint configurations}. To change from configuration
$Q_{e}$ to an unrelated configuration $Q'$ (i.e.  $Q_{e} \not\frown Q' \frown
Q'$) it is possible to set $Q_{e+2} = Q'$ and set $Q_{e+1}$ to be the
\textit{joint configuration} of $Q_{e}$ and $Q'$: \[Q_{e+1} = \{ q \cup q' \mid
q \in Q_{e}, q' \in Q' \}\] as this satisfies that $Q_{e} \frown Q_{e + 1}
\frown Q_{e+1} \frown Q_{e + 2} \frown Q_{e + 2}$.

If $w(a) = 0$ for all $a$ then $w$ is said to be \textit{weightless} and $M(w)
= \varnothing$. Clearly if there is an instance $i$ such that $Q_{e(i)} =
\varnothing$ then no value can ever be chosen for $i$ or any subsequent
instance, by invariant \ref{paxos-chosen}. On the other hand $Q \frown
\varnothing$ for all $Q$ so changing to a weightless configuration is always
permitted, and this has the effect of stopping a Paxos cluster at a particular
instance \cite{reconfiguring-a-state-machine,stoppable-paxos}.

There is no requirement for the eras of consecutive instances to differ by at
most one so an era may be skipped at instance $i$ by letting $e(i) = e(i-1) +
2$.  This recovers the ability to perform arbitrary configuration changes in a
single step as in Stoppable Paxos. In more detail, if the system is currently
using configuration $Q_{e}$ and an operator wishes to change to an unrelated
configuration $Q'$ then she can set $Q_{e+2} = Q'$ and pick an appropriate
$Q_{e+1}$ (such as $\varnothing$) which satisfies that $Q_{e} \frown Q_{e + 1}
\frown Q_{e+1} \frown Q'$.  However, re-running phase I can only be delayed as
described in section \ref{fully-concurrent} above if the era increases by $1$
and in this situation the era increases by $2$, so a new phase I must be
completed before phase II of any new instances can be started.  To prevent this
causing the pipeline to stall, the operator chooses a sufficiently large
$\alpha > 0$ and sets $e(i+\alpha) = e(i)+2$ and $e(j) = e(i)$ for $i < j \le i
+ \alpha$, effectively delaying the configuration change for $\alpha$ instances
in the hope that this is long enough to have completed the new phase I.

\section{Conclusion}

The approach described here generalises Dynamic Paxos\cite{cheap-paxos} to
support changing the pipeline length parameter $\alpha$ and running with an
unlimited-length pipeline while a configuration change is not in progress.

In that sense, it can be compared to that of Stoppable
Paxos\cite{stoppable-paxos} which allows for an unlimited number of proposals
to run in parallel within each configuration, but requires a temporary
arbitrary limit on concurrency while a reconfiguration takes place.

In both Dynamic and Stoppable Paxos, if the selected limit is either too small
or too large then it may affect the system's performance. The approach
described here avoids the need to select any such limit and responds to
changing system conditions without needing further tuning. Once a
reconfiguration is chosen, it completes after a single round-trip to a quorum
of nodes and it continues to serve clients while this is in progress, no matter
how long it takes.

This is achieved by using Raft-style reconfigurations\cite{raft} which can be
performed with an unlimited pipeline throughout. Unlike in Raft, here a
configuration change only takes effect once it is chosen, which avoids the need
to back-track to an earlier state if a leader fails during reconfiguration. It
generalises the simple majorities used in Raft to integer-weighted majorities
which can reduce the costs of dealing with correlated failures during
maintenance.

It achieves equivalent goals to those of Vertical Paxos\cite{vertical-paxos}
except that here there is no requirement for a separate oracle to manage the
configuration of the system.

It is also noted that there is no need for every message to include the
corresponding value, or even a hash of the value, which may allow for even
cheaper implementations of Cheap Paxos\cite{cheap-paxos} and can simplify the
transfer of state\cite{vertical-paxos} required when new nodes are
commissioned.

In the author's opinion, it was no harder to correctly implement a
reconfigurable replicated state machine based on this approach than using other
reconfigurable variants of Paxos. The invariants of fig.
\ref{paxos-invariants-figure} are only slightly more complicated than those of
Static Paxos and are straightforward to maintain.

% An example of a floating figure using the graphicx package.
% Note that \label must occur AFTER (or within) \caption.
% For figures, \caption should occur after the \includegraphics.
% Note that IEEEtran v1.7 and later has special internal code that
% is designed to preserve the operation of \label within \caption
% even when the captionsoff option is in effect. However, because
% of issues like this, it may be the safest practice to put all your
% \label just after \caption rather than within \caption{}.
%
% Reminder: the "draftcls" or "draftclsnofoot", not "draft", class
% option should be used if it is desired that the figures are to be
% displayed while in draft mode.
%


% Note that the IEEE typically puts floats only at the top, even when this
% results in a large percentage of a column being occupied by floats.


% An example of a double column floating figure using two subfigures.
% (The subfig.sty package must be loaded for this to work.)
% The subfigure \label commands are set within each subfloat command,
% and the \label for the overall figure must come after \caption.
% \hfil is used as a separator to get equal spacing.
% Watch out that the combined width of all the subfigures on a 
% line do not exceed the text width or a line break will occur.
%
%\begin{figure*}[!t]
%\centering
%\subfloat[Case I]{\includegraphics[width=2.5in]{box}%
%\label{fig_first_case}}
%\hfil
%\subfloat[Case II]{\includegraphics[width=2.5in]{box}%
%\label{fig_second_case}}
%\caption{Simulation results for the network.}
%\label{fig_sim}
%\end{figure*}
%
% Note that often IEEE papers with subfigures do not employ subfigure
% captions (using the optional argument to \subfloat[]), but instead will
% reference/describe all of them (a), (b), etc., within the main caption.
% Be aware that for subfig.sty to generate the (a), (b), etc., subfigure
% labels, the optional argument to \subfloat must be present. If a
% subcaption is not desired, just leave its contents blank,
% e.g., \subfloat[].


% An example of a floating table. Note that, for IEEE style tables, the
% \caption command should come BEFORE the table and, given that table
% captions serve much like titles, are usually capitalized except for words
% such as a, an, and, as, at, but, by, for, in, nor, of, on, or, the, to
% and up, which are usually not capitalized unless they are the first or
% last word of the caption. Table text will default to \footnotesize as
% the IEEE normally uses this smaller font for tables.
% The \label must come after \caption as always.
%
%\begin{table}[!t]
%% increase table row spacing, adjust to taste
%\renewcommand{\arraystretch}{1.3}
% if using array.sty, it might be a good idea to tweak the value of
% \extrarowheight as needed to properly center the text within the cells
%\caption{An Example of a Table}
%\label{table_example}
%\centering
%% Some packages, such as MDW tools, offer better commands for making tables
%% than the plain LaTeX2e tabular which is used here.
%\begin{tabular}{|c||c|}
%\hline
%One & Two\\
%\hline
%Three & Four\\
%\hline
%\end{tabular}
%\end{table}


% Note that the IEEE does not put floats in the very first column
% - or typically anywhere on the first page for that matter. Also,
% in-text middle ("here") positioning is typically not used, but it
% is allowed and encouraged for Computer Society conferences (but
% not Computer Society journals). Most IEEE journals/conferences use
% top floats exclusively. 
% Note that, LaTeX2e, unlike IEEE journals/conferences, places
% footnotes above bottom floats. This can be corrected via the
% \fnbelowfloat command of the stfloats package.

\appendices

\section{}\label{weights-appendix}

\begin{theorem} \label{weights-nearly-equal} If $w, w' : \mathbb A \to \mathbb
N$ are weight functions such that $\sum_{a \in \mathbb A} |w'(a) - w(a)| \le 1$
then $M(w) \frown M(w')$.  \end{theorem}

\begin{proof}Since $w$ and $w'$ take integer values they must take equal values
  except possibly at a single node, so let $a_0$ be a node such that $w(a) =
  w'(a)$ for all $a \ne a_0$.
%
Let $q \in M(w)$ and $q' \in M(w')$. By the definition of $M$, and since $w$
and $w'$ take only integer values,
%
$\sum_{a \in q} 2 w(a) \ge \sum_{a \in \mathbb A} w(a) + 1$
%
and
%
$\sum_{a \in q'} 2 w'(a) \ge \sum_{a \in \mathbb A} w'(a) + 1$.
%
Let $d_{\mathbb A} \triangleq w'(a_0) - w(a_0)$ so that $\sum_{a \in \mathbb A}
w'(a) = \sum_{a \in \mathbb A} w(a) + d_{\mathbb A}$ and $|d_\mathbb A| \le 1$.
Also let \[ d_{q'} \triangleq \begin{cases}
%
d_{\mathbb A} & a_0 \in q' \\
%
0 & \textrm{otherwise,}
%
\end{cases} \] so that $\sum_{a \in q'} w'(a) = \sum_{a \in q'} w(a) + d_{q'}$.
%
Then
%
\[\begin{split}
%
\sum_{a \in \mathbb A} 2w(a) &+ d_{\mathbb A} + 2 \\
%
&= \left( \sum_{a \in \mathbb A} w(a)  + 1\right) +  \left( \sum_{a \in \mathbb
A} w'(a) + 1\right) \\
%
&\le \sum_{a \in q}  2w(a) +    \sum_{a \in q'} 2w'(a) \\
%
&= \sum_{a \in q}  2w(a) +  \sum_{a \in q'} 2w(a) + 2d_{q'}\\
%
&= \sum_{a \in q \cup q'} 2w(a) +  \sum_{a \in q \cap q'} 2w(a) + 2d_{q'}\\
%
&\le \sum_{a \in \mathbb A} 2w(a) +    \sum_{a \in q \cap q'} 2w(a) + 2d_{q'}\\
%
\end{split}\] so that $\sum_{a \in q \cap q'} 2w(a) \ge d_{\mathbb A} + 2 -
2d_{q'} = 2 \pm d_\mathbb A \ge 1$ and hence $q \cap q' \ne \varnothing$ as
desired.  \end{proof}

\begin{corollary} \label{weights-equal} If $w : \mathbb A \to \mathbb N$ is a
weight function then \[M(w) \frown M(w).\]  \end{corollary}

\begin{proof} This is a special case of theorem \ref{weights-nearly-equal},
where $w' = w$.  \end{proof}


\section{Consistency of the Synod algorithm}
\label{synod-safety}

\begin{lemma}\label{synod-acc-bprom}If $\acc{}{a}{b_2}$,
$\bprom{}{a}{b_1}{b_3}$ and $b_2 \prec b_1$ then $b_2 \preceq b_3$.\end{lemma}

\begin{proof} From invariant \ref{synod-bprom} it follows that $b_3$ is the
largest ballot such that $b_3 \prec b_1$ and $\acc{}{a}{b_3}$, but $b_2$ is
also such a ballot and therefore $b_2 \preceq b_3$ as required.  \end{proof}

\begin{lemma}\label{synod-lemma} If $\chosen{}{b_2}$, $\prop{}{b_1}$ and $b_2
\prec b_1$ then $v(b_1) = v(b_2)$. \end{lemma}

\begin{proof}Suppose for a contradiction that $v(b_1) \ne v(b_2)$ and since
$\prec$ is wellfounded suppose without loss of generality that $b_1$ is the
minimal such ballot.  Since $\chosen{}{b_2}$ by invariant \ref{synod-chosen}
there is a quorum $q^\textrm{II} \in Q^\textrm{II}(b_2)$ such that
$\acc{}{a}{b_2}$ for every $a \in q^\textrm{II}$.  By invariant
\ref{synod-fprom} it cannot be that $\fprom{}{a}{b_1}$ for any $a \in
q^\textrm{II}$.  Also since $\prop{}{b_1}$ by invariant \ref{synod-prop} there
is a quorum $q^\textrm{I} \in Q^\textrm{I}(b_1)$ such that either
$\fprom{}{a}{b_1}$ or $\exists b'.  \bprom{}{a}{b_1}{b'}$ for all $a \in
q^\textrm{I}$.  Let $P \triangleq \{ b' \mid \exists a \in q^\textrm{I}.
\bprom{}{a}{b_1}{b'} \}$.  By invariant \ref{synod-quorums},
${Q^\textrm{I}(b_1) \frown Q^\textrm{II}(b_2)}$ and hence $q^\textrm{I} \cap
q^\textrm{II} \ne \varnothing$ so it follows that $P \ne \varnothing$, which
means that $v(b_1) = v(\mathrm{max}(P))$ by invariant \ref{synod-prop}. Let
$a_{\mathrm{max}} \in q^\textrm{I}$ be such that
$\bprom{}{a_{\mathrm{max}}}{b_1}{\mathrm{max}(P)}$.  By invariant
\ref{synod-bprom} it follows that $\mathrm{max}(P) \prec b_1$ and also that
$\acc{}{a_{\mathrm{max}}}{\mathrm{max}(P)}$ and hence
$\prop{}{\mathrm{max}(P)}$ by invariant \ref{synod-acc}. Furthermore by lemma
\ref{synod-acc-bprom} it follows that $b_2 \preceq \mathrm{max}(P)$ and since
$b_1$ was assumed to be the smallest counterexample it must be that
$v(\mathrm{max}(P)) = v(b_2)$.  Hence $v(b_1) = v(b_2)$ which is a
contradiction as required.  \end{proof}

\begin{theorem}\label{synod-safety-theorem} If $\chosen{}{b_1}$ and
$\chosen{}{b_2}$ then $v(b_1) = v(b_2)$.  \end{theorem}

\begin{proof} Without loss of generality assume that ${b_2 \prec b_1}$. By
invariant \ref{synod-chosen} there is a quorum $q \in Q^\textrm{II}(b_1)$ such
that $\acc{}{a}{b_1}$ for every node $a \in q$ and therefore $\prop{}{b_1}$ by
invariant \ref{synod-acc}.  Therefore by lemma \ref{synod-lemma} it follows
that $v(b_1) = v(b_2)$ as required.  \end{proof}

\section{Consistency of the Paxos algorithm}\label{paxos-invariants}

\begin{lemma}\label{paxos-synod-quorum-invariant} For $b_1 \succ b_2 \in
\mathbb B$, if $\prop{i}{b_1}$ and $\chosen{i}{b_2}$ then
${\QI{e(b_1)} \frown \QII{e(i)}}$.  \end{lemma}

\begin{proof} $\prop{i}{b_1}$ implies that $\mprom{i'}{a}{b_1}$ or
$\fprom{i'}{a}{b_1}$ or $\bprom{i'}{a}{b_1}{b'}$ for some node $a$ and some $i'
\le i$. Therefore $e(i) \le e(b_2) + 1 \le e(b_1) + 1 \le e(i') + 1 \le e(i) +
1$ since $\chosen{i}{b_2}$ and $e$ is nondecreasing so that $e(i) \in \{
e(b_1), e(b_1) + 1 \}$ and hence ${\QI{e(b_1)} \frown \QII{e(i)}}$ by invariant
\ref{paxos-quorums}.  \end{proof}

\begin{theorem}\label{paxos-safety-theorem} If $\chosen{i}{b_1}$ and
$\chosen{i}{b_2}$ then ${v_i(b_1) = v_i(b_2)}$.  \end{theorem}

\begin{proof} If $\chosen{i}{b_1}$ then the Paxos invariants imply the Synod
invariants for instance $i$.  In more detail, let
\[\begin{array}{rl}
Q^\textrm{I}(b) &\triangleq \QI{e(b)} \\
Q^\textrm{II}(b) &\triangleq \QII{e(i)} \\
\fprom{}{a}{b} &\triangleq \fprom{i}{a}{b} \\
& \ {}\vee \exists i' \le i. \mprom{i'}{a}{b} \\
\bprom{}{a}{b}{b'} &\triangleq \bprom{i}{a}{b}{b'} \\
\prop{}{b} &\triangleq \prop{i}{b} \\
\acc{}{a}{b} &\triangleq \acc{i}{a}{b} \\
\chosen{}{b} &\triangleq \chosen{i}{b} \textrm{ and} \\
v(b) &\triangleq v_i(b) .\\
\end{array}
\]
Synod's invariant \ref{synod-quorums} follows from lemma
\ref{paxos-synod-quorum-invariant} and the remaining invariants are simple to
show so by theorem \ref{synod-safety-theorem} it follows that $v_i(b_1) =
v_i(b_2)$ as required.  \end{proof}

% use section* for acknowledgment
\section*{Acknowledgment}

The author would like to thank Leslie Lamport, Dahlia Malkhi and Leander
Nikolaus Jehl for their encouragement and comments on earlier drafts of this
paper. The author is also very grateful to Tracsis plc for supporting this
work.



% Can use something like this to put references on a page
% by themselves when using endfloat and the captionsoff option.
\ifCLASSOPTIONcaptionsoff
  \newpage
\fi



% trigger a \newpage just before the given reference
% number - used to balance the columns on the last page
% adjust value as needed - may need to be readjusted if
% the document is modified later
%\IEEEtriggeratref{8}
% The "triggered" command can be changed if desired:
%\IEEEtriggercmd{\enlargethispage{-5in}}

% references section

% can use a bibliography generated by BibTeX as a .bbl file
% BibTeX documentation can be easily obtained at:
% http://mirror.ctan.org/biblio/bibtex/contrib/doc/
% The IEEEtran BibTeX style support page is at:
% http://www.michaelshell.org/tex/ieeetran/bibtex/
\bibliographystyle{IEEEtran}
% argument is your BibTeX string definitions and bibliography database(s)
\bibliography{paxos}
%
% <OR> manually copy in the resultant .bbl file
% set second argument of \begin to the number of references
% (used to reserve space for the reference number labels box)

% biography section
% 
% If you have an EPS/PDF photo (graphicx package needed) extra braces are
% needed around the contents of the optional argument to biography to prevent
% the LaTeX parser from getting confused when it sees the complicated
% \includegraphics command within an optional argument. (You could create
% your own custom macro containing the \includegraphics command to make things
% simpler here.)
%\begin{IEEEbiography}[{\includegraphics[width=1in,height=1.25in,clip,keepaspectratio]{mshell}}]{Michael Shell}
% or if you just want to reserve a space for a photo:

%\begin{IEEEbiography}{Michael Shell} Biography text here.  \end{IEEEbiography}
%
%% if you will not have a photo at all:
\ifCLASSOPTIONpeerreview
\else
\begin{IEEEbiographynophoto}{David C. Turner} received the MMath degree and the
Ph.D. degree from the University of Cambridge in 2011 and 2009 respectively.

Since 2010 he has worked as a software engineer at Tracsis plc building systems
to support and optimise operational planning processes in the transport
industry.

\end{IEEEbiographynophoto}
\fi

%
%% insert where needed to balance the two columns on the last page with
%% biographies
%%\newpage
%
%\begin{IEEEbiographynophoto}{Jane Doe}
%Biography text here.
%\end{IEEEbiographynophoto}

% You can push biographies down or up by placing
% a \vfill before or after them. The appropriate
% use of \vfill depends on what kind of text is
% on the last page and whether or not the columns
% are being equalized.

\vfill

% Can be used to pull up biographies so that the bottom of the last one
% is flush with the other column.
%\enlargethispage{-5in}



% that's all folks
\end{document}


